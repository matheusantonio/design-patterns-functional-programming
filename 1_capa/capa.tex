\renewcommand{\imprimircapa}{%
  \begin{capa}%
    \center
    \ABNTEXchapterfont\Large \imprimirinstituicao

    \vspace*{1cm}

    {\ABNTEXchapterfont\large\imprimirautor}


    \vfill

    \begin{center}
      \ABNTEXchapterfont\bfseries\LARGE\imprimirtitulo
    \end{center}

    \vfill


    

    \large\imprimirlocal

    \large\imprimirdata

    \vspace*{1cm}

  \end{capa}
}



\makeatletter
\renewcommand{\folhaderostocontent}{

\begin{center}
  \center
    \ABNTEXchapterfont\Large \imprimirinstituicao

    \vspace*{1cm}

    {\ABNTEXchapterfont\large\imprimirautor}


  \vspace*{\fill}\vspace*{\fill}

  \begin{center}
    \ABNTEXchapterfont\bfseries\Large\imprimirtitulo
  \end{center}

  \vspace*{\fill}

  \abntex@ifnotempty{\imprimirpreambulo}{%    
 
    \hspace{.45\textwidth}

    \begin{flushright}  
      \begin{minipage}{.5\textwidth}
        \SingleSpacing
        \imprimirpreambulo
      \end{minipage}%
    \end{flushright}

    \vspace*{\fill}
  }%


  

  {\large\imprimirorientadorRotulo~\imprimirorientador\par}


  \vspace*{\fill}

  {\large\imprimirlocal}

  \par

  {\large\imprimirdata}

  \vspace*{1cm}

\end{center}
}

\makeatother








% ---
% Capa
% ---
\imprimircapa
% ---

% ---
% Folha de rosto
% (o * indica que haverá a ficha bibliográfica)
% ---
\imprimirfolhaderosto*
% ---

% ---
% Inserir a ficha bibliografica
% ---

% Isto é um exemplo de Ficha Catalográfica, ou ``Dados internacionais de
% catalogação-na-publicação''. Você pode utilizar este modelo como referência. 
% Porém, provavelmente a biblioteca da sua universidade lhe fornecerá um PDF
% com a ficha catalográfica definitiva após a defesa do trabalho. Quando estiver
% com o documento, salve-o como PDF no diretório do seu projeto e substitua todo
% o conteúdo de implementação deste arquivo pelo comando abaixo:
%
% \begin{fichacatalografica}
%     \includepdf{fig_ficha_catalografica.pdf}
% \end{fichacatalografica}

%\begin{fichacatalografica}
%	\sffamily
%	\vspace*{\fill}					% Posição vertical
%	\begin{center}					% Minipage Centralizado
%	\fbox{\begin{minipage}[c][8cm]{13.5cm}		% Largura
%	\small
%	\imprimirautor
%	%Sobrenome, Nome do autor
%	
%	\hspace{0.5cm} \imprimirtitulo  / \imprimirautor. --
%	\imprimirlocal, \imprimirdata-
%	
%	\hspace{0.5cm} \thelastpage p. : il. (algumas color.) ; 30 cm.\\
%	
%	\hspace{0.5cm} \imprimirorientadorRotulo~\imprimirorientador\\
%	
%	\hspace{0.5cm}
%	\parbox[t]{\textwidth}{\imprimirtipotrabalho~--~\imprimirinstituicao,
%	\imprimirdata.}\\
%	
%	\hspace{0.5cm}
%		1. Palavra-chave1.
%		2. Palavra-chave2.
%		2. Palavra-chave3.
%		I. Orientador.
%		II. Universidade xxx.
%		III. Faculdade de xxx.
%		IV. Título 			
%	\end{minipage}}
%	\end{center}
%\end{fichacatalografica}
% ---


% ---
% Inserir folha de aprovação
% ---

% Isto é um exemplo de Folha de aprovação, elemento obrigatório da NBR
% 14724/2011 (seção 4.2.1.3). Você pode utilizar este modelo até a aprovação
% do trabalho. Após isso, substitua todo o conteúdo deste arquivo por uma
% imagem da página assinada pela banca com o comando abaixo:
%
%\begin{folhadeaprovacao}
%\includepdf{folhadeaprovacao_final.pdf}
%\end{folhadeaprovacao}

\begin{folhadeaprovacao}

  \begin{center}
    {\ABNTEXchapterfont\large\imprimirautor}

    \vspace*{\fill}\vspace*{\fill}
    \begin{center}
      \ABNTEXchapterfont\bfseries\Large\imprimirtitulo
    \end{center}
    \vspace*{\fill}
    
    \hspace{.45\textwidth}
    \begin{minipage}{.5\textwidth}
        \imprimirpreambulo
    \end{minipage}%
    \vspace*{\fill}
   \end{center}
        
   Trabalho aprovado. \imprimirlocal, 15 de setembro de 2021:

   \assinatura{\textbf{\imprimirorientador} \\ Orientador} 
   %\assinatura{\textbf{Professor} \\ Convidado 1}
   %\assinatura{\textbf{Professor} \\ Convidado 2}
   %\assinatura{\textbf{Professor} \\ Convidado 3}
   %\assinatura{\textbf{Professor} \\ Convidado 4}
      
   \begin{center}
    \vspace*{0.5cm}
    {\large\imprimirlocal}
    \par
    {\large\imprimirdata}
    \vspace*{1cm}
  \end{center}
  
\end{folhadeaprovacao}
% ---

% ---
% Dedicatória
% ---
%\begin{dedicatoria}
%   \vspace*{\fill}
%   \centering
%   \noindent
%   \textit{} \vspace*{\fill}
%\end{dedicatoria}
% ---

% ---
% Agradecimentos
% ---
%\begin{agradecimentos}
%Os agradecimentos principais são direcionados à Gerald Weber, Miguel Frasson,
%Leslie H. Watter, Bruno Parente Lima, Flávio de Vasconcellos Corrêa, Otavio Real
%Salvador, Renato Machnievscz\footnote{Os nomes dos integrantes do primeiro
%projeto abn\TeX\ foram extraídos de
%\url{http://codigolivre.org.br/projects/abntex/}} e todos aqueles que
%contribuíram para que a produção de trabalhos acadêmicos conforme
%as normas ABNT com \LaTeX\ fosse possível.
%
%Agradecimentos especiais são direcionados ao Centro de Pesquisa em Arquitetura
%da Informação\footnote{\url{http://www.cpai.unb.br/}} da Universidade de
%Brasília (CPAI), ao grupo de usuários
%\emph{latex-br}\footnote{\url{http://groups.google.com/group/latex-br}} e aos
%novos voluntários do grupo
%\emph{\abnTeX}\footnote{\url{http://groups.google.com/group/abntex2} e
%\url{http://www.abntex.net.br/}}~que contribuíram e que ainda
%contribuirão para a evolução do \abnTeX.
%
%\end{agradecimentos}
% ---

% ---
% Epígrafe
% ---
%\begin{epigrafe}
%    \vspace*{\fill}
%	\begin{flushright}
%		\textit{``Não vos amoldeis às estruturas deste mundo, \\
%		mas transformai-vos pela renovação da mente, \\
%		a fim de distinguir qual é a vontade de Deus: \\
%		o que é bom, o que Lhe é agradável, o que é perfeito.\\
%		(Bíblia Sagrada, Romanos 12, 2)}
%	\end{flushright}
%\end{epigrafe}
% ---


% ---
% RESUMOS
% ---

% resumo em português
\setlength{\absparsep}{18pt} % ajusta o espaçamento dos parágrafos do resumo
\begin{resumo}
  O presente trabalho tem como objetivo analisar 
  um conjunto de padrões de projeto no contexto do 
  paradigma de programação funcional. Os padrões 
  de projeto apresentam soluções tradicionais para problemas 
  comuns de \textit{design} de \textit{software}, 
  destacando-se os vinte e três padrões 
  \textit{Gang of Four}, que apresentam soluções 
  voltadas para o desenvolvimento de \textit{software} 
  orientado a objetos. Porém, como a forma 
  de construir um \textit{software} difere muito 
  do paradigma funcional para o orientado a 
  objetos, existe a dúvida de como ou se os 
  problemas apresentados pelos padrões podem ser 
  solucionados com o uso de uma linguagem 
  funcional. Dessa forma, o trabalho analisará, 
  do ponto de vista funcional, cada um dos 
  vinte e três padrões \textit{Gang of Four}, 
  verificando se o problema de orientação a objetos 
  proposto também existe no contexto funcional e 
  como pode ser resolvido no mesmo. Ao 
  fim, deseja-se concluir se o uso dos recursos de 
  programação funcional pode contribuir para a 
  solução de cada padrão e, caso a conclusão não 
  seja a mesma para todos os padrões, 
  identificar características que possibilitem 
  agrupar as soluções encontradas.

 \textbf{Palavras-chave}: padrões de projeto. programação funcional.
\end{resumo}

\begin{comment}
  O presente trabalho tem como objetivo analisar 
  o conceito de padrões de projeto no contexto do 
  paradigma de programação funcional. Os padrões 
  de projeto apresentam soluções comuns para problemas 
  comuns de design de software, destacando-se os 
  vinte e três padrões Gang of Four, que apresentam 
  soluções comuns para problemas relacionados ao 
  paradigma orientado a objetos. Porém, como a forma 
  de construir um software difere muito do paradigma 
  funcional para o orientado a 
  objetos, existe a dúvida de como ou se esses padrões 
  podem ser reaproveitados, além da possibilidade de o 
  uso do paradigma funcional solucionar os problemas 
  oriundos da orientação a objetos.
  Dessa forma, o trabalho buscará analisar, 
  do ponto de vista funcional, cada um dos 
  23 padrões GoF, verificando se o problema de 
  orientação a objetos proposto também existe 
  no contexto funcional e se é resolvido pelo 
  padrão em questão. Também serão analisados, se 
  existirem, os casos em que o problema deixa de 
  existir ou é solucionado de outra forma. Ao 
  fim, deseja-se concluir se 
  o uso dos recursos de programação funcional 
  contribui para a solução de cada padrão GoF 
  e, caso a conclusão não seja a mesma para 
  todos os padrões, tentar identificar as 
  características de cada grupo.


 \textbf{Palavras-chave}: padrões de projeto. programação funcional.
\end{comment}

% resumo em inglês
\begin{resumo}[Abstract]
 \begin{otherlanguage*}{english}
  The present work aims to analyze a set of 
  design patterns in a functional programming 
  paradigm context. The design patterns present 
  traditional solutions for common software design 
  problems, standing out the twenty three Gang 
  of Four patterns, which present solutions 
  aimed at the object oriented software 
  development. However, since the way of building 
  a software differs a lot between a functional 
  paradigm and an object oriented paradigm, there is 
  the question of how or whether the problems 
  presented by the patterns can be solved with 
  a functional language. In this way, the work will 
  seek to analyze, from the functional point of view, 
  the twenty three Gang of Four patterns, verifying if 
  the proposed object orientation problem also 
  exists in the functional context and how it could 
  be solved by it. At the end, it is wished to 
  conclude if the use of functional programming 
  resources can add to the solution of each 
  pattern and, if the conclusion isn't the same 
  for all patterns, identify characteristics 
  that enable a grouping of the found solutions.

   \vspace{\onelineskip}
 
   \noindent 
   \textbf{Keywords}: design patterns. functional programming.
 \end{otherlanguage*}
\end{resumo}


% ---
% inserir lista de ilustrações
% ---
\pdfbookmark[0]{\listfigurename}{lof}
\listoffigures*
\cleardoublepage
% ---

% ---
% inserir lista de quadros
% ---
\pdfbookmark[0]{\listofquadrosname}{loq}
\listofquadros*
\cleardoublepage
% ---

% ---
% inserir lista de listings
% ---
\pdfbookmark[0]{\lstlistlistingname}{lol}
\begin{KeepFromToc}
\lstlistoflistings
\end{KeepFromToc}
\cleardoublepage
% ---

% ---
% inserir lista de abreviaturas e siglas
% ---
%\begin{siglas}
%	\item[GOF] Gang of Four
%  \end{siglas}
  % ---
  
  % ---
  % inserir lista de símbolos
  % ---
  %\begin{simbolos}
  %	\item[$ \Gamma $] Letra grega Gama
  %	\item[$ \Lambda $] Lambda
  %	\item[$ \zeta $] Letra grega minúscula zeta
  %	\item[$ \in $] Pertence
  %\end{simbolos}
  % ---
  
  % ---
  % inserir o sumario
  % ---
  \pdfbookmark[0]{\contentsname}{toc}
  \tableofcontents*
  \cleardoublepage
  % ---
