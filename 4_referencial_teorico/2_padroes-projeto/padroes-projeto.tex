% ---
% Capitulo de revisão de literatura
% ---
\chapter{Padrões de Projeto}
% ---

Durante o processo de desenvolvimento de software, 
problemas de design são comuns durante a 
fase de projeto. Alguns desses problemas em específico 
são tão comuns que foi feito um esforço catalogá-los em 
um livro junto a soluções simples e reutilizáveis para 
os mesmos. Essas soluções tornaram-se conhecidas como 
os Padrões de Projeto Gang of Four (GoF).

Com foco no design orientado a objetos, hoje estão 
entre os padrões de 
projeto mais conhecidos. Os responsáveis por compilá-los  
foram Erich Gamma, Richard Helm, 
Ralph Johson e John Vlissides. Ao todo, vinte e três 
padrões foram catalogados, os mesmos serão o alvo desse 
trabalho.

De acordo com (o livro), um padrão possui quatro elementos 
essenciais: Um nome, um problema, uma solução e suas 
consequências. O nome é uma característica importante 
por tornar mais fácil referenciar um padrão. O problema 
descreve a situação em que o padrão é aplicado enquanto 
a solução descreve como um conjunto de elementos pode 
resolver o problema proposto. Já as consequências 
mostram as vantagens e desvantagens do uso do padrão 
para um problema.

Como forma de organizar os padrões, (o livro) os separa 
por finalidade e por escopo. A separação por finalidade 
divide os padrões entre padrões criacionais, 
destinados ao processo de criação de objetos, padrões 
estruturais, que lidam com a forma em que o conjunto de 
classes e objetos está disposto e padrões comportamentais, 
focados na forma em que classes e objetos comunicam-se 
e distribuem suas responsabilidades. A separação por 
escopo divide os padrões no escopo de classe ou de objeto, 
onde o primeiro lida com a relação entre classes e 
subclasses através de herança, enquanto o segundo lida 
com formas de relacionamento mais dinâmicas entre os 
objetos, como delegação. Os padrões nesse trabalho 
serão separados apenas por finalidade, porém 
características relacionadas à separação por escopo 
podem ser mencionadas durante a análise de certos 
padrões.

A descrição de cada padrão (no livro) segue uma estrutura 
muito similar, utilizada principalmente para apresentar 
os quatro elementos essenciais mencionados anteriormente. 
Entre elas, o nome, cujo propósito já foi mencionado, e 
a classificação, descrita no parágrafo anterior, além de 
outros nomes pelos quais o padrão pode ser referenciado. 
Também são descritas a intenção do padrão e seu objetivo, 
como o que o padrão realiza e qual problema busca 
solucionar. Em seguida, é descrito um cenário que 
ilustra o problema solucionado pelo padrão, e as 
situações nas quais ele pode ser aplicado. Sua estrutura, 
representada através de um diagrama de classes, é 
acompanhada de uma descrição de cada classe ou objeto 
que participa do padrão, incluindo como cada um 
colabora para a solução do padrão. As consequências do 
uso do padrão também são apresentadas, com vantagens e 
desvantagens decorrentes de seu uso, além de sugestões de 
técnicas de implementação e se existem considerações 
específicas para uma determinada linguagem. É 
demonstrado um exemplo 
em código ilustrando como utilizar o padrão e exemplos 
de usos em sistemas reais. Por fim, são enumerados 
outros padrões que podem estar relacionados e como se dá 
esse relacionamento.

