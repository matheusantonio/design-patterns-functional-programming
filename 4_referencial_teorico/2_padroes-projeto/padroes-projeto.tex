% ---
% Capitulo de revisão de literatura
% ---
\chapter{Padrões de Projeto}
% ---

Durante o processo de desenvolvimento de software, 
problemas de design são comuns durante a 
fase de projeto. Alguns desses problemas 
eram tão comuns que foi feito um esforço catalogá-los em 
um livro que oferece possíveis soluções. 
Essas soluções tornaram-se conhecidas como 
os Padrões de Projeto Gang of Four, abreviados 
para GoF.

Por definição, um padrão de projeto é uma solução 
reutilizável para um problema comum de design. Apesar 
de neste trabalho ser utilizada no contexto de engenharia 
de software, o conceito foi introduzido pelo arquiteto 
Christopher Alexander no livro A Pattern Language.

Com foco no design orientado a objetos, hoje os 
padrões GoF estão entre os padrões de 
projeto de software mais conhecidos. Os responsáveis 
por compilá-los foram Erich Gamma, Richard Helm, 
Ralph Johson e John Vlissides, o que deu origem ao 
nome Gang of Four. Ao todo, vinte e três 
padrões foram catalogados, os mesmos serão o alvo deste 
trabalho.

De acordo com (o livro), um padrão possui quatro elementos 
essenciais: Um nome, um problema, uma solução e suas 
consequências. O nome é uma característica importante 
por tornar mais fácil referenciar um padrão, o problema 
descreve a situação em que o padrão é aplicado e 
a solução descreve como um conjunto de elementos pode 
resolver o problema proposto. Já as consequências 
mostram as vantagens e desvantagens do uso do padrão 
para um problema.

Como forma de organizar os padrões, (o livro) os separa 
por finalidade e por escopo. A separação por finalidade 
divide os padrões entre padrões criacionais, 
destinados ao processo de criação de objetos, padrões 
estruturais, que lidam com a forma em que o conjunto de 
classes e objetos está disposto e padrões comportamentais, 
focados na forma em que classes e objetos comunicam-se 
e distribuem suas responsabilidades. A separação por 
escopo divide os padrões no escopo de classe ou de objeto, 
onde o primeiro lida com a relação entre classes e 
subclasses através de herança, enquanto o segundo lida 
com formas de relacionamento mais dinâmicas entre os 
objetos, como delegação. Os padrões nesse trabalho 
serão separados apenas por finalidade, porém 
características que remetem ao escopo 
podem ser consideradas durante a análise de alguns 
padrões.

A descrição de cada padrão (no livro) segue uma estrutura 
muito similar, utilizada principalmente para apresentar 
os quatro elementos essenciais mencionados anteriormente. 
Entre elas, o nome dado pelo livro e a classificação, 
descrita no parágrafo anterior, além de 
outros nomes pelos quais o padrão pode ser referenciado. 
Também são descritos a intenção do padrão e seu objetivo, 
como o que é realizado pelo padrão e qual problema é 
solucionado. Em sequência, é descrito um cenário que 
ilustra o problema solucionado seguido das 
situações nas quais ele pode ser aplicado. Sua estrutura, 
representada através de um diagrama de classes, é 
acompanhada de uma descrição de cada classe ou objeto 
que participa do padrão, incluindo como cada um 
colabora para a solução do padrão. As consequências do 
uso do padrão também são apresentadas, com vantagens e 
desvantagens decorrentes de seu uso, além de sugestões de 
técnicas de implementação e a existência de considerações 
específicas para determinadas linguagens de programação. 
É demonstrado um exemplo 
em código ilustrando como utilizar o padrão seguido de 
referências de uso em sistemas reais. Por fim, são 
enumerados outros padrões que podem estar relacionados 
e como ocorre esse relacionamento.

Neste trabalho, uma descrição mais sucinta de cada 
padrão será apresentada durante o desenvolvimento, onde 
serão apresentados apenas os elementos essenciais para 
a análise do padrão a partir do paradigma funcional.