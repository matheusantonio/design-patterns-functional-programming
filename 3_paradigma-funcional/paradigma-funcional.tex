% ---
% Capitulo de revisão de literatura
% ---
\chapter{O Paradigma Funcional}
% ---

Para que uma linguagem seja considerada funcional, existem 
algumas funcionalidades que ela deve implementar, assim 
como características que ela não deve possuir. Algumas dessas 
características serão exploradas a seguir. É importante 
lembrar que o fato de uma linguagem não prevenir contra 
algumas dessas características desencorajadas não significa 
que elas não podem ser evitadas. Por mais que uma linguagem 
não seja implementada baseada em um determinado paradigma, 
nada impede que as características e convenções dele sejam 
seguidos como boas práticas. Também é comum linguagens que 
não são exatamente funcionais implementarem parte desses 
recursos. É comum esse tipo de linguagem ser referida como 
multiparadigma, mesmo que um determinado paradigma se 
sobressaia sobre os demais.
Também é importante ressaltar, apesar de isso ser 
evidente tanto pelo que foi descrito no parágrafo anterior 
quanto pela linguagem utilizada no decorrer deste trabalho, 
que o fato de alguns paradigmas apresentarem características 
muito diferentes ou até mutuamente exclusivas, nada impede 
que mais de um paradigma seja usado de forma complementar 
durante a construção de um programa. O ideal é que a melhor 
solução seja usada para o problema proposto, independente 
do paradigma utilizado.


\section{Funções Puras e Efeitos Colaterais}

É comum entre os paradigmas imperativo ou orientado a objetos 
o uso de operações que consultam tabelas de um banco de dados 
ou escrevem um valor em um arquivo. Existe algo em comum 
entre todas essas operações: os efeitos colaterais. 

Quando uma função acessa dados externos à aplicação ou 
mesmo ao seu escopo, é difícil prever o que pode acontecer. 
Normalmente, medidas como tratamento de exceções são tomadas 
para interromper uma tentativa de acesso mau sucedida cuja 
causa não pode ser reparada pelo programa (por exemplo, 
o usuário fornecer um nome para um arquivo que não existe e 
o programa tenta acessá-lo). Esse comportamento, por mais 
que inevitável, faz com que uma função acabe nem sempre se 
comportando da forma que ela deveria: Esse tipo de função é 
conhecida como impura.

Partindo da definição de uma função impura, uma função pura 
pode ser definida como uma que opera apenas sobre os valores 
passados a ela como parâmetro. Ou seja, ela não realiza 
nenhuma operação que possa causar um efeito colateral 
inesperado que quebre o propósito para o qual a função foi 
construída.

Uma propriedade importante de funções puras é que, 
independente de quantas vezes uma função for executada, 
para os mesmos parâmetros de entrada, a saída será sempre 
a mesma. Essa é uma propriedade muito útil para a realização 
de testes e para isolar erros em um programa, tornando muito 
mais simples o processo de debug. Isso seria impossível para 
funções que lidam com efeitos colaterais, onde uma função 
pode retornar um valor incorreto independente de estar se 
comportando da forma correta ou não.

É importante notar que um programa que possui apenas funções 
puras é praticamente inviável, já que é constantemente 
necessário realizar operações que dependem de acessos 
externos ao programa. O paradigma funcional orienta, porém, 
que esses efeitos colaterais sejam isolados na aplicação. 
Dessa forma, as funções que realizam efeitos colaterais como 
recuperar dados de uma API ou atualizar uma base de dados 
devem ser executadas no início e no fim de uma execução, 
preservando o máximo de pureza possível.

\section{Imutabilidade}

Em orientação a objetos, dados costumam ser encapsulados em 
objetos, onde métodos de acesso podem recuperá-los ou 
modificá-los. A ideia de modificar um valor é desencorajada 
no paradigma funcional, ou seja, não existem variáveis.

Em um programa funcional, se um nome x em um determinado 
escopo recebe um valor, é impossível assosciar a x um 
valor diferente: esse é o conceito de imutabilidade. O 
problema é que é comum que um determinado valor seja 
modificado no decorrer de um programa. Por exemplo, uma 
estrutura que armazene um 

\section{Funções de Alta Ordem}

Apesar de ser um recurso não tão incomum, funções de alta 
ordem são importantes para definir uma linguagem funcional. 
Em linguagens que implementam o paradigma funcional, funções 
costumam ser tratadas como tipos, da mesma forma que inteiros, 
caracteres e booleanos. Dessa forma, uma função pode também 
ser considerada um valor que possui um tipo que depende de 
outros tipos, da mesma forma que uma lista é um tipo que 
depende do tipo de dado armazenado.

Essa propriedade é importante pois permite que funções 
sejam também passadas como parâmetros por outras funções e 
serem retornadas por outras funções. E essa é exatamente a 
definição de uma função de alta ordem: É uma função que 
pode receber funções como parâmetro e retornar funções (ou 
ambos).

\section{Monads}

Monad é considerado um conceito difícil da programação 
funcional por ser definido através de leis matemáticas 
que podem ser complexas. Iniciando pelo básico, um Monad 
é uma forma de estruturar e combinar sequências de 
operações em um programa funcional. Algo que é alcançado 
de forma simples em um paradigma imperativo ou orientado 
a objetos acaba sendo teoricamente mais difícil em um 
paradigma funcional graças aos outros conceitos de 
programação funcional que impedem ou desencorajam que 
operações sejam simplesmente executadas sequencialmente.

Normalmente, um Monad encapsula um valor.
