\section{Singleton}

O padrão Singleton garante que um objeto possuirá apenas uma 
instância. Além disso, fornece um único ponto, acessível 
globalmente, a essa instância. Essa implementação é útil 
para implementar classes que fornecem serviços sem que seja 
necessário instanciar vários objetos idênticos em 
locais diferentes do código.

A figura \ref{singleton_struct} demonstra a implementação 
do padrão. A classe Singleton possui um método construtor 
privado e armazena no atributo estático uniqueInstance uma 
instância de Singleton. Através do método de classe 
Instance, é verificado se já existe uma instância 
armazenada no atributo uniqueInstance. Caso já exista, 
ela é retornada. Caso não, a instância única é criada 
para ser retornada nas chamadas posteriores de Instance.

\begin{figure}[htb]
	\caption{\label{singleton_struct}Estrutura do Singleton}
	\begin{center}
	    \includegraphics[scale=0.6]{5_padroes-contexto-funcional/5.1_criacionais/5.1.5_singleton/singleton_estrutura.png}
	\end{center}
\end{figure}

\subsection*{Exemplo Orientado a Objetos}

Uma classe define as operações para realizar transações com 
uma base de dados. Como a instância dela é idêntica independente 
do cliente que a utiliza, não existe a necessidade de replicar 
essas instâncias pelo código. Ela pode ser transformada em 
um Singleton, o que faz com que toda classe que deseja fazer 
uma transação na base de dados apenas solicite uma instância 
e realize as operações. A definição da classe do exemplo 
pode ser vista na figura \ref{singleton_exemplo} e no 
código \ref{oosingleton}.

\begin{figure}[htb]
	\caption{\label{singleton_exemplo}Exemplo de Singleton}
	\begin{center}
	    \includegraphics[scale=0.6]{5_padroes-contexto-funcional/5.1_criacionais/5.1.5_singleton/singleton_exemplo.png}
	\end{center}
\end{figure}

\begin{lstlisting}[caption={Singleton Orientação a Objetos},label=oosingleton]

class Database private(){
  def Query(sql : String) : Object = {
    //Execute query
    null
  }
  def Command(sql : String) : Unit = {
    //Execute command
  }
}

object Database {
  private var instance : Database = null

  def Instance() : Database = {
    if(instance == null){
      instance = new Database()
    }
    instance
  }
}

\end{lstlisting}

\subsection*{Contexto Funcional}

Em um contexto sem classes e objetos, 
um singleton poderia ser considerado como 
uma variável ou função global acessível a 
todo o programa. Essa ideia viola o conceito 
de função pura, já que uma função que acessa 
um singleton deixa de depender apenas de seus 
parâmetros. Para o caso em que o Singleton 
armazene algum estado, o conceito de imutabilidade 
também precisaria ser violado, já que o 
valor que armazena o Singleton precisaria ser 
mutável para ser alterado de dentro de uma 
função. Com isso, não existe uma implementação 
equivalente ao Singleton no contexto funcional.