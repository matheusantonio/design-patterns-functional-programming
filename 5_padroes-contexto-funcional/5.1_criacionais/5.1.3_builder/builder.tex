\section{Builder}

Quando é necessário criar objetos  
complexos, o padrão \textit{Builder} retira a 
responsabilidade de criação do objeto a 
ser criado e a coloca em classes separadas, 
que constroem partes do objeto. 
Isso permite que um mesmo processo de criação 
possa gerar representações diferentes de um mesmo 
objeto.\cite{gamma:1995}

A Figura \ref{builder_struct} apresenta 
a estrutura do \textit{Builder}, onde a interface 
\texttt{Builder} define uma operação para criar uma 
parte do objeto. A classe \texttt{ConcreteBuilder}, que 
implementa a interface \texttt{Builder}, 
implementa os métodos de construção de 
uma parte do tipo \texttt{Product}. A classe \texttt{Director} 
é responsável por executar a criação das 
partes do objeto, através da operação 
\texttt{Construct}.

\begin{figure}[htb]
	\caption{\label{builder_struct}Estrutura do \textit{Builder}.}
	\begin{center}
	    \includegraphics[scale=0.5]{5_padroes-contexto-funcional/5.1_criacionais/5.1.3_builder/builder_estrutura.png}
	\end{center}
  \caption*{Fonte: O Autor (2021)}
\end{figure}

\subsection*{Exemplo Orientado a Objetos}

Como exemplo, é possível observar um leitor de 
documentos do tipo \textit{RTF} (\textit{Rich Text Format}), 
que deve permitir a conversão de documentos \textit{RTF} 
para outros formatos, como texto em ASCII ou em um 
\textit{widget} de texto que pode ser editado de 
forma interativa. Como a quantidade de formatos 
possíveis é grande, deve ser possível adicionar 
novos formatos sem que seja necessário modificar 
a classe do leitor de documentos \textit{RTF}. 

O diagrama de classes apresentado na Figura 
\ref{builder_exemplo} demonstra o uso do padrão 
\textit{Builder} para esse exemplo. Para cada formato possível 
de conversão, uma nova classe \textit{Builder} é criada. 
As classes \texttt{ASCIIConverter}, \texttt{TeXConverter} e 
\texttt{TextWidgetConverter} representam, respectivamente, 
os \textit{builders} para os conversores para texto 
em ASCII, LaTeX e \textit{widget} de texto interativo. 
A classe \texttt{RTFReader} chama as operações de construção 
dos conversores desejados. O exemplo de 
implementação dessa abordagem é apresentado no 
Código \ref{oobuilder}.

\begin{figure}[htb]
	\caption{\label{builder_exemplo}Exemplo de \textit{Builder}.}
	\begin{center}
	    \includegraphics[scale=0.5]{5_padroes-contexto-funcional/5.1_criacionais/5.1.3_builder/builder_exemplo.png}
	\end{center}
  \caption*{Fonte: O Autor (2021)}
\end{figure}

\begin{lstlisting}[caption={\textit{Builder} Orientado a Objetos.},label=oobuilder]

trait TextConverter {
  def ConvertCharacter(char : Char)
  def ConvertFontChange(font : String)
  def ConvertParagraph()
} 

class ASCIIConverter(val asciiText: ASCIIText) extends TextConverter {

  def ConvertCharacter(char : Char): Unit = {
    //Converter char
  }

  def ConvertFontChange(font : String): Unit = {
    //Converter fonte
  }

  def ConvertParagraph(): Unit = {
    //Converte parágrafo
  }
}

class TeXConverter(val texText : TeXText) extends TextConverter {
  def ConvertCharacter(char: Char): Unit = {
    //Converter char
  }

  def ConvertFontChange(font : String): Unit = {
    //Converter fonte
  }

  def ConvertParagraph(): Unit = {
    //Converte parágrafo
  }
}

class TextWidgetConverter(val textWidget: TextWidget) extends TextConverter {
  def ConvertCharacter(char: Char): Unit = {
    //Converter char
  }

  def ConvertFontChange(font : String): Unit = {
    //Converter fonte
  }

  def ConvertParagraph(): Unit = {
    //Converte parágrafo
  }
}

class RTFReader(var textConverter: TextConverter) {

  def SetTextConverter(textConverter: TextConverter): Unit = {
    this.textConverter = textConverter
  }

  def ParseRTF(): Unit = {
    val tokens : List[Token] = List()
    // ...
    for(t <- tokens) {
      t.Type match {
        case TokenType.CHAR => textConverter.ConvertCharacter(t.Character)
        case TokenType.FONT => textConverter.ConvertFontChange(t.Font)
        case TokenType.PARAGRAPH => textConverter.ConvertParagraph()
      }
    }
    // ...
  }
}

\end{lstlisting}
\legend{Fonte: O Autor (2021)}

\subsection*{Contexto Funcional}

É possível aproveitar as funções de alta ordem 
da programação funcional para simplificar o 
padrão \textit{Builder}. Ao invés de definir novas 
classes para cada tipo de \textit{builder}, uma função 
pode receber como 
parâmetro as operações de construção desejadas. 
No Código \ref{fpbuilder}, essa função é a 
\texttt{ParseRTFBuilder}, definida na linha 2. 
Para facilitar o reuso de tipos diferentes 
de \textit{builder}, ela retorna uma nova 
função que recebe como parâmetro uma 
lista de \textit{tokens} e 
retorna uma nova lista no formato desejado. 

A função retornada é análoga ao método 
\texttt{ParseRTF} da classe \texttt{RTFReader} do exemplo 
orientado a objetos. Com essa implementação, 
para cada variação de \textit{builder}, 
basta executar a função 
\texttt{ParseRTFBuilder} passando as operações 
desejadas. Isso pode ser visto na linha 16, 
onde o valor \texttt{ParseRTFToASCII} é o resultado 
da definição do \textit{builder} que converte 
RTF em texto no formato ASCII. Como as funções 
podem ser definidas em tempo de execução, essa 
implementação permite que os \textit{builders} 
sejam criados dinamicamente.

\begin{lstlisting}[caption={\textit{Builder} Funcional.},label=fpbuilder]
    
def ParseRTFBuilder(convertCharacter : Char => Token,
             convertFontChange : String => Token,
             convertParagraph : String => Token)
: List[Token] => List[Token] = (tokens : List[Token]) => {
  val parseRest = (tokens : List[Token]) => ParseRTFBuilder(convertCharacter, convertFontChange, convertParagraph)(tokens)
  // ...
  tokens.head.Type match {
    case TokenType.CHAR => convertCharacter(tokens.head.Character) :: parseRest(tokens.tail)
    case TokenType.FONT => convertFontChange(tokens.head.Font) :: parseRest(tokens.tail)
    case TokenType.PARAGRAPH => convertParagraph(tokens.head.Paragraph) :: parseRest(tokens.tail)
  }
  // ...
}

val ParseRTFToASCII = ParseRTFBuilder(
              ConvertASCIICharacter,
              ConvertASCIIFontChange,
              ConvertASCIIParagraph)
    
\end{lstlisting}
\legend{Fonte: O Autor (2021)}