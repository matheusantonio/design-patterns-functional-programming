\section{Bridge}

O padrão Bridge permite variar as abstrações e as 
implementações de um problema de forma independente, 
definindo uma interface que serve como ponte entre ambas. 
As operações da implementação são delegadas para essa 
nova interface, permitindo que as abstrações sejam 
implementadas sem precisar conhecer o tipo de 
implementação que está sendo utilizado. A estrutura do 
padrão pode ser vista na figura \ref{bridge_struct}.

\begin{figure}[htb]
	\caption{\label{bridge_struct}Estrutura do Bridge}
	\begin{center}
	    \includegraphics[scale=0.5]{5_padroes-contexto-funcional/5.2_estruturais/5.2.2_bridge/bridge_estrutura.png}
	\end{center}
\end{figure}

\subsection*{Exemplo Orientado a Objetos}

Como exemplo, pode ser considerada a implementação de 
uma janela em um \textit{toolkit} para construir interfaces 
de usuários, que permite o uso de implementações diferentes 
de janela: PM e XWindow. Além disso, é preciso definir tipos 
diferentes de janela, como janelas para ícones e janelas 
de transitórias. Para que não seja necessário implementar 
uma versão diferente de janela de ícone e transitória 
para cada implementação diferente de janela, o padrão 
Bridge pode ser usado para separar a implementação 
da abstração em duas hierarquias diferentes. O diagrama 
de classes da figura \ref{bridge_exemplo} e o código 
\ref{oobridge} demonstram o uso do padrão para esse 
exemplo.

\begin{figure}[htb]
	\caption{\label{bridge_exemplo}Exemplo de Bridge}
	\begin{center}
	    \includegraphics[scale=0.5]{5_padroes-contexto-funcional/5.2_estruturais/5.2.2_bridge/bridge_exemplo.png}
	\end{center}
\end{figure}

\begin{lstlisting}[caption={Bridge Orientado a Objetos},label=oobridge]

abstract class Window(imp : WindowImp) {
  def DrawText() : Unit = {
    imp.DevDrawText()
  }

  def DrawRect() : Unit = {
    imp.DevDrawLine()
    imp.DevDrawLine()
    imp.DevDrawLine()
    imp.DevDrawLine()
  }
}

class IconWindow(imp : WindowImp) extends Window(imp) {
  def DrawBorder() : Unit = {
    DrawRect()
    DrawText()
  }
}

class TransientWindow(imp : WindowImp) extends Window(imp){
  def DrawCloseBox() : Unit = {
    DrawRect()
  }
}

trait WindowImp {
  def DevDrawText()
  def DevDrawLine()
}

class XWindowImp extends WindowImp {
  def DevDrawText() : Unit = {
    //Desenha texto para janela X
  }
  def DevDrawLine() : Unit = {
    //Desenha linha para janela X
  }
}

class PMWindowImp extends WindowImp {
  def DevDrawLine(): Unit = {
    //Desenha linha para janela PM
  }
  def DevDrawText(): Unit = {
    //Desenha texo para janela PM
  }
}

\end{lstlisting}


\subsection*{Contexto Funcional}

\begin{lstlisting}[caption={Bridge Funcional},label=fpbridge]
    

    
\end{lstlisting}