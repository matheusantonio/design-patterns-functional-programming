\section{Proxy}

O padrão \textit{Proxy} adiciona uma camada de acesso 
a um objeto, fazendo com que qualquer interação 
com o objeto \textit{Proxy} seja controlada antes de ser 
repassada para o objeto real. Existe mais 
de uma motivação para seu uso, apesar 
da estrutura do padrão, 
apresentada na Figura \ref{proxy_struct},  
permanecer a mesma. Essas motivações 
podem ser divididas em três grupos de 
proxy: remoto, virtual e de proteção. \cite{gamma:1995}

Um \textit{Proxy} remoto pode ser utilizado como um 
representante de um objeto, servindo como um 
intermediário e armazenando dados que devem 
ser repassados para o objeto real. Já um 
\textit{Proxy} virtual armazena informações sobre 
um objeto de forma que ele possa ser criado 
apenas quando for necessário, economizando 
espaço ou tempo de execução. Por fim, um 
\texttt{Proxy} de proteção pode controlar o acesso 
a um objeto, exigindo que um objeto cliente 
precise cumprir algum requisito antes de 
acessar o objeto real. \cite{gamma:1995}

\begin{figure}[htb]
	\caption{\label{proxy_struct}Estrutura do \textit{Proxy}.}
	\begin{center}
	    \includegraphics[scale=0.5]{5_padroes-contexto-funcional/5.2_estruturais/5.2.7_proxy/proxy_estrutura.png}
	\end{center}
  \caption*{Fonte: O Autor (2021)}
\end{figure}

\subsection*{Exemplo Orientado a Objetos}

Como exemplo de \textit{Proxy} podem ser utilizadas 
imagens que são incluídas em editores de 
texto. É desejado que um documento seja aberto 
rapidamente e o carregamento das imagens 
presentes nele pode causar uma lentidão 
desnecessária. Para isso, é criado um objeto 
\texttt{Proxy} que armazena em seus atributos as 
informações necessárias, como o caminho 
para a imagem e seu posicionamento no 
documento. Quando o documento termina de 
ser aberto, as imagens são carregadas de 
fato através de seus \textit{proxys}. Esse é um 
exemplo de \textit{Proxy} virtual, seu diagrama de 
classes pode ser visto na Figura \ref{proxy_exemplo} 
e a implementação no Código \ref{ooproxy}.

\begin{lstlisting}[caption={\textit{Proxy} Orientado a Objetos.},label=ooproxy]

trait Graphic {
  def Draw(point : Point)
  def GetExtent() : Point
  def Store(ostream : Stream[Byte])
  def Load(istream : Stream[Byte])
}

class Image (private val filename : String) extends Graphic {

  private var extent : Point = new Point

  def Draw(point: Point): Unit = {
    //Desenha imagem na tela
  }

  def GetExtent(): Point = extent

  def Store(ostream: Stream[Byte]): Unit = {
    //Salva imagem em um arquivo
  }

  def Load(istream: Stream[Byte]): Unit = {
    //Carrega imagem de um arquivo
  }
}

class ImageProxy(var filename : String, var extent : Point) extends Graphic {

  private var image : Image = null

  override def Draw(point: Point): Unit = {
    if(image == null){
      image = new Image(filename)
    }
    image.Draw(point)
  }

  override def GetExtent(): Point = {
    if(image==null){
      image = new Image(filename)
    }
    image.GetExtent()
  }

  def Store(ostream: Stream[Byte]): Unit = {
    //Salva imagem em um arquivo
  }

  def Load(istream: Stream[Byte]): Unit = {
    //Carrega imagem de um arquivo
  }
}

\end{lstlisting}
\legend{Fonte: O Autor (2021)}

\begin{figure}[htb]
	\caption{\label{proxy_exemplo}Exemplo de \textit{Proxy}.}
	\begin{center}
	    \includegraphics[scale=0.5]{5_padroes-contexto-funcional/5.2_estruturais/5.2.7_proxy/proxy_exemplo.png}
	\end{center}
  \caption*{Fonte: O Autor (2021)}
\end{figure}

\subsection*{Contexto Funcional}

Para implementar um \textit{Proxy}, uma função do 
editor de documentos pode receber como 
parâmetro uma função cuja assinatura seja 
a mesma do método \texttt{Draw} no exemplo orientado 
a objetos. Dessa forma, tanto é possível 
receber a função real quanto a função 
intermediária, sem que a função cliente 
esteja ciente de qual está sendo utilizada. 
A função intermediária, assim como nos 
métodos da classe intermediária no exemplo 
orientado a objetos, executará a função 
real, de acordo com o tipo de \textit{Proxy} que 
deve ser implementado. 

No Código \ref{fpproxy}, a função intermediária 
\texttt{DrawImageProxy}, definida na linha 6, chama a 
função real \texttt{DrawImage}, definida na linha 2. 
A função \texttt{EditorFunction}, definida na linha 11, 
representa uma função cliente qualquer que 
precisa desenhar a imagem. Ao invés de chamar 
diretamente a função \texttt{DrawImage}, ela recebe 
como parâmetro uma função com a mesma assinatura, 
para que seja possível que ela receba a função 
\textit{Proxy}.

Existem algumas desvantagens quanto a essa 
implementação. Não é possível que haja 
estado atrelado ao \textit{Proxy}, pois para que 
esse estado seja refletido na função 
cliente, é necessário que uma nova função, 
atualizada com o novo estado do \textit{Proxy}, seja 
retornada pelas funções intermediárias. 
Isso faz com que o cliente precise gerenciar 
a mudança do \textit{Proxy}, violando 
um dos propósitos do padrão. 

\begin{lstlisting}[caption={\textit{Proxy} Funcional.},label=fpproxy]
    
def DrawImage(point : Point, image : Image) : Image = {
  // Desenha a imagem
}

def DrawImageProxy(point : Point, image : Image) : Image = {
  // Proxy realiza as operações necessárias antes de desenhar a imagem
  DrawImage(point, image)
}

def EditorFunction(filename : String, Draw : (Point, Image) => Image) : Unit = {
  //...
  img = Draw(point, img)
  //...
}
    
\end{lstlisting}
\legend{Fonte: O Autor (2021)}