\chapter{Padrões Estruturais}

\section{Adapter}

Quando a interface de uma classe, objeto ou biblioteca não 
é compatível com a interface atual do cliente que deseja 
utilizar essa interface, o padrão Adapter fornece uma 
solução que evita a refatoração e a dependência da interface 
do cliente para a interface desejada.

Existem duas formas de realizar essa adaptação. Um Adapter 
de classe (Figura \ref{adapter_struct}), que só é possível 
para linguagens que implementam herança múltipla, implementa 
uma classe que herda tanto da classe que representa a interface 
da aplicação quanto da classe que representa a interface que 
deseja ser utilizada. 

\begin{figure}[htb]
	\caption{\label{adapter_struct}Estrutura do Adapter de Classe}
	\begin{center}
	    \includegraphics[scale=0.5]{5_padroes-contexto-funcional/5.2_estruturais/5.2.1_adapter/diagram2.png}
	\end{center}
\end{figure}

Já o Adapter de Objeto (Figura \ref{adapter_alt_struct}) 
herda apenas da classe que representa 
a interface da aplicação e reimplementa a operação desejada 
de forma que, após adaptar para a operação para a interface 
desejada, delega a realização da mesma para um objeto que 
implemente essa interface.

\begin{figure}[htb]
	\caption{\label{adapter_alt_struct}Estrutura do Adapter de Objeto}
	\begin{center}
	    \includegraphics[scale=0.4]{5_padroes-contexto-funcional/5.2_estruturais/5.2.1_adapter/diagram.png}
	\end{center}
\end{figure}

\subsection*{Exemplo Orientado a Objetos}

Como exemplo é apresentada uma ferramenta gráfica 
que permite a edição de diversos objetos gráficos 
simples, entre eles linhas e polígonos. Porém, 
a aplicação deseja também editar elementos textuais, 
que são mais complexos de se gerenciar. Como já existem 
ferramentas prontas para gerenciar esse tipo de 
elemento, é desejado reutilizá-lo. Já que as 
ferramentas prontas não foram feitas pensando na 
aplicação de ferramenta gráfica do exemplo, uma 
classe Adapter é implementada para adaptar a 
ferramenta textual para a aplicação que deseja 
utilizá-la. Para esse exemplo, é utilizada a 
abordagem de Adapter de objeto, onde um objeto 
do tipo da ferramenta textual é armazenado. O 
diagrama de classes do exemplo pode ser visto na 
figura \ref{adapter_exemplo}, enquanto a 
implementação em código pode ser vista no código 
\ref{ooadapter}.


\begin{figure}[htb]
	\caption{\label{adapter_exemplo}Exemplo de Adapter}
	\begin{center}
	    \includegraphics[scale=0.45]{5_padroes-contexto-funcional/5.2_estruturais/5.2.1_adapter/exemplo_adapter.png}
	\end{center}
\end{figure}


\begin{lstlisting}[caption={Adapter Orientado a Objetos},label=ooadapter]

class Shape(
	var bottomLeft : Point, 
	var topRight : Point
) {

}

class Line() extends Shape {

}

class TextShape(
	textView : TextView,
	var bottomLeft : Point, 
	var topRight : Point) extends Shape {

}

class TextView(
	var x : Coord, var y : Coord,
	var width : Coord : var height : Coord
) {


}


\end{lstlisting}



\subsection*{Contexto Funcional}

\begin{comment}
Existem duas formas simples de implementar um Adapter em uma linguagem 
funcional: usando funções de alta ordem e composição de funções.

Através de funções de alta ordem é possível passar por parâmetro, 
quando necessário, uma função que adapta o valor definido no cliente 
para o valor que precisa ser recebido pela função incompatível. O 
problema dessa abordagem é a necessidade do cliente conhecer a função 
Adapter e a biblioteca.

Já com composição de funções, uma função composta da função Adapter 
e da função incompatível é fornecida para o cliente, que sem precisar 
saber que está usando um Adapter, pode realizar a operação incompatível 
sem problemas.
\end{comment}

\begin{lstlisting}[caption={Adapter Funcional},label=fpadapter]
    

    
\end{lstlisting}