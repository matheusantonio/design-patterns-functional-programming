
% ---
% Capitulo de revisão de literatura
% ---
\chapter{Padrões de Projeto no Contexto Funcional}
% ---

O conceito de objeto não existe no paradigma funcional. 
Para ater-se a não reaproveitar recursos e conceitos 
oriundos da Orientação a Objetos nos exemplos que 
utilizam os recursos e conceitos oriundos da 
programação funcional, será usada uma estrutura 
equivalente quando for necessário o uso de alguma 
estrutura semelhante a um objeto.
Um objeto pode ser definido como uma representação 
do mundo real que possui características (atributos) 
e comportamentos (métodos). Para representar as 
características, será utilizado o recurso case 
class de scala. Já os comportamentos serão definidos 
por funções que recebem como entrada um valor do 
case class criado e retorna uma nova variável do 
mesmo tipo case class ou o valor de alguma de suas 
características. Por exemplo, a classe a seguir, 
construída a partir do paradigma orientado a objetos:



\begin{lstlisting}[caption={Classe comum em Orientação a Objetos},label=ooclass]
    
    class Person(var name : String, var age : Int){

        def getName() : String = this.name

        def setName(name : String) : Unit = this.name = name

        def getAge() : Int = this.age

        def setAge(age : Int) : Unit = this.age = age

    }   

\end{lstlisting}



Pode ser representada da seguinte forma no paradigma funcional:

\begin{lstlisting}[caption={Representação de uma classe no contexto funcional},label=pfclass]
    
    case class Person(name: String, age: Int)

    def getName(person : Person) : String = person.name

    def setName(person : Person, name : String) : Person = 
        person.copy(name = name)

    def getAge(person : Person) : Int = person.age

    def setAge(person : Person, age : Int) : Person =
        person.copy(age = age)

\end{lstlisting}