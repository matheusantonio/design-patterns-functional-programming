
% ---
% Capitulo de revisão de literatura
% ---
\chapter{Orientação a Objetos no Contexto Funcional}
% ---

% Conceitos para mapear

Parte dos padrões de projeto que serão 
analisados dependem de conceitos 
de orientação a objetos como classes ou 
encapsulamento, o que torna necessário 
realizar um mapeamento desses conceitos 
para o paradigma funcional. A intenção 
desse mapeamento não é implementar 
orientação a objetos em uma linguagem 
funcional, mas entender qual é a utilidade 
de cada um desses conceitos e de que 
forma a programação funcional pode 
oferecer essa mesma utilidade. É importante 
ressaltar que esse mapeamento não será usado 
de forma metódica ao analisar os padrões. 
Ele é uma referência para contextualizar 
o leitor nos exemplos em código 
implementados de forma funcional. 


% classes e objetos
\section{Classes e Objetos}

Um objeto pode ser definido como uma representação 
do mundo real que possui características e comportamentos, 
enquanto uma classe é uma abstração dessa representação 
que define quais características e comportamentos um objeto 
deve possuir\cite{umlsystems}. Essas características 
e comportamentos são representados em orientação a 
objetos como atributos e métodos, respectivamente. 
O código \ref{ooclass} demonstra uma classe que 
possui os atributos nome e idade, além dos métodos 
getNome, setNome, getIdade e setIdade, que realizam 
operações com esses atributos.

\begin{lstlisting}[caption={Classe comum em Orientação a Objetos},label=ooclass]
    
    class Pessoa(var nome : String, var idade : Int){

        def getNome() : String = this.nome

        def setNome(nome : String) : Unit = this.nome = nome

        def getIdade() : Int = this.idade

        def setIdade(idade : Int) : Unit = this.idade = idade

    }   

\end{lstlisting}

Dessa forma, é necessário definir uma estrutura em 
programação funcional que possua características e 
funções que operam sobre essas características. 
Para agrupar características pode ser utilizada uma 
tupla, uma estrutura que armazena uma quantidade 
fixa de valores com tipos predefinidos\cite{tuplesscala}. 
Como as tuplas não podem ser modificadas, elas 
respeitam o conceito de imutabilidade das 
linguagens funcionais.

Para representar os métodos de uma classe em uma 
linguagem funcional, já que nossa estrutura de dados 
imutável não armazenará funções\footnote{Apesar de não 
ser uma abordagem utilizada para este mapeamento, é 
possível armazenar funções em tuplas.} e já 
que é necessário que nossas funções sejam puras, 
uma abordagem de implementação desses 
métodos é definir funções que recebam 
como parâmetro um valor do tipo definido em nossa 
estrutura de dados imutável. Seguindo esses dois 
princípios, uma versão funcional da classe apresentada 
no código \ref{ooclass} pode ser vista no código \ref{fpclass}.


\begin{lstlisting}[caption={Representação de uma classe no contexto funcional},label=fpclass]
    
    type Pessoa = (String, Int)

    def getNome(pessoa : Pessoa) : String = pessoa._1 

    def setNome(pessoa : Pessoa, name : String) : Pessoa = 
        (name, pessoa._2)

    def getIdade(pessoa : Pessoa) : Int = pessoa._2

    def setIdade(pessoa : Pessoa, age : Int) : Pessoa =
        (pessoa._1, age)

\end{lstlisting}

%associação
\section{Associação, Agregação e Composição}

Uma associação pode ser definida como uma 
conexão entre as classes que indica algum 
relacionamento entre elas\cite{Sommerville10}. 
O código \ref{ooassociation} demonstra uma 
associação entre as classes Cidade e Estado, onde 
a classe Estado possui uma coleção de atributos 
do tipo Cidade. Para que haja uma associação 
entre duas classes, basta que pelo menos 
uma delas tenha em seus atributos uma 
referência à outra.

\begin{lstlisting}[caption={Exemplo de associação entre classes},label=ooassociation]
    
    class Cidade(var nome : String){

        def getNome() : String = this.nome;
        def setNome(nome : String) : Unit {
            this.nome = nome;
        }
    }

    class Estado(var nome : String, var cidades : List[Cidade]){

        def getNome() : String = this.nome;
        def setNome(nome : String) : Unit {
            this.nome = nome;
        }
        def getCidades() : List[Cidade] = this.cidades;
        def addCidade(cidade : Cidade) : Unit {
            this.cidades = this.cidades :+ cidade;
        }
    }

\end{lstlisting}

Como foi visto anteriormente, os atributos 
podem ser representados por valores salvos 
dentro de uma tupla associada a um tipo. 
Portanto, uma associação dentro do contexto 
funcional pode ser implementada armazenando 
um valor de um tipo A entre os valores da tupla 
de um tipo B. O código \ref{fpassociation} 
demonstra o exemplo anterior implementado 
de forma funcional.

\begin{lstlisting}[caption={Exemplo de associação no contexto funcional},label=fpassociation]
    
    type Cidade = (String)
    
    def getNome(cidade : Cidade) : String = cidade._1;
    def setNome(cidade : Cidade, nome : String) : Cidade = (nome)

    type Estado = (String, List[Cidade])
    
    def getNome(estado : Estado) : String = estado._1;
    def setNome(estado : Estado, nome : String) : Estado = (nome, estado._2)
    
    def getCities(estado : Estado) : List[Cidade] = estado._2;
    def addCity(estado : Estado, cidade : Cidade) : Estado =
        (estado._1, estado._2 :+ cidade)

\end{lstlisting}

% encapsulamento
\section{Encapsulamento}

A abordagem da seção anterior implementa 
classes e objetos, porém precisa ser 
reavaliada para que possa levar em consideração 
o encapsulamento. Encapsulamento pode ser definido 
como uma forma de limitar o acesso a um conjunto 
de dados ou comportamentos de um objeto \cite{quarkoo}. 
A motivação para isso pode vir tanto da necessidade 
de concentrar as alterações externas que um objeto 
pode sofrer em apenas um lugar quanto evitar que 
esse objeto assuma um estado que não deveria ser 
representado. 

Com a ideia de imutabilidade, pode-se 
assumir que um valor não será alterado em partes 
diferentes de uma aplicação, mas é possível 
que funções responsáveis por criar ou modificar\footnote{
    Uma função que modifica um valor é entendida 
    como uma função que recebe um valor existente 
    por parâmetro e retorna um novo valor do mesmo 
    tipo.
} 
um valor de um determinado tipo estejam 
espalhadas pela aplicação, facilitando uma 
situação em que um estado que não deveria ser 
representado por esse valor seja criado. 
Dessa forma, implementar alguma forma de 
encapsulamento ainda é importante no 
contexto funcional.

Existe mais de uma abordagem que torna 
possível implementar o encapsulamento em 
linguagens funcionais, o uso de GADTs - 
\textit{Generalized Algebraic 
Data Types}\cite{existentialhaskell} é uma 
delas. Closures também podem 
ser utilizadas ao armazenar valores de 
atributos enquanto retorna as funções 
necessárias para acessá-los ou modificá-los. 
Um exemplo equivalente ao do código \ref{fpclass} 
pode ser visto no código \ref{fpclosure}, 
implementado utilizando a linguagem funcional Clojure. 
\cite{classlessjs} Nele, a função pessoa 
funciona como um construtor que recebe como 
parâmetro o nome e a idade de uma pessoa e 
retorna um dicionário com as funções para 
recuperar ou modificar o estado de uma pessoa.
As funções de modificação retornam uma nova versão 
da pessoa com o estado alterado.

\begin{lstlisting}[caption={Representação de uma classe com closures},label=fpclosure]
    
    (defn pessoa [nome idade]
        {:getNome nome
         :setNome (fn [_nome] (pessoa _nome idade))
         :getIdade idade
         :setIdade (fn [_idade] (pessoa nome _idade))})

\end{lstlisting}

Apesar de não ser um conceito de programação 
funcional, também é possível aproveitar a ideia 
de modularização para esconder detalhes de 
implementação \cite{mlmodules}. Por exemplo, o 
código \ref{modulesencap}, implementado em 
Haskell, mostra o tipo Pessoa com um construtor 
P. Apesar do tipo Pessoa ser exportado para fora do 
módulo, P não é, tornando impossível para qualquer 
função que acesse esse módulo criar algo do tipo 
Pessoa. Dessa forma, apenas a função newPessoa, 
também exportada pelo módulo, 
pode criar novos valores do tipo 
Pessoa. Funções implementadas dentro do módulo 
também podem deixar de ser exportadas, o que 
as tornaria semelhantes a métodos privados 
de uma classe.

\begin{lstlisting}[caption={Módulos como forma de encapsulamento},label=modulesencap]
    
    module Pessoa (Pessoa, newPessoa, getNome, setNome, getIdade, setIdade) where

    data Pessoa = P (String, Int)

    newPessoa :: String -> Int -> Pessoa
    newPessoa nome idade = P (nome, idade)

    getNome :: Pessoa -> String
    getNome (P (nome, _)) = nome

    setNome :: Pessoa -> String -> Pessoa
    setNome (P (_, idade)) nome = P (nome, idade)

    getIdade :: Pessoa -> Int
    getIdade (P (_, idade)) = idade

    setIdade :: Pessoa -> Int -> Pessoa
    setIdade (P (nome, _)) idade = P (nome, idade)

\end{lstlisting}

Todas essas abordagens são válidas para a 
implementação do encapsulamento, sendo a 
linguagem utilizada um fator mais decisivo 
do que a abordagem em si. Por exemplo, é 
mais simples implementar a abordagem de closures 
em Clojure por ela ser dinamicamente tipada, 
permitindo que um dicionário sem estrutura 
predefinida seja retornado. Linguagens que exigem 
uma definição mais estrita do tipo de retorno 
de uma função podem dificultar tanto a 
implementação dessas funções quanto seu uso 
no resto do programa.

Sendo o objetivo dessa seção demonstrar que 
o encapsulamento pode ser implementado e 
não definir como implementá-lo, 
a abordagem utilizada para o encapsulamento 
durante a análise dos padrões será 
omitida, a menos que ela seja relevante para 
sua implementação. Essa omissão 
também tem como objetivo não particularizar o 
método de encapsulamento utilizado durante a 
implementação dos exemplos.

% interfaces
\section{Interfaces}

Uma interface pode ser entendida como um contrato 
entre uma classe e o mundo externo, indicando que 
uma classe que implementa uma interface também 
implementará as operações definidas 
pela mesma\cite{oracleooconcepts}. 

Um exemplo do uso de interfaces é demonstrado no código 
\ref{oopinterface1}, 
onde a interface é necessária para garantir que as 
classes SomadorMaisUm e MultiplicadorPorDois implementem 
a operação calcular, que recebe como parâmetro 
um valor do tipo inteiro e retorna outro valor 
inteiro.

% Exemplo 1 de Interfaces

\begin{lstlisting}[caption={Interfaces em Orientação a Objetos},label=oopinterface1]
    
    trait ICalculaInteiro {
        def calcular(x : Int) : Int
    }

    class SomadorMaisUm extends ICalculaInteiro {
        def calcular(x : Int) : Int = x + 1
    }

    class MultiplicadorPorDois extends ICalculaInteiro {
        def calcular(x : Int) : Int = 2*x
    }

    def calcularInteiro(x : Int, calculador : ICalculaInteiro) : Int {
        return calculador.calcular(x)
    }

\end{lstlisting}

Utilizando funções de alta ordem e levando em 
consideração que as funções que representam nossos 
métodos não estão encapsulados em classes e 
não dependem de atributos, é possível substituir o 
objeto sendo passado por parâmetro na função 
calcularInteiro por uma função qualquer que recebe 
como parâmetro um valor inteiro e retorna outro 
valor inteiro. Essa alternativa pode ser vista 
no código \ref{fpinterface1}.

\begin{lstlisting}[caption={Interfaces em Programação Funcional},label=fpinterface1]
    
    def somaUm(x : Int) : Int = x + 1

    def multiplicaPorDois(x : Int) : Int = 2*x

    def calcularInteiro(x : Int, calcular : (Int => Int)) =
        calcular(x)
    
\end{lstlisting}



% herança
\section{Herança}

Quando é desejado que uma classe seja incluída ou 
utilizada como base para a criação de outra classe, 
usa-se a herança\cite{quarkoo}. Dessa forma, é 
possível criar implementações mais específicas 
de classes já existentes e reaproveitar o código. 
O exemplo \ref{ooinheritance} demonstra 
o uso da herança entre as classes Animal e Dog. 
Ao invés de reimplementar os métodos da classe 
Animal, a classe Cachorro usa herança para reaproveitá-los. 

\begin{lstlisting}[caption={Herança em Orientação a Objetos},label=ooinheritance]
    
    class Animal(var nome : String) {
        def getNome() : String = nome
        
        def comer() : String {
            return "Meu nome é " + nome + " e eu posso comer";
        }
    }

    class Cachorro extends Animal {
        
        def Cachorro(nome : String) {
            super(nome);
        }

        def latir() : String {
            return "Au! Meu nome é " + super.getNome();
        }

        def comer() : String {
            return super.comer() + "\nEu como comida de cachorro";
        }
    }

\end{lstlisting}

No contexto funcional, um comportamento semelhante 
pode ser alcançado através da composição. Um tipo A 
que deseja herdar as funcionalidades de um tipo B 
deve possuir uma instância desse mesmo tipo em seus 
atributos. Para os métodos do tipo A, basta que as 
funções do tipo B sejam compostas das funções 
necessárias do tipo A. O código \ref{fpinheritance} 
demonstra o exemplo anterior, onde um tipo Animal 
armazena um valor String que representa o nome 
enquanto o tipo Cachorro armazena um valor Animal. 
As funções latir e comer que recebem como parâmetro 
um valor do tipo Cachorro reutilizam as funções getNome 
e comer que recebem como parâmetro um valor do tipo 
Animal. Nesse exemplo, o tipo Animal representa 
uma classe pai e o tipo Cachorro uma class filha.

\begin{lstlisting}[caption={Herança em Programação Funcional},label=fpinheritance]
    
    type Animal = (String)

    def getName(animal : Animal) : String = animal._1;
    def eat(animal : Animal) : String = 
        "Meu nome é " + getNome(animal) + " e eu posso comer"

    type Cachorro = (Animal)

    def getAnimal(cachorro : Cachorro) : Animal = cachorro._1;
    def bark(cachorro : Cachorro) : String = 
        "Au! Meu nome é " + getName(getAnimal(cachorro))
    def eat(cachorro : Cachorro) : String = 
        eat(getAnimal(Cachorro)) + "\nEu como comida de cachorro";

\end{lstlisting}

É possível perceber que a implementação da herança 
assemelha-se à implementação de uma associação, 
por isso ela apresenta uma desvantagem: 
qualquer função do tipo que representa 
a classe pai necessitará de uma função 
intermediária do tipo que representa a classe 
filha para acessá-la. 
No contexto orientado a objetos, esses dois 
relacionamentos são diferentes, pois a 
herança trata-se de um relacionamento entre 
classes enquanto a associação é um relacionamento 
entre objetos\cite{umlsystems}. 