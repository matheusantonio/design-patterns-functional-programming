\chapter{Trabalhos Relacionados}

Apesar de não existirem muitos trabalhos que envolvem
relacionar padrões de projeto com o paradigma funcional, 
diversas revisões dos padrões GoF já foram feitas, 
tanto utilizando como motivação especificamente o 
paradigma funcional como este trabalho, quanto 
utilizando uma visão mais abrangente que aproveita 
outros recursos e evoluções de linguagens de 
programação.

Alguns desses trabalhos serão apresentados a seguir.
Nem todos eles focam apenas nos padrões de 
projeto GoF, alguns inclusive propõem padrões baseados 
em conceitos de programação funcional.

Scott Wlaschin, em sua palestra "Functional Programming 
Design Patterns", apresenta conceitos de programação 
funcional como combinação de funções, funções de alta 
ordem e monads e como esses recursos podem ser 
interpretados como padrões para a solução de problemas 
de design de software funcional.

Parte de uma série de artigos denominada "Functional 
Thinking", escritos por Neal Ford e disponibilizada 
no site da IBM, descreve como alguns padrões de projeto 
podem ser interpretados no contexto funcional e 
apresenta três possibilidades para essa interpretação: 
os padrões são absorvidos pelos recursos da 
linguagem; continuam existindo, porém possuindo 
uma implementa diferente; ou são solucionados 
utilizando recursos que outras linguagens ou 
paradigmas não possuem.

Em uma palestra disponibilizada no InfoQ, Stuart Sierra 
apresenta os "Clojure Design Patterns", onde alguns 
padrões GoF, entre eles Observer e Strategy, são 
revisitados a partir de um ponto de vista funcional. 
Porém, a maior parte da palestra propõe 
diversos padrões derivados do paradigma funcional.

Já a palestra "From GoF to lambda", apresentada por 
Mario Fusco, demonstra como alguns dos padrões GoF 
podem ser revistos com o recurso de funções lambda, 
incluídas na versão 8 da linguagem Java.

Por fim, Peter Norvig apresenta "Design Patterns in 
Dynamic Languages", que apesar de não ser focado 
no paradigma funcional, dedica-se a revisitar alguns 
padrões de projeto utilizando recursos linguagens 
de programação dinâmicas. 
