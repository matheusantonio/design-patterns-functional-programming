\chapter{Trabalhos Relacionados}

Apesar de não existirem muitos trabalhos que envolvem
relacionar padrões de projeto com o paradigma funcional, 
diversas revisões dos padrões \textit{Gang of Four} 
já foram feitas 
\cite{nealford,peternorvig,scottwlaschin,stuartsierra,mariofusco}.
Essas revisões são baseadas tanto no paradigma 
funcional - como neste trabalho - quanto em  
conceitos mais abrangentes, aproveitando 
outros recursos e evoluções de linguagens de 
programação posteriores ao paradigma orientado a 
objetos como era conhecido quando os padrões GoF 
foram catalogados.

Alguns desses trabalhos serão apresentados a seguir.
A maioria não se restringe aos padrões de 
projeto \textit{Gang of Four}, alguns inclusive propõem 
padrões baseados em conceitos de programação funcional.

Scott Wlaschin, em sua palestra \textit{Functional Programming 
Design Patterns} \cite{scottwlaschin}, apresenta conceitos 
de programação funcional como combinação de funções, 
funções de alta ordem e mônadas. Em seguida, é 
demonstrado como esses recursos podem ser 
interpretados como padrões para solucionar problemas 
de \textit{design} de \textit{software} funcional.

Parte de uma série de artigos denominada \textit{Functional 
Thinking}, escritos por Neal Ford e disponibilizada 
no site da IBM \cite{nealford}, descreve como alguns padrões 
de projeto podem ser interpretados no contexto funcional e 
apresenta três possibilidades de interpretação: 
os padrões são absorvidos pelos recursos da 
linguagem; os padrões continuam existindo, porém possuindo 
uma implementa diferente; os padrões são solucionados 
utilizando recursos que outras linguagens ou 
paradigmas não possuem.

Em uma palestra disponibilizada no \textit{InfoQ} 
\cite{stuartsierra}, Stuart Sierra 
apresenta os \textit{Clojure Design Patterns}, onde alguns 
padrões \textit{Gang of Four}, entre eles Observer e 
Strategy, são revisitados a partir de um ponto de 
vista funcional. Porém, a maior parte da palestra propõe 
diversos padrões derivados do paradigma funcional.

Já a palestra \textit{From GoF to lambda} \cite{mariofusco}, 
apresentada por Mario Fusco, demonstra como alguns 
dos padrões \textit{Gang of Four} podem ser revistos 
com o recurso de funções lambda que a linguagem Java 
passou a implementar a partir da versão 8.

Por fim, Peter Norvig apresenta \textit{Design Patterns in 
Dynamic Languages}\cite{peternorvig}, que apesar de não ser 
focado no paradigma funcional, dedica-se a revisitar alguns 
padrões de projeto \textit{Gang of Four} utilizando recursos 
de linguagens de programação dinâmicas. 
