%% abtex2-modelo-trabalho-academico.tex, v-1.9.7 laurocesar
%% Copyright 2012-2018 by abnTeX2 group at http://www.abntex.net.br/ 
%%
%% This work may be distributed and/or modified under the
%% conditions of the LaTeX Project Public License, either version 1.3
%% of this license or (at your option) any later version.
%% The latest version of this license is in
%%   http://www.latex-project.org/lppl.txt
%% and version 1.3 or later is part of all distributions of LaTeX
%% version 2005/12/01 or later.
%%
%% This work has the LPPL maintenance status `maintained'.
%% 
%% The Current Maintainer of this work is the abnTeX2 team, led
%% by Lauro César Araujo. Further information are available on 
%% http://www.abntex.net.br/
%%
%% This work consists of the files abntex2-modelo-trabalho-academico.tex,
%% abntex2-modelo-include-comandos and abntex2-modelo-references.bib
%%

% ------------------------------------------------------------------------
% ------------------------------------------------------------------------
% abnTeX2: Modelo de Trabalho Academico (tese de doutorado, dissertacao de
% mestrado e trabalhos monograficos em geral) em conformidade com 
% ABNT NBR 14724:2011: Informacao e documentacao - Trabalhos academicos -
% Apresentacao
% ------------------------------------------------------------------------
% ------------------------------------------------------------------------

\documentclass[
	% -- opções da classe memoir --
	12pt,				% tamanho da fonte
	openright,			% capítulos começam em pág ímpar (insere página vazia caso preciso)
	twoside,			% para impressão em recto e verso. Oposto a oneside
	a4paper,			% tamanho do papel. 
	% -- opções da classe abntex2 --
	%chapter=TITLE,		% títulos de capítulos convertidos em letras maiúsculas
	%section=TITLE,		% títulos de seções convertidos em letras maiúsculas
	%subsection=TITLE,	% títulos de subseções convertidos em letras maiúsculas
	%subsubsection=TITLE,% títulos de subsubseções convertidos em letras maiúsculas
	% -- opções do pacote babel --
	english,			% idioma adicional para hifenização
	french,				% idioma adicional para hifenização
	spanish,			% idioma adicional para hifenização
	brazil				% o último idioma é o principal do documento
	]{abntex2}

% ---
% Pacotes básicos 
% ---
\usepackage{lmodern}			% Usa a fonte Latin Modern			
\usepackage[T1]{fontenc}		% Selecao de codigos de fonte.
\usepackage[utf8]{inputenc}		% Codificacao do documento (conversão automática dos acentos)
\usepackage{indentfirst}		% Indenta o primeiro parágrafo de cada seção.
\usepackage[dvipsnames]{xcolor}				% Controle das cores
\usepackage{graphicx}			% Inclusão de gráficos
\usepackage{microtype} 			% para melhorias de justificação
% ---
		
% ---
% Pacotes adicionais, usados apenas no âmbito do Modelo Canônico do abnteX2
% ---
\usepackage{lipsum}				% para geração de dummy text
% ---

% ---
% Pacotes de citações
% ---
\usepackage[brazilian,hyperpageref]{backref}	 % Paginas com as citações na bibl
\usepackage[num]{abntex2cite}	% Citações padrão ABNT

% ---
% Formatação de código-fonte
% ---
\usepackage{listings}

% Altera o nome padrão do rótulo usado no comando \autoref{}
\renewcommand{\lstlistingname}{Código}

% Altera o rótulo a ser usando no elemento pré-textual "Lista de código"
\renewcommand{\lstlistlistingname}{Lista de códigos}

% Configura a ``Lista de Códigos'' conforme as regras da ABNT (para abnTeX2)
\begingroup\makeatletter
\let\newcounter\@gobble\let\setcounter\@gobbletwo
  \globaldefs\@ne \let\c@loldepth\@ne
  \newlistof{listings}{lol}{\lstlistlistingname}
  \newlistentry{lstlisting}{lol}{0}
\endgroup

\renewcommand{\cftlstlistingaftersnum}{\hfill--\hfill}

\let\oldlstlistoflistings\lstlistoflistings
\renewcommand{\lstlistoflistings}{%
   \begingroup%
   \let\oldnumberline\numberline%
   \renewcommand{\numberline}{\lstlistingname\space\oldnumberline}%
   \oldlstlistoflistings%
   \endgroup}

% Cria uma nova customização para a linguagem Prolog
\lstloadlanguages{Prolog}
\lstdefinestyle{prologCustom}{
  alsoother={0123456789_},
  backgroundcolor=\color{white},   % choose the background color; you must add \usepackage{color} or \usepackage{xcolor}
  % the size of the fonts that are used for the code
  basicstyle=\ttfamily\ABNTEXfontereduzida, 
  %backgroundcolor=\color{theshade},
  breakatwhitespace=false,         % sets if automatic breaks should only happen at whitespace
  breaklines=true,                 % sets automatic line breaking
  captionpos=b,                    % sets the caption-position to bottom
  commentstyle=\color{Green},      % comment style
  deletekeywords={...},            % if you want to delete keywords from the given language
  escapeinside={\%*}{*)},          % if you want to add LaTeX within your code
  extendedchars=true,              % lets you use non-ASCII characters; for 8-bits encodings only, does not work with UTF-8
  frame=single,                    % adds a frame around the code
  inputencoding=utf8,
  keepspaces=true,                 % keeps spaces in text, useful for keeping indentation of code (possibly needs columns=flexible)
  keywordstyle=\color{blue},       % keyword style
  language=scala,                 % the language of the code
  literate={á}{{\'a}}1 {ã}{{\~a}}1 {é}{{\'e}}1 {è}{{\`{e}}}1 {ê}{{\^{e}}}1 {ë}{{\¨{e}}}1 {É}{{\'{E}}}1 {Ê}{{\^{E}}}1 {û}{{\^{u}}}1 {ú}{{\'{u}}}1 {â}{{\^{a}}}1 {à}{{\`{a}}}1 {á}{{\'{a}}}1 {ã}{{\~{a}}}1 {Á}{{\'{A}}}1 {Â}{{\^{A}}}1 {Ã}{{\~{A}}}1 {ç}{{\c{c}}}1 {Ç}{{\c{C}}}1 {õ}{{\~{o}}}1 {ó}{{\'{o}}}1 {ô}{{\^{o}}}1 {Õ}{{\~{O}}}1 {Ó}{{\'{O}}}1 {Ô}{{\^{O}}}1 {î}{{\^{i}}}1 {Î}{{\^{I}}}1 {í}{{\'{i}}}1 {Í}{{\~{Í}}}1,
  % if you want to add more keywords to the set
  morekeywords={*, :-},
  numberbychapter=false,
  % the style that is used for the line-numbers
  %numberstyle=\tiny\color{theframe}\sffamily, 
  %rulecolor=\color{theframe},         % if not set, the frame-color may be changed on line-breaks within not-black text (e.g. comments (green here))
  showspaces=false,                % show spaces everywhere adding particular underscores; it overrides 'showstringspaces'
  showstringspaces=false,          % underline spaces within strings only
  showtabs=false,                  % show tabs within strings adding particular underscores
  stepnumber=4,                    % the step between two line-numbers. If it's 1, each line will be numbered
  stringstyle=\color{YellowOrange}\itshape,     % string literal style
  tabsize=2,                       % sets default tabsize to 2 spaces
  title=\lstname,                  % show the filename of files included with \lstinputlisting; also try caption instead of title
  framexleftmargin=10pt,
  framexleftmargin=15pt
}
\lstset{escapechar=@,style=prologCustom}
% ---



% --- 
% CONFIGURAÇÕES DE PACOTES
% --- 

% ---
% Configurações do pacote backref
% Usado sem a opção hyperpageref de backref
\renewcommand{\backrefpagesname}{Citado na(s) página(s):~}
% Texto padrão antes do número das páginas
\renewcommand{\backref}{}
% Define os textos da citação
\renewcommand*{\backrefalt}[4]{
	\ifcase #1 %
		Nenhuma citação no texto.%
	\or
		Citado na página #2.%
	\else
		Citado #1 vezes nas páginas #2.%
	\fi}%
% ---

% ---
% Informações de dados para CAPA e FOLHA DE ROSTO
% ---
\titulo{Padrões de Projeto\\ e o Paradigma Funcional}
\autor{Matheus Antonio Oliveira Cardoso}
\local{Brasil}
\data{2021, v-1.9.7}
\orientador{Carlos Bazilio Martins}
% \coorientador{Equipe \abnTeX}
\instituicao{%
  Universidade Federal Fluminense -- UFF
  \par
  Instituto de Ciência e Tecnologia
  \par
  Ciência da Computação}
\tipotrabalho{Tese (Graduação)}
% O preambulo deve conter o tipo do trabalho, o objetivo, 
% o nome da instituição e a área de concentração 
\preambulo{Modelo canônico de trabalho monográfico acadêmico em conformidade com
as normas ABNT apresentado à comunidade de usuários \LaTeX.}
% ---


% ---
% Configurações de aparência do PDF final

% alterando o aspecto da cor azul
\definecolor{blue}{RGB}{41,5,195}

% informações do PDF
\makeatletter
\hypersetup{
     	%pagebackref=true,
		pdftitle={\@title}, 
		pdfauthor={\@author},
    	pdfsubject={\imprimirpreambulo},
	    pdfcreator={LaTeX with abnTeX2},
		pdfkeywords={abnt}{latex}{abntex}{abntex2}{trabalho acadêmico}, 
		colorlinks=true,       		% false: boxed links; true: colored links
    	linkcolor=blue,          	% color of internal links
    	citecolor=blue,        		% color of links to bibliography
    	filecolor=magenta,      		% color of file links
		urlcolor=blue,
		bookmarksdepth=4
}
\makeatother
% --- 

% ---
% Posiciona figuras e tabelas no topo da página quando adicionadas sozinhas
% em um página em branco. Ver https://github.com/abntex/abntex2/issues/170
\makeatletter
\setlength{\@fptop}{5pt} % Set distance from top of page to first float
\makeatother
% ---

% ---
% Possibilita criação de Quadros e Lista de quadros.
% Ver https://github.com/abntex/abntex2/issues/176
%
\newcommand{\quadroname}{Quadro}
\newcommand{\listofquadrosname}{Lista de quadros}

\newfloat[chapter]{quadro}{loq}{\quadroname}
\newlistof{listofquadros}{loq}{\listofquadrosname}
\newlistentry{quadro}{loq}{0}

% configurações para atender às regras da ABNT
\setfloatadjustment{quadro}{\centering}
\counterwithout{quadro}{chapter}
\renewcommand{\cftquadroname}{\quadroname\space} 
\renewcommand*{\cftquadroaftersnum}{\hfill--\hfill}

\setfloatlocations{quadro}{hbtp} % Ver https://github.com/abntex/abntex2/issues/176
% ---

% --- 
% Espaçamentos entre linhas e parágrafos 
% --- 

% O tamanho do parágrafo é dado por:
\setlength{\parindent}{1.3cm}

% Controle do espaçamento entre um parágrafo e outro:
\setlength{\parskip}{0.2cm}  % tente também \onelineskip

% ---
% compila o indice
% ---
\makeindex
% ---

% ----
% Início do documento
% ----
\begin{document}

% Seleciona o idioma do documento (conforme pacotes do babel)
%\selectlanguage{english}
\selectlanguage{brazil}

% Retira espaço extra obsoleto entre as frases.
\frenchspacing 

\renewcommand{\imprimircapa}{%
  \begin{capa}%
    \center
    \ABNTEXchapterfont\Large \imprimirinstituicao

    \vspace*{1cm}

    {\ABNTEXchapterfont\large\imprimirautor}


    \vfill

    \begin{center}
      \ABNTEXchapterfont\bfseries\LARGE\imprimirtitulo
    \end{center}

    \vfill


    

    \large\imprimirlocal

    \large\imprimirdata

    \vspace*{1cm}

  \end{capa}
}



\makeatletter
\renewcommand{\folhaderostocontent}{

\begin{center}
  \center
    \ABNTEXchapterfont\Large \imprimirinstituicao

    \vspace*{1cm}

    {\ABNTEXchapterfont\large\imprimirautor}


  \vspace*{\fill}\vspace*{\fill}

  \begin{center}
    \ABNTEXchapterfont\bfseries\Large\imprimirtitulo
  \end{center}

  \vspace*{\fill}

  \abntex@ifnotempty{\imprimirpreambulo}{%    
 
    \hspace{.45\textwidth}

    \begin{flushright}  
      \begin{minipage}{.5\textwidth}
        \SingleSpacing
        \imprimirpreambulo
      \end{minipage}%
    \end{flushright}

    \vspace*{\fill}
  }%


  

  {\large\imprimirorientadorRotulo~\imprimirorientador\par}


  \vspace*{\fill}

  {\large\imprimirlocal}

  \par

  {\large\imprimirdata}

  \vspace*{1cm}

\end{center}
}

\makeatother








% ---
% Capa
% ---
\imprimircapa
% ---

% ---
% Folha de rosto
% (o * indica que haverá a ficha bibliográfica)
% ---
\imprimirfolhaderosto*
% ---

% ---
% Inserir a ficha bibliografica
% ---

% Isto é um exemplo de Ficha Catalográfica, ou ``Dados internacionais de
% catalogação-na-publicação''. Você pode utilizar este modelo como referência. 
% Porém, provavelmente a biblioteca da sua universidade lhe fornecerá um PDF
% com a ficha catalográfica definitiva após a defesa do trabalho. Quando estiver
% com o documento, salve-o como PDF no diretório do seu projeto e substitua todo
% o conteúdo de implementação deste arquivo pelo comando abaixo:
%
% \begin{fichacatalografica}
%     \includepdf{fig_ficha_catalografica.pdf}
% \end{fichacatalografica}

%\begin{fichacatalografica}
%	\sffamily
%	\vspace*{\fill}					% Posição vertical
%	\begin{center}					% Minipage Centralizado
%	\fbox{\begin{minipage}[c][8cm]{13.5cm}		% Largura
%	\small
%	\imprimirautor
%	%Sobrenome, Nome do autor
%	
%	\hspace{0.5cm} \imprimirtitulo  / \imprimirautor. --
%	\imprimirlocal, \imprimirdata-
%	
%	\hspace{0.5cm} \thelastpage p. : il. (algumas color.) ; 30 cm.\\
%	
%	\hspace{0.5cm} \imprimirorientadorRotulo~\imprimirorientador\\
%	
%	\hspace{0.5cm}
%	\parbox[t]{\textwidth}{\imprimirtipotrabalho~--~\imprimirinstituicao,
%	\imprimirdata.}\\
%	
%	\hspace{0.5cm}
%		1. Palavra-chave1.
%		2. Palavra-chave2.
%		2. Palavra-chave3.
%		I. Orientador.
%		II. Universidade xxx.
%		III. Faculdade de xxx.
%		IV. Título 			
%	\end{minipage}}
%	\end{center}
%\end{fichacatalografica}
% ---


% ---
% Inserir folha de aprovação
% ---

% Isto é um exemplo de Folha de aprovação, elemento obrigatório da NBR
% 14724/2011 (seção 4.2.1.3). Você pode utilizar este modelo até a aprovação
% do trabalho. Após isso, substitua todo o conteúdo deste arquivo por uma
% imagem da página assinada pela banca com o comando abaixo:
%
%\begin{folhadeaprovacao}
%\includepdf{folhadeaprovacao_final.pdf}
%\end{folhadeaprovacao}

\begin{folhadeaprovacao}

  \begin{center}
    {\ABNTEXchapterfont\large\imprimirautor}

    \vspace*{\fill}\vspace*{\fill}
    \begin{center}
      \ABNTEXchapterfont\bfseries\Large\imprimirtitulo
    \end{center}
    \vspace*{\fill}
    
    \hspace{.45\textwidth}
    \begin{minipage}{.5\textwidth}
        \imprimirpreambulo
    \end{minipage}%
    \vspace*{\fill}
   \end{center}
        
   Trabalho aprovado. \imprimirlocal, 15 de setembro de 2021:

   \assinatura{\textbf{\imprimirorientador} \\ Orientador} 
   %\assinatura{\textbf{Professor} \\ Convidado 1}
   %\assinatura{\textbf{Professor} \\ Convidado 2}
   %\assinatura{\textbf{Professor} \\ Convidado 3}
   %\assinatura{\textbf{Professor} \\ Convidado 4}
      
   \begin{center}
    \vspace*{0.5cm}
    {\large\imprimirlocal}
    \par
    {\large\imprimirdata}
    \vspace*{1cm}
  \end{center}
  
\end{folhadeaprovacao}
% ---

% ---
% Dedicatória
% ---
%\begin{dedicatoria}
%   \vspace*{\fill}
%   \centering
%   \noindent
%   \textit{} \vspace*{\fill}
%\end{dedicatoria}
% ---

% ---
% Agradecimentos
% ---
%\begin{agradecimentos}
%Os agradecimentos principais são direcionados à Gerald Weber, Miguel Frasson,
%Leslie H. Watter, Bruno Parente Lima, Flávio de Vasconcellos Corrêa, Otavio Real
%Salvador, Renato Machnievscz\footnote{Os nomes dos integrantes do primeiro
%projeto abn\TeX\ foram extraídos de
%\url{http://codigolivre.org.br/projects/abntex/}} e todos aqueles que
%contribuíram para que a produção de trabalhos acadêmicos conforme
%as normas ABNT com \LaTeX\ fosse possível.
%
%Agradecimentos especiais são direcionados ao Centro de Pesquisa em Arquitetura
%da Informação\footnote{\url{http://www.cpai.unb.br/}} da Universidade de
%Brasília (CPAI), ao grupo de usuários
%\emph{latex-br}\footnote{\url{http://groups.google.com/group/latex-br}} e aos
%novos voluntários do grupo
%\emph{\abnTeX}\footnote{\url{http://groups.google.com/group/abntex2} e
%\url{http://www.abntex.net.br/}}~que contribuíram e que ainda
%contribuirão para a evolução do \abnTeX.
%
%\end{agradecimentos}
% ---

% ---
% Epígrafe
% ---
%\begin{epigrafe}
%    \vspace*{\fill}
%	\begin{flushright}
%		\textit{``Não vos amoldeis às estruturas deste mundo, \\
%		mas transformai-vos pela renovação da mente, \\
%		a fim de distinguir qual é a vontade de Deus: \\
%		o que é bom, o que Lhe é agradável, o que é perfeito.\\
%		(Bíblia Sagrada, Romanos 12, 2)}
%	\end{flushright}
%\end{epigrafe}
% ---


% ---
% RESUMOS
% ---

% resumo em português
\setlength{\absparsep}{18pt} % ajusta o espaçamento dos parágrafos do resumo
\begin{resumo}
  O presente trabalho tem como objetivo analisar 
  um conjunto de padrões de projeto no contexto do 
  paradigma de programação funcional. Os padrões 
  de projeto apresentam soluções tradicionais para problemas 
  comuns de \textit{design} de \textit{software}, 
  destacando-se os vinte e três padrões 
  \textit{Gang of Four}, que apresentam soluções 
  voltadas para o desenvolvimento de \textit{software} 
  orientado a objetos. Porém, como a forma 
  de construir um \textit{software} difere muito 
  do paradigma funcional para o orientado a 
  objetos, existe a dúvida de como ou se os 
  problemas apresentados pelos padrões podem ser 
  solucionados com o uso de uma linguagem 
  funcional. Dessa forma, o trabalho analisará, 
  do ponto de vista funcional, cada um dos 
  vinte e três padrões \textit{Gang of Four}, 
  verificando se o problema de orientação a objetos 
  proposto também existe no contexto funcional e 
  como pode ser resolvido no mesmo. Ao 
  fim, deseja-se concluir se o uso dos recursos de 
  programação funcional pode contribuir para a 
  solução de cada padrão e, caso a conclusão não 
  seja a mesma para todos os padrões, 
  identificar características que possibilitem 
  agrupar as soluções encontradas.

 \textbf{Palavras-chave}: padrões de projeto. programação funcional.
\end{resumo}

\begin{comment}
  O presente trabalho tem como objetivo analisar 
  o conceito de padrões de projeto no contexto do 
  paradigma de programação funcional. Os padrões 
  de projeto apresentam soluções comuns para problemas 
  comuns de design de software, destacando-se os 
  vinte e três padrões Gang of Four, que apresentam 
  soluções comuns para problemas relacionados ao 
  paradigma orientado a objetos. Porém, como a forma 
  de construir um software difere muito do paradigma 
  funcional para o orientado a 
  objetos, existe a dúvida de como ou se esses padrões 
  podem ser reaproveitados, além da possibilidade de o 
  uso do paradigma funcional solucionar os problemas 
  oriundos da orientação a objetos.
  Dessa forma, o trabalho buscará analisar, 
  do ponto de vista funcional, cada um dos 
  23 padrões GoF, verificando se o problema de 
  orientação a objetos proposto também existe 
  no contexto funcional e se é resolvido pelo 
  padrão em questão. Também serão analisados, se 
  existirem, os casos em que o problema deixa de 
  existir ou é solucionado de outra forma. Ao 
  fim, deseja-se concluir se 
  o uso dos recursos de programação funcional 
  contribui para a solução de cada padrão GoF 
  e, caso a conclusão não seja a mesma para 
  todos os padrões, tentar identificar as 
  características de cada grupo.


 \textbf{Palavras-chave}: padrões de projeto. programação funcional.
\end{comment}

% resumo em inglês
\begin{resumo}[Abstract]
 \begin{otherlanguage*}{english}
  The present work aims to analyze a set of 
  design patterns in a functional programming 
  paradigm context. The design patterns present 
  traditional solutions for common software design 
  problems, standing out the twenty three Gang 
  of Four patterns, which present solutions 
  aimed at the object oriented software 
  development. However, since the way of building 
  a software differs a lot between a functional 
  paradigm and an object oriented paradigm, there is 
  the question of how or whether the problems 
  presented by the patterns can be solved with 
  a functional language. In this way, the work will 
  seek to analyze, from the functional point of view, 
  the twenty three Gang of Four patterns, verifying if 
  the proposed object orientation problem also 
  exists in the functional context and how it could 
  be solved by it. At the end, it is wished to 
  conclude if the use of functional programming 
  resources can add to the solution of each 
  pattern and, if the conclusion isn't the same 
  for all patterns, identify characteristics 
  that enable a grouping of the found solutions.

   \vspace{\onelineskip}
 
   \noindent 
   \textbf{Keywords}: design patterns. functional programming.
 \end{otherlanguage*}
\end{resumo}


% ---
% inserir lista de ilustrações
% ---
\pdfbookmark[0]{\listfigurename}{lof}
\listoffigures*
\cleardoublepage
% ---

% ---
% inserir lista de quadros
% ---
\pdfbookmark[0]{\listofquadrosname}{loq}
\listofquadros*
\cleardoublepage
% ---

% ---
% inserir lista de listings
% ---
\pdfbookmark[0]{\lstlistlistingname}{lol}
\begin{KeepFromToc}
\lstlistoflistings
\end{KeepFromToc}
\cleardoublepage
% ---

% ---
% inserir lista de abreviaturas e siglas
% ---
%\begin{siglas}
%	\item[GOF] Gang of Four
%  \end{siglas}
  % ---
  
  % ---
  % inserir lista de símbolos
  % ---
  %\begin{simbolos}
  %	\item[$ \Gamma $] Letra grega Gama
  %	\item[$ \Lambda $] Lambda
  %	\item[$ \zeta $] Letra grega minúscula zeta
  %	\item[$ \in $] Pertence
  %\end{simbolos}
  % ---
  
  % ---
  % inserir o sumario
  % ---
  \pdfbookmark[0]{\contentsname}{toc}
  \tableofcontents*
  \cleardoublepage
  % ---


%% [Esboço de introdução]
% [resumir o que são padrões de projeto e como eles são encontrados no dia a dia
% mencionar GOF
% introduzir o conceito de programação funcional, como é uma abordagem útil para o desenvolvimento e como não há nenhum tipo de catálogo de padrões ou de problemas] {apresentação do tema}
% [Explicar as abordagens (padrões OO que podem ser aproveitados, padrões OO que não são necessários e padrões relacionados a PF)] {delimitação do tema}
% [ ? talvez parte disso esteja sendo abordado no item anterior] {apresentar hipóteses}
% [verificar que tipos de problemas são reaproveitáveis entre os dois paradigmas através dos padrões e que novos problemas surgem através de uma análise dos recursos da linguagem, verificar se existe relação entre isso e a classificação dos padrões (criacional, comportamental, estrutural)] {objetivos e metodologia}
% [esclarecer como isso é útil para contribuir para a evolução de linguagens que não possuem certos recursos que tornam necessários padrões complexos] {relevância da pesquisa}
% [explicar o que é PF → explicar o que são padrões de projeto → abordar cada padrão GOF no contexto funcional → Abordar possíveis padrões vindos do paradigma funcional → mostrar os resultados obtidos nas análises → concluir] {estrutura do trabalho}

% ----------------------------------------------------------
% Introdução (exemplo de capítulo sem numeração, mas presente no Sumário)
% ----------------------------------------------------------
\chapter{Introdução}
% ----------------------------------------------------------
%
%contextualização

Enquanto Alan Turing definia o que viria a 
se tornar a Máquina de Turing, Alonzo Church 
trabalhava em uma abordagem diferente: o Cálculo Lambda
\cite{church1932set,church1936unsolvable,sep-turing-machine}. 
O primeiro baseia-se em modificação do estado em 
uma fita, enquanto o segundo abordava a aplicação 
de funções. Apesar do princípio por trás de ambos 
ser bem diferente, sua aplicação é computacionalmente 
equivalente, permitindo que as duas ideias possam 
ser usadas para resolver os mesmos problemas
\cite{sep-church-turing}.

%delimitação do tema

Durante o processo de construção de um 
software, podem surgir problemas de \textit{design} 
recorrentes, sendo alguns desses problemas 
tão comuns que tornaram necessário definir soluções 
reutilizáveis para os mesmos, os padrões 
de projeto. Por o paradigma 
de programação orientado a objetos ser o mais 
popular e mais utilizado no mercado de trabalho, 
existem catálogos de padrões de projetos voltados 
para o desenvolvimento orientado a objetos. 
Entretanto, existem outros paradigmas de 
programação, como a programação funcional, 
que apresenta uma abordagem diferente 
do desenvolvimento de software.

%problema de pesquisa

O paradigma orientado a objetos possui uma 
abordagem mais equivalente à máquina de Turing, 
enquanto o paradigma funcional 
tem como origem o próprio cálculo lambda. 
Dessa forma, é possível que os padrões de 
projeto desenvolvidos com base no paradigma 
orientado a objetos também sejam aplicáveis 
durante o desenvolvimento no contexto funcional. 

%objetivos gerais e específicos do trabalho

Com isso, este trabalho tem como objetivo 
analisar um conjunto de padrões de projeto 
conhecidos utilizando conceitos de programação 
funcinal. A intenção é verificar se os 
problemas mencionados pelo padrão existem 
em um contexto funcional e quais 
seriam as consequências de implementá-lo 
levando em consideração tanto as vantagens 
quanto as limitações desse paradigma.

%justificativa das escolhas

O conjunto de padrões de projeto escolhido 
para a análise é composto dos vinte e três 
padrões de projeto \textit{Gang of Four}. 
Esses padrões foram catalogados em um livro 
por quatro desenvolvedores e representam 
um conjunto conhecido de padrões de projeto 
de desenvolvimento orientado a objetos. 
Já a linguagem funcional escolhida é a 
linguagem Scala. Essa linguagem 
apresenta tanto conceitos de orientação 
a objetos quanto de programação funcional, 
o que permite que os exemplos em código 
apresentados no trabalho sejam feitos 
utilizando a mesma linguagem, facilitando 
a assimilação dos mesmos por parte do 
leitor. Entretanto, ela permite tanto 
escrever código apenas orientado a 
objetos quanto código apenas funcional, 
o que permite que os exemplos em 
código não misturem ambos os conceitos 
quando não desejado.

%metodologia



%estrutura de capitulos

\begin{comment}
    
Na parte de desenvolvimento, haverá uma introdução 
sobre como alguns conceitos de orientação a objetos, 
como classes e encapsulamento, podem ser representados 
em uma 
linguagem funcional. Em seguida, os capítulos serão 
divididos entre padrões criacionais, estruturais e 
comportamentais. Ao todo, os vinte e três padrões 
GoF serão abordados nesses três capítulos, onde serão 
apresentadas as ideias básicas do problema que o padrão 
busca resolver e como o resolve, seguido da abordagem 
funcional de resolver o mesmo problema.
    
Após analisar todos os padrões, o capítulo de resultados 
analisará as vantagens e desvantagens da abordagem 
funcional para cada solução, destacando onde ela 
contribuiu, onde atrapalhou, ou onde não fazia sentido ser 
implementada. Essas classificações dependerão das análises 
que serão realizadas na etapa de desenvolvimento.
    
Por fim, no capítulo de conclusão serão analisadas 
as consequências dessas análises e como elas podem 
impactar o conceito de padrões de projeto e as vantagens 
e desvantagens de revisá-los no ponto de vista funcional.
    
\end{comment}




% ----------------------------------------------------------
% PARTE
% ----------------------------------------------------------
\part{Conceitos Básicos}
% ----------------------------------------------------------

\chapter{Trabalhos Relacionados}

Apesar de não existirem muitos trabalhos que envolvem
relacionar padrões de projeto com o paradigma funcional, 
diversas revisões dos padrões GoF já foram feitas.
\cite{nealford,peternorvig,scottwlaschin,stuartsierra,mariofusco} 
Essas revisões são orientadas tanto ao paradigma 
funcional - como neste trabalho - quanto a  
uma visão mais abrangente, que aproveita 
outros recursos e evoluções de linguagens de 
programação posteriores ao paradigma orientado a 
objetos como era 
conhecido quando os padrões GoF foram catalogados.

Alguns desses trabalhos serão apresentados a seguir.
A maioria não se restringe aos padrões de 
projeto GoF, alguns inclusive propõem padrões baseados 
em conceitos de programação funcional.

Scott Wlaschin, em sua palestra "Functional Programming 
Design Patterns" \cite{scottwlaschin}, apresenta conceitos de programação 
funcional como combinação de funções, funções de alta 
ordem e mônadas. Em seguida, é demonstrado como esses 
recursos podem ser 
interpretados como padrões para solucionar problemas 
de design de software funcional.

Parte de uma série de artigos denominada "Functional 
Thinking", escritos por Neal Ford e disponibilizada 
no site da IBM \cite{nealford}, descreve como alguns padrões de projeto 
podem ser interpretados no contexto funcional e 
apresenta três possibilidades para essa interpretação: 
os padrões são absorvidos pelos recursos da 
linguagem; continuam existindo, porém possuindo 
uma implementa diferente; ou são solucionados 
utilizando recursos que outras linguagens ou 
paradigmas não possuem.

Em uma palestra disponibilizada no InfoQ \cite{stuartsierra}, Stuart Sierra 
apresenta os "Clojure Design Patterns", onde alguns 
padrões GoF, entre eles Observer e Strategy, são 
revisitados a partir de um ponto de vista funcional. 
Porém, a maior parte da palestra propõe 
diversos padrões derivados do paradigma funcional.

Já a palestra "From GoF to lambda" \cite{mariofusco}, apresentada por 
Mario Fusco, demonstra como alguns dos padrões GoF 
podem ser revistos com o recurso de funções lambda, 
incluídas na versão 8 da linguagem Java.

Por fim, Peter Norvig apresenta "Design Patterns in 
Dynamic Languages" \cite{peternorvig}, que apesar de não ser focado 
no paradigma funcional, dedica-se a revisitar alguns 
padrões de projeto GoF utilizando recursos de linguagens 
de programação dinâmicas. 



% ---
% Capitulo de revisão de literatura
% ---
\chapter{O Paradigma Funcional}
% ---

Originado no cálculo lambda, o paradigma funcional 
apresenta uma forma diferente de desenvolver quando 
comparado à programação imperativa ou orientada a 
objetos. A seguir, serão vistos os principais 
conceitos de programação funcional que serão 
levados em consideração durante a análise dos 
padrões de projeto.

% Funções puras
\section{Funções Puras}

Funções puras operam apenas nos parâmetros fornecidos. 
Elas não leem ou escrevem em qualquer valor que esteja 
fora do corpo da função\cite{purefunctionscala, functionalscala}.

A função apresentada no Código \ref{purefunction} 
opera apenas nos valores \textit{x} e \textit{y} que 
são passados como parâmetros da função. A partir 
dessa restrição, algumas conclusões relevantes podem 
ser tiradas. Por exemplo, uma função pura sempre 
retornará o mesmo valor para os mesmos parâmetros: 
caso a função \texttt{add} vista receba os parâmetros 1 para 
\textit{x} e 2 para \textit{y}, não importa quantas vezes 
ela seja chamada, 
o resultado da operação sempre será 3\cite{functionalscala}.

\begin{lstlisting}[caption={Exemplo de Função Pura.},label=purefunction]

    def add(x, y){
        return x + y;
    }

\end{lstlisting}
\legend{Fonte: O Autor (2021)}

O Código \ref{impurefunction} apresenta uma função 
não pura, \texttt{modify}. Essa função não é pura pois 
depende de um valor externo - a variável \textit{z} - para 
realizar uma operação. Existe ainda um outro problema 
com esse tipo de função: sua execução implica em 
um efeito colateral.

\begin{lstlisting}[caption={Exemplo de Função Impura.},label=impurefunction]

    var z = 10;

    def modify(x, y) {
        z = x + y;
    }

\end{lstlisting}
\legend{Fonte: O Autor (2021)}

Efeitos colaterais ocorrem em consultas ou 
alterações a bases de dados, modificação de 
arquivos ou até mesmo envio de dados a um 
servidor\cite{purefunctionscala, functionalscala}. 
Também ocorrem quando variáveis fora 
do escopo da função são modificadas ou lidas. 
Esse tipo de comportamento é muito comum em 
paradigmas de programação imperativos ou 
orientados a objetos, porém 
podem causar dificuldades no processo de 
\textit{debug} de um código, afinal, 
quanto maior for a quantidade de locais onde 
uma variável pode ser modificada, maior 
será o esforço para encontrar a origem de 
um problema relacionado ao valor que essa 
variável pode assumir.

Apesar disso, um programa precisa realizar 
efeitos colaterais, como os já citados: 
leitura e escrita em arquivos ou bancos de 
dados, requisições em servidores, exibição em 
uma tela. Por isso, a ideia no \textit{design} de 
\textit{software} funcional não é apenas utilizar 
funções puras, mas concentrar esses efeitos 
colaterais necessários em um local isolado das 
funções puras, o que facilita o 
processo de \textit{debugging}\cite{purefunctionscala}.



% imutabilidade
\section{Imutabilidade}

Em programação funcional, a ideia de variáveis não 
existe, ou ao menos possui uma definição diferente\cite{braveclojure}.
Em paradigmas procedurais é comum encontar trechos 
de código parecidos com o apresentado no Código \ref{mutablevar}.

\begin{lstlisting}[caption={Exemplo de Código Mutável.},label=mutablevar]

    var x = 1;
    x = x + 1;


\end{lstlisting}
\legend{Fonte: O Autor (2021)}

Porém, esse tipo de operação não é permitida no 
paradigma funcional, já que ele segue o princípio 
da imutabilidade, onde uma variável \footnote{
Em um contexto onde se utiliza imutabilidade, uma 
variável é entendida como um valor armazenado 
e não um valor variável.} que armazena um valor 
não pode ter esse valor alterado até o fim da 
execução do programa. Esse princípio é violado  
pelo Código \ref{mutablevar}.

Em um programa funcional, a modificação do valor 
de uma variável é feita copiando o valor para uma 
nova variável que passará a representar esse valor\cite{functionalscala}.
Por exemplo, o Código \ref{mutablevar} poderia ser 
escrito como no Código \ref{imutablevar}.

\begin{lstlisting}[caption={Exemplo de Código Imutável.},label=imutablevar]

    var x = 1;
    var z = x + 1;


\end{lstlisting}
\legend{Fonte: O Autor (2021)}

Apesar de parecer problemático quando é necessário 
modificar um único valor em uma lista ou uma estrutura 
de dados maior e mais complexa, o compilador torna o 
processo de cópia mais eficiente, sem que seja 
necessário de fato copiar toda a estrutura\cite{functionalscala}. 
Dessa forma, a imutabilidade impede que um valor 
atribuído seja alterado apenas durante a 
programação.


% Funções de alta ordem
\section{Funções de Alta Ordem}

% C# e Java já possuem suporte para HoF, mencionar aqui
Funções de alta ordem são funções que recebem 
outras funções como parâmetro e ainda podem 
retornar funções\cite{realworldhaskell, functionalscala}. 
Esse é um recurso 
não tão comum em linguagens orientadas a 
objeto ou procedurais, mas não é exclusivo das 
linguagens funcionais. Javascript\cite{eloquentjs}, 
Python\cite{denerocomposing} e C\#\cite{buonannofunctcsharp}
são alguns exemplos de linguagens que possuem suporte 
para funções de alta ordem.

Um bom exemplo de simplicidade do uso de 
funções de alta ordem é a função \texttt{map}\cite{hofscala}. Seu objetivo 
é aplicar uma função a todos os elementos de uma 
coleção e retornar a nova coleção resultante. 
Para que isso seja possível, a função \texttt{map} precisa 
receber como parâmetro a função que será aplicada. 
O Código \ref{hof} demonstra o uso dessa função.

\begin{lstlisting}[caption={Exemplo de Função de Alta Ordem.},label=hof]

    def add1(x){
        return x + 1;
    }

    let result = map(add1, [1, 2, 3, 4, 5]);
    // O resultado dessa operação é a lista [2, 3, 4, 5, 6]
        

\end{lstlisting}
\legend{Fonte: O Autor (2021)}

Em uma linguagem que não aceita funções sendo 
passadas por parâmetro, uma operação simples como 
essa poderia se tornar mais verbosa e menos legível, 
como no Código \ref{nohof}.

\begin{lstlisting}[caption={Exemplo sem Funções de Alta Ordem.},label=nohof]

    def add1(x){
        return x + 1;
    }

    let mylist = [1, 2, 3, 4, 5]
    let result = []

    foreach(n : mylist) {
        result.push(add1(n))
    }

\end{lstlisting}
\legend{Fonte: O Autor (2021)}

\begin{comment}
Talvez a implementação da função map seja parecida 
com a função acima, porém, um programador que não 
conhece o programa levaria muito menos tempo para 
entender a primeira implementação do que a segunda. 
Além disso, para cada função diferente que poderia 
ser aplicada a essa mesma coleção, a mesma 
implementação teria que ser repetida.
\end{comment}



% Currying
\section{Currying}

\textit{Currying} é uma técnica de programação funcional que 
permite que uma função com mais de um parâmetro seja 
chamada como se possuísse apenas um\cite{realworldhaskell, functionalscala}. 
Por exemplo, a função \texttt{add} apresentada no Código 
\ref{nocurrex} poderia ser escrita como no Código 
\ref{currex}.

\begin{lstlisting}[caption={Exemplo sem \textit{Currying}.},label=nocurrex]

    def add(x, y){
        return x + y;
    }

\end{lstlisting}
\legend{Fonte: O Autor (2021)}

\begin{lstlisting}[caption={Exemplo de \textit{Currying}.},label=currex]

    def add(x){
        return y => x + y;
    }

\end{lstlisting}
\legend{Fonte: O Autor (2021)}

Dessa forma, o \textit{currying} faz com que qualquer 
função possa ser tratada como uma função unária. 
Essa técnica simplifica a composição de funções 
que possuem quantidades diferentes de parâmetros. 
Em algumas linguagens funcionais não é 
necessário refatorar o código como foi feito 
acima, já que as funções implementam essa técnica 
nativamente\cite{realworldhaskell}.


% Closures
\section{Closures}

Uma \textit{closure} é um comportamento que pode 
ocorrer em funções que retornam outras funções. Ela 
consiste no uso, por uma função retornada, de um valor 
que está no escopo da função que a retorna.\cite{mflambdas}.
O Código \ref{closureex} 
demonstra uma closure. Nele é definida a função \texttt{adder} que 
recebe como parâmetro um valor \textit{x} e retorna uma função 
que recebe como parâmetro outro valor \textit{y}, retornando a 
soma dos dois valores. A variável \textit{add10} receberá 
o retorno da chamada da função \texttt{adder} para o valor 10. 
Com isso, \texttt{add10} será uma função que receberá como 
parâmetro um número e adicionará 10 a ele. Quando 
\texttt{add10} é chamada com o valor 5 sendo passado como 
parâmetro, o retorno da função é 15.
Para que isso seja possível, a função retornada por 
\texttt{adder} precisou ter acesso ao valor da variável 
\textit{x} mesmo 
após o fim da execução de \texttt{adder}. Isso foi possível 
por a variável \textit{x} estar dentro do escopo da função 
quando ela foi criada.

\begin{lstlisting}[caption={Exemplo de \textit{Closure}.},label=closureex]

    def adder(x){
        return y => x + y;
    }

    let add10 = adder(10)

    res = add10(5)
    // O resultado acima é 15


\end{lstlisting}
\legend{Fonte: O Autor (2021)}


% Composição de funções
\section{Composição de Funções}

Reuso de código é um objetivo desejável para qualquer 
paradigma de programação, e o paradigma funcional 
aproveita a composição de funções para seguir esse 
princípio\cite{realworldhaskell}.
O Código \ref{fogex} exemplifica esse recurso.

\begin{lstlisting}[caption={Exemplo de Composição de Funções.},label=fogex]

    def add1(x){
        return x + 1
    }

    def mul2(x){
        return x * 2
    }

    def sub4(x){
        return x - 4
    }

    def add1ThenMul2ThenSub4(x) {
        return sub4(mul2(add1(x)))
    }

    let res = add1ThenMul2ThenSub4(1);
    // O resultado da função é 0

\end{lstlisting}
\legend{Fonte: O Autor (2021)}

É comum qualquer linguagem permitir esse tipo 
de comportamento. Entretanto, 
utilizar funções menores e mais simples para compor 
funções maiores e mais complicadas é uma forma de 
\textit{design} comum em linguagens funcionais. Uma vantagem é 
que em linguagens funcionais as composições podem 
tornar-se mais legíveis utilizando funções 
de alta ordem, como pode ser visto no Código 
\ref{hofogex} \footnote{Aqui, a função \texttt{compose} 
recebe as funções \texttt{add1} e \texttt{mul2} e retorna a 
composição delas. A função resultante 
é recebida como parâmetro de \texttt{compose} novamente, 
assim como a função \texttt{sub4}, resultando em uma função 
equivalente a \texttt{sub4(mul2(add1()))}}.

\begin{lstlisting}[caption={Exemplo de Composição de Funções.},label=hofogex]

    def add1(x){
        return x + 1
    }

    def mul2(x){
        return x * 2
    }

    def sub4(x){
        return x - 4
    }

    let res = (sub4 compose mul2 compose add1)(1);
    // O resultado da função é 0

\end{lstlisting}
\legend{Fonte: O Autor (2021)}


% ---
% Capitulo de revisão de literatura
% ---
\chapter{Padrões de Projeto}
% ---

Durante o processo de desenvolvimento de software, 
problemas de design são comuns durante a 
fase de projeto. Alguns desses problemas 
eram tão comuns que foi feito um esforço para catalogá-los 
em um livro \cite{gamma:1995} que oferece possíveis soluções para os mesmos. 
Essas soluções tornaram-se conhecidas como 
os Padrões de Projeto Gang of Four, abreviados 
para Padrões de Projeto GoF.

Por definição, um padrão de projeto é uma solução 
reutilizável para um problema comum de design. Apesar 
deste trabalho se restringir ao contexto de engenharia 
de software, o conceito foi introduzido pelo arquiteto 
Christopher Alexander no livro A Pattern Language 
\cite{alexanderpatternlanguage}.

Com foco no design orientado a objetos, hoje os 
padrões GoF estão entre os padrões de 
projeto de software mais conhecidos. Os responsáveis 
por compilá-los foram Erich Gamma, Richard Helm, 
Ralph Johson e John Vlissides, o que deu origem ao 
nome Gang of Four. Ao todo, vinte e três 
padrões foram catalogados, os mesmos que serão o alvo 
deste trabalho.

De acordo com o livro, um padrão possui quatro elementos 
essenciais: Um nome, um problema, uma solução e suas 
consequências. O nome é uma característica importante 
por tornar mais fácil referenciar um padrão. O problema 
descreve a situação em que o padrão é aplicado e 
a solução descreve como um conjunto de elementos pode 
resolver o problema proposto. Já as consequências 
mostram as vantagens e desvantagens do uso do padrão 
para um problema.

Como forma de organizar os padrões, o livro os separa 
por finalidade e por escopo. A separação por finalidade 
divide os padrões entre padrões criacionais, 
destinados ao processo de criação de objetos, padrões 
estruturais, que lidam com a forma em que o conjunto de 
classes e objetos está disposto e padrões comportamentais, 
focados na forma em que classes e objetos comunicam-se 
e distribuem suas responsabilidades. A separação por 
escopo divide os padrões no escopo de classe ou de objeto, 
onde o primeiro lida com a relação entre classes e 
subclasses através de herança, enquanto o segundo lida 
com formas de relacionamento mais dinâmicas entre os 
objetos, como delegação. Os padrões nesse trabalho 
serão separados apenas por finalidade, porém 
características que remetem ao escopo 
podem ser consideradas durante a análise.


\section{Exemplo de padrão de projeto: Singleton}

A descrição de cada padrão no livro segue uma estrutura 
muito similar, utilizada principalmente para apresentar 
os quatro elementos essenciais mencionados anteriormente. 
Como exemplo para demonstrar a forma como o livro 
apresenta cada padrão, o padrão criacional Singleton será 
demonstrado com uma breve explicação de cada tópico. 
Uma descrição mais sucinta dos outros padrões será 
apresentada durante o desenvolvimento deste 
trabalho, onde serão considerados apenas os elementos 
essenciais dos padrões na análise 
a partir do paradigma funcional.

\subsection*{Intenção}

A intenção é uma forma curta de descrever o que o padrão 
faz, qual é sua intenção e que problema ele busca resolver. 
O Singleton busca garantir que uma classe tenha apenas uma
instância, acessível globalmente.

\subsection*{Motivação}

Este tópico ilustra um problema e demonstra como a estrutura 
do padrão o soluciona, tornando mais simples a compreensão 
das descrições mais abstratas que vêm a seguir. Para o 
Singleton, é utilizado como exemplo o spooler de uma 
impressora, um sistema de arquivos ou um gerenciador de 
janelas. Para todos esses casos, apenas um precisa existir, 
ou seja, uma classe que representa algum desses elementos 
só precisa possuir uma instância de fácil acesso. É proposto 
tornar a própria classe responsável por gerenciar essa 
instância, garantindo que nenhum outra instância dela 
mesma seja criada e garantindo um meio de acesso a essa 
única instância.

\subsection*{Aplicabilidade}

A aplicabilidade descreve situações nas quais o padrão 
pode ser aplicado, exemplos de maus projetos que ele pode 
ajudar a tratar e ainda como reconhecer essas situações. 
No caso do Singleton, ele é utilizável quando for necessário 
possuir apenas uma instância de uma classe através de um 
ponto de acesso conhecido e quando essa única instância 
precisa ser extensível através de subclasses. 

\subsection*{Estrutura}

A estrutura apresenta o padrão graficamente, através de uma 
notação baseada na Object Modeling Technique (OMT) e às vezes 
em diagramas de interação. No caso do Singleton, apenas o 
seguinte diagrama é utilizado:

\begin{figure}[htb]
	\caption{\label{fig_grafico}Estrutura do Singleton utilizada como exemplo}
	\begin{center}
	    \includegraphics[scale=0.5]{4_referencial_teorico/2_padroes-projeto/singleton_structure.png}
    \end{center}
    \legend{Fonte: \cite{gamma:1995}}
\end{figure}

\subsection*{Participantes}

Descreve as responsabilidades de cada classe que 
participa do padrão. Neste caso, existe 
apenas uma: o próprio Singleton, que define 
a operação de classe Instance, permitindo aos clientes 
acessarem sua única instância. Também pode ser o 
responsável por criar sua própria instância única.

\subsection*{Colaborações}

Este tópico explica como as classes participantes 
colaboram para executar as responsabilidades 
especificadas. Para o Singleton, os clientes (ou seja, 
os objetos que o acessam) acessam a instância única 
pela operação Instance.

\subsection*{Consequências}

As consequências descrevem os custos e benefícios para 
que o padrão possa realizar seu objetivo, além dos 
aspectos da estrtura de um sistema que ele permite variar 
independentemente. O Singleton enumera cinco benefícios:

Primeiro, acesso controlado à instância única, já que 
a única instância é encapsulada dentro da classe 
Singleton, ela possui controle total de como e quando 
ela pode ser acessada pelos clientes.

Segundo, espaço de nomes reduzido. Uma alternativa 
para o Singleton talvez fosse o uso de variáveis 
globais, porém o padrão evita que o espaço de nomes 
seja poluído com variáveis globais que utilizam 
instâncias únicas.

Terceiro, ele permite um refinamento de operações e da 
representação, ou seja, permite ao Singleton ter 
subclasses.

Quarto, permite um número variável de instâncias. Nesse 
caso, o padrão permite que, após ele ser implementado, 
seja simples mudar de ideia e a própria classe Singleton, 
dentro da operação Instance, volte a permitir um número 
indefinido ou até controlado de instâncias.

Quinto, é mais flexível do que operações de classe. Além 
das variáveis globais, operações de classe seriam outra 
alternativa para o Singleton, porém isso tornaria mais 
difícil voltar a ter mais de uma instância da classe, 
além de impedir, em certas linguagens, que subclasses 
redefinam operações estáticas polimorficamente.

\subsection*{Implementação}

Explicita sugestões, técnicas ou riscos que devem 
ser conhecidos durante a implementação do padrão, 
além de considerações específicas de algumas 
linguagens. Para o Singleton, existem duas 
explicações de implementação.

A primeira refere-se à garantia da existência de 
apenas uma instância, onde é sugerido tornar a operação 
de criação do Singleton em uma operação de classe 
que possui acesso a um atributo que armazena a 
instância do Singleton, caso já exista. A segunda 
trata da criação de subclasses da classe Singleton, 
onde é sugerido registrar cada instância por nome 
para que uma classe cliente possa acessar o singleton 
desejado sem precisar conhecer todas as instâncias 
existentes. Ambas as implementações são 
exemplificadas na seção de exemplo de código.


\subsection*{Exemplo de Código}

Como o nome já diz, demonstra o padrão através de 
um exemplo em código. O exemplo do Singleton é um 
construtor de labirintos, onde a classe que é 
responsável pela fabricação dos labirintos necessita 
de apenas uma instância. São apresentadas duas versões: 
uma onde não há uso de subclasses e uma onde há o 
uso.

\begin{lstlisting}[caption={Exemplo de Singleton sem subclasses}, label=singletonnosub]
    
    class MazeFactory {
    public:
        static MazeFactory* Instance();

        // interface existente vai aqui
    protected:
        MazeFactory();
    private:
        static MazeFactory* _instance;
    };


    // implementação:

    MazeFactory* MazeFactory::_instance = 0;

    MazeFactory* MazeFactory::Instance () {
        if (_instance == 0) {
            _instance = new MazeFactory;
        } 
        return _instance;
    } 

\end{lstlisting}
\legend{Fonte: \cite{gamma:1995}}


\begin{lstlisting}[caption={Exemplo de Singleton com subclasses},label=singletonsub]
    
    MazeFactory* MazeFactory::Instance () {
        if (_instance == 0) {
            const char* mazeStyle = getenv("MAZESTYLE");

            if (strcmp(mazeStyle, "bombed") == 0) {
                _instance = new BombedMazeFactory;

            } else if (strcmp(mazeStyle, "enchanted") == 0) {
                _instance = new EnchantedMazeFactory;

            // ... outras subclasses possíveis

            } else { // default
                _instance = new MazeFactory;
            }
        }
        return _instance;
    } 

\end{lstlisting}
\legend{Fonte: \cite{gamma:1995}}


\subsection*{Usos Conhecidos}

Demonstra usos desse padrão em sistemas reais. No caso 
do Singleton, é mencionado o relacionamento entre 
classes e suas metaclasses e o toolkit para construção 
de interfaces de usuário InterViews, que usa o padrão para 
acessar as únicas instâncias das classes Session e 
WidketKit, entre outras.

\subsection*{Padrões Relacionados}

Os padrões relacionados apresentam padrões que 
possuem alguma relação ou que podem ser usados juntos 
do padrão proposto. São mencionados padrões que 
podem ser implementados utilizando o Singleton, que 
são o AbstractFactory, o Builder e o Prototype.





% ----------------------------------------------------------
% Essa parte é um requisito apenas para TCC I e deverá 
% ser removida após a aprovação na disciplina.
% ----------------------------------------------------------
\phantompart

\chapter{Resumo dos demais capítulos}

Na parte de desenvolvimento, haverá uma introdução 
sobre como alguns conceitos de orientação a objetos, 
como classes e encapsulamento, serão representados na 
linguagem funcional. Em seguida, os capítulos serão 
dividos entre padrões criacionais, estruturais e 
comportamentais. Ao todos, os vinte e três padrões 
GoF serão abordados nesses três capítulos, onde serão 
apresentadas as ideias básicas do problema que o padrão 
busca resolver e como o resolve, seguido da abordagem 
funcional de resolver o mesmo problema.

Após analisar todos os padrões, o capítulo de resultados 
analisará as vantagens e desvantagens da abordagem 
funcional para cada solução, destacando onde ela 
contribuiu, onde atrapalhou, ou onde não fazia sentido ser 
implementada. Essas classificações dependerão das análises 
que serão realizadas na etapa de desenvolvimento.

Por fim, no capítulo de conclusão serão analisadas 
as consequências dessas análises e como elas podem 
impactar o conceito de padrões de projeto e as vantagens 
e desvantagens de revisá-los no ponto de vista funcional.

% ----------------------------------------------------------
% Remover até aqui após TCC I
% ----------------------------------------------------------



% ----------------------------------------------------------
% PARTE
% ----------------------------------------------------------
%\part{Desenvolvimento}
% ----------------------------------------------------------

%
% ---
% Capitulo de revisão de literatura
% ---
\chapter{Orientação a Objetos no Contexto Funcional}
% ---

% Conceitos para mapear

Parte dos padrões de projeto que serão 
analisados dependem de conceitos 
de orientação a objetos como classes ou 
encapsulamento, o que torna necessário 
realizar um mapeamento desses conceitos 
para o paradigma funcional. A intenção 
desse mapeamento não é implementar 
orientação a objetos em uma linguagem 
funcional, mas entender qual é a utilidade 
de cada um desses conceitos e de que 
forma a programação funcional pode 
oferecer essa mesma utilidade. É importante 
ressaltar que esse mapeamento não será usado 
de forma metódica ao analisar os padrões. 
Ele é uma referência para contextualizar 
o leitor nos exemplos em código 
implementados de forma funcional. 


% classes e objetos
\section{Classes e Objetos}

Um objeto pode ser definido como uma representação 
do mundo real que possui características e comportamentos, 
enquanto uma classe é uma abstração dessa representação 
que define quais características e comportamentos um objeto 
deve possuir\cite{umlsystems}. Essas características 
e comportamentos são representados em orientação a 
objetos como atributos e métodos, respectivamente. 
O código \ref{ooclass} demonstra uma classe que 
possui os atributos nome e idade, além dos métodos 
getNome, setNome, getIdade e setIdade, que realizam 
operações com esses atributos.

\begin{lstlisting}[caption={Classe comum em Orientação a Objetos},label=ooclass]
    
    class Pessoa(var nome : String, var idade : Int){

        def getNome() : String = this.nome

        def setNome(nome : String) : Unit = this.nome = nome

        def getIdade() : Int = this.idade

        def setIdade(idade : Int) : Unit = this.idade = idade

    }   

\end{lstlisting}

Dessa forma, é necessário definir uma estrutura em 
programação funcional que possua características e 
funções que operam sobre essas características. 
Para agrupar características pode ser utilizada uma 
tupla, uma estrutura que armazena uma quantidade 
fixa de valores com tipos predefinidos\cite{tuplesscala}. 
Como as tuplas não podem ser modificadas, elas 
respeitam o conceito de imutabilidade das 
linguagens funcionais.

Para representar os métodos de uma classe em uma 
linguagem funcional, já que nossa estrutura de dados 
imutável não armazenará funções\footnote{Apesar de não 
ser uma abordagem utilizada para este mapeamento, é 
possível armazenar funções em tuplas.} e já 
que é necessário que nossas funções sejam puras, 
uma abordagem de implementação desses 
métodos é definir funções que recebam 
como parâmetro um valor do tipo definido em nossa 
estrutura de dados imutável. Seguindo esses dois 
princípios, uma versão funcional da classe apresentada 
no código \ref{ooclass} pode ser vista no código \ref{fpclass}.


\begin{lstlisting}[caption={Representação de uma classe no contexto funcional},label=fpclass]
    
    type Pessoa = (String, Int)

    def getNome(pessoa : Pessoa) : String = pessoa._1 

    def setNome(pessoa : Pessoa, name : String) : Pessoa = 
        (name, pessoa._2)

    def getIdade(pessoa : Pessoa) : Int = pessoa._2

    def setIdade(pessoa : Pessoa, age : Int) : Pessoa =
        (pessoa._1, age)

\end{lstlisting}

%associação
\section{Associação, Agregação e Composição}

Uma associação pode ser definida como uma 
conexão entre as classes que indica algum 
relacionamento entre elas\cite{Sommerville10}. 
O código \ref{ooassociation} demonstra uma 
associação entre as classes Cidade e Estado, onde 
a classe Estado possui uma coleção de atributos 
do tipo Cidade. Para que haja uma associação 
entre duas classes, basta que pelo menos 
uma delas tenha em seus atributos uma 
referência à outra.

\begin{lstlisting}[caption={Exemplo de associação entre classes},label=ooassociation]
    
    class Cidade(var nome : String){

        def getNome() : String = this.nome;
        def setNome(nome : String) : Unit {
            this.nome = nome;
        }
    }

    class Estado(var nome : String, var cidades : List[Cidade]){

        def getNome() : String = this.nome;
        def setNome(nome : String) : Unit {
            this.nome = nome;
        }
        def getCidades() : List[Cidade] = this.cidades;
        def addCidade(cidade : Cidade) : Unit {
            this.cidades = this.cidades :+ cidade;
        }
    }

\end{lstlisting}

Como foi visto anteriormente, os atributos 
podem ser representados por valores salvos 
dentro de uma tupla associada a um tipo. 
Portanto, uma associação dentro do contexto 
funcional pode ser implementada armazenando 
um valor de um tipo A entre os valores da tupla 
de um tipo B. O código \ref{fpassociation} 
demonstra o exemplo anterior implementado 
de forma funcional.

\begin{lstlisting}[caption={Exemplo de associação no contexto funcional},label=fpassociation]
    
    type Cidade = (String)
    
    def getNome(cidade : Cidade) : String = cidade._1;
    def setNome(cidade : Cidade, nome : String) : Cidade = (nome)

    type Estado = (String, List[Cidade])
    
    def getNome(estado : Estado) : String = estado._1;
    def setNome(estado : Estado, nome : String) : Estado = (nome, estado._2)
    
    def getCities(estado : Estado) : List[Cidade] = estado._2;
    def addCity(estado : Estado, cidade : Cidade) : Estado =
        (estado._1, estado._2 :+ cidade)

\end{lstlisting}

% encapsulamento
\section{Encapsulamento}

A abordagem da seção anterior implementa 
classes e objetos, porém precisa ser 
reavaliada para que possa levar em consideração 
o encapsulamento. Encapsulamento pode ser definido 
como uma forma de limitar o acesso a um conjunto 
de dados ou comportamentos de um objeto \cite{quarkoo}. 
A motivação para isso pode vir tanto da necessidade 
de concentrar as alterações externas que um objeto 
pode sofrer em apenas um lugar quanto evitar que 
esse objeto assuma um estado que não deveria ser 
representado. 

Com a ideia de imutabilidade, pode-se 
assumir que um valor não será alterado em partes 
diferentes de uma aplicação, mas é possível 
que funções responsáveis por criar ou modificar\footnote{
    Uma função que modifica um valor é entendida 
    como uma função que recebe um valor existente 
    por parâmetro e retorna um novo valor do mesmo 
    tipo.
} 
um valor de um determinado tipo estejam 
espalhadas pela aplicação, facilitando uma 
situação em que um estado que não deveria ser 
representado por esse valor seja criado. 
Dessa forma, implementar alguma forma de 
encapsulamento ainda é importante no 
contexto funcional.

Existe mais de uma abordagem que torna 
possível implementar o encapsulamento em 
linguagens funcionais, o uso de GADTs - 
\textit{Generalized Algebraic 
Data Types}\cite{existentialhaskell} é uma 
delas. Closures também podem 
ser utilizadas ao armazenar valores de 
atributos enquanto retorna as funções 
necessárias para acessá-los ou modificá-los. 
Um exemplo equivalente ao do código \ref{fpclass} 
pode ser visto no código \ref{fpclosure}, 
implementado utilizando a linguagem funcional Clojure. 
\cite{classlessjs} Nele, a função pessoa 
funciona como um construtor que recebe como 
parâmetro o nome e a idade de uma pessoa e 
retorna um dicionário com as funções para 
recuperar ou modificar o estado de uma pessoa.
As funções de modificação retornam uma nova versão 
da pessoa com o estado alterado.

\begin{lstlisting}[caption={Representação de uma classe com closures},label=fpclosure]
    
    (defn pessoa [nome idade]
        {:getNome nome
         :setNome (fn [_nome] (pessoa _nome idade))
         :getIdade idade
         :setIdade (fn [_idade] (pessoa nome _idade))})

\end{lstlisting}

Apesar de não ser um conceito de programação 
funcional, também é possível aproveitar a ideia 
de modularização para esconder detalhes de 
implementação \cite{mlmodules}. Por exemplo, o 
código \ref{modulesencap}, implementado em 
Haskell, mostra o tipo Pessoa com um construtor 
P. Apesar do tipo Pessoa ser exportado para fora do 
módulo, P não é, tornando impossível para qualquer 
função que acesse esse módulo criar algo do tipo 
Pessoa. Dessa forma, apenas a função newPessoa, 
também exportada pelo módulo, 
pode criar novos valores do tipo 
Pessoa. Funções implementadas dentro do módulo 
também podem deixar de ser exportadas, o que 
as tornaria semelhantes a métodos privados 
de uma classe.

\begin{lstlisting}[caption={Módulos como forma de encapsulamento},label=modulesencap]
    
    module Pessoa (Pessoa, newPessoa, getNome, setNome, getIdade, setIdade) where

    data Pessoa = P (String, Int)

    newPessoa :: String -> Int -> Pessoa
    newPessoa nome idade = P (nome, idade)

    getNome :: Pessoa -> String
    getNome (P (nome, _)) = nome

    setNome :: Pessoa -> String -> Pessoa
    setNome (P (_, idade)) nome = P (nome, idade)

    getIdade :: Pessoa -> Int
    getIdade (P (_, idade)) = idade

    setIdade :: Pessoa -> Int -> Pessoa
    setIdade (P (nome, _)) idade = P (nome, idade)

\end{lstlisting}

Todas essas abordagens são válidas para a 
implementação do encapsulamento, sendo a 
linguagem utilizada um fator mais decisivo 
do que a abordagem em si. Por exemplo, é 
mais simples implementar a abordagem de closures 
em Clojure por ela ser dinamicamente tipada, 
permitindo que um dicionário sem estrutura 
predefinida seja retornado. Linguagens que exigem 
uma definição mais estrita do tipo de retorno 
de uma função podem dificultar tanto a 
implementação dessas funções quanto seu uso 
no resto do programa.

Sendo o objetivo dessa seção demonstrar que 
o encapsulamento pode ser implementado e 
não definir como implementá-lo, 
a abordagem utilizada para o encapsulamento 
durante a análise dos padrões será 
omitida, a menos que ela seja relevante para 
sua implementação. Essa omissão 
também tem como objetivo não particularizar o 
método de encapsulamento utilizado durante a 
implementação dos exemplos.

% interfaces
\section{Interfaces}

Uma interface pode ser entendida como um contrato 
entre uma classe e o mundo externo, indicando que 
uma classe que implementa uma interface também 
implementará as operações definidas 
pela mesma\cite{oracleooconcepts}. 

Um exemplo do uso de interfaces é demonstrado no código 
\ref{oopinterface1}, 
onde a interface é necessária para garantir que as 
classes SomadorMaisUm e MultiplicadorPorDois implementem 
a operação calcular, que recebe como parâmetro 
um valor do tipo inteiro e retorna outro valor 
inteiro.

% Exemplo 1 de Interfaces

\begin{lstlisting}[caption={Interfaces em Orientação a Objetos},label=oopinterface1]
    
    trait ICalculaInteiro {
        def calcular(x : Int) : Int
    }

    class SomadorMaisUm extends ICalculaInteiro {
        def calcular(x : Int) : Int = x + 1
    }

    class MultiplicadorPorDois extends ICalculaInteiro {
        def calcular(x : Int) : Int = 2*x
    }

    def calcularInteiro(x : Int, calculador : ICalculaInteiro) : Int {
        return calculador.calcular(x)
    }

\end{lstlisting}

Utilizando funções de alta ordem e levando em 
consideração que as funções que representam nossos 
métodos não estão encapsulados em classes e 
não dependem de atributos, é possível substituir o 
objeto sendo passado por parâmetro na função 
calcularInteiro por uma função qualquer que recebe 
como parâmetro um valor inteiro e retorna outro 
valor inteiro. Essa alternativa pode ser vista 
no código \ref{fpinterface1}.

\begin{lstlisting}[caption={Interfaces em Programação Funcional},label=fpinterface1]
    
    def somaUm(x : Int) : Int = x + 1

    def multiplicaPorDois(x : Int) : Int = 2*x

    def calcularInteiro(x : Int, calcular : (Int => Int)) =
        calcular(x)
    
\end{lstlisting}



% herança
\section{Herança}

Quando é desejado que uma classe seja incluída ou 
utilizada como base para a criação de outra classe, 
usa-se a herança\cite{quarkoo}. Dessa forma, é 
possível criar implementações mais específicas 
de classes já existentes e reaproveitar o código. 
O exemplo \ref{ooinheritance} demonstra 
o uso da herança entre as classes Animal e Dog. 
Ao invés de reimplementar os métodos da classe 
Animal, a classe Cachorro usa herança para reaproveitá-los. 

\begin{lstlisting}[caption={Herança em Orientação a Objetos},label=ooinheritance]
    
    class Animal(var nome : String) {
        def getNome() : String = nome
        
        def comer() : String {
            return "Meu nome é " + nome + " e eu posso comer";
        }
    }

    class Cachorro extends Animal {
        
        def Cachorro(nome : String) {
            super(nome);
        }

        def latir() : String {
            return "Au! Meu nome é " + super.getNome();
        }

        def comer() : String {
            return super.comer() + "\nEu como comida de cachorro";
        }
    }

\end{lstlisting}

No contexto funcional, um comportamento semelhante 
pode ser alcançado através da composição. Um tipo A 
que deseja herdar as funcionalidades de um tipo B 
deve possuir uma instância desse mesmo tipo em seus 
atributos. Para os métodos do tipo A, basta que as 
funções do tipo B sejam compostas das funções 
necessárias do tipo A. O código \ref{fpinheritance} 
demonstra o exemplo anterior, onde um tipo Animal 
armazena um valor String que representa o nome 
enquanto o tipo Cachorro armazena um valor Animal. 
As funções latir e comer que recebem como parâmetro 
um valor do tipo Cachorro reutilizam as funções getNome 
e comer que recebem como parâmetro um valor do tipo 
Animal. Nesse exemplo, o tipo Animal representa 
uma classe pai e o tipo Cachorro uma class filha.

\begin{lstlisting}[caption={Herança em Programação Funcional},label=fpinheritance]
    
    type Animal = (String)

    def getName(animal : Animal) : String = animal._1;
    def eat(animal : Animal) : String = 
        "Meu nome é " + getNome(animal) + " e eu posso comer"

    type Cachorro = (Animal)

    def getAnimal(cachorro : Cachorro) : Animal = cachorro._1;
    def bark(cachorro : Cachorro) : String = 
        "Au! Meu nome é " + getName(getAnimal(cachorro))
    def eat(cachorro : Cachorro) : String = 
        eat(getAnimal(Cachorro)) + "\nEu como comida de cachorro";

\end{lstlisting}

É possível perceber que a implementação da herança 
assemelha-se à implementação de uma associação, 
por isso ela apresenta uma desvantagem: 
qualquer função do tipo que representa 
a classe pai necessitará de uma função 
intermediária do tipo que representa a classe 
filha para acessá-la. 
No contexto orientado a objetos, esses dois 
relacionamentos são diferentes, pois a 
herança trata-se de um relacionamento entre 
classes enquanto a associação é um relacionamento 
entre objetos\cite{umlsystems}. 

% ----------------------------------------------------------
% CRIACIONAIS
% ----------------------------------------------------------


%\chapter{Padrões Criacionais}

\section{Factory Method}

O padrão Factory Method define uma interface 
para criar objetos de forma que a responsabilidade 
para a criação desses objetos seja da classe que 
irá implementá-la. Dessa forma, versões diferentes 
ou implementações específicas de um mesmo tipo 
de objeto podem ser implementadas sem que a 
aplicação que utilizará essas implementações 
precise conhecê-las.

Na figura \ref{fmethod_struct} é demonstrada 
a estrutura do padrão, onde a classe abstrata Creator 
é responsável por definir a operação abstrata 
que cria o objeto, FactoryMethod. A classe 
ConcreteCreator herda dessa interface e 
implementa o FactoryMethod criando um objeto 
do tipo ConcreteProduct, que é uma implementação 
específica de Product.

\begin{figure}[htb]
	\caption{\label{fmethod_struct}Estrutura do Factory Method}
	\begin{center}
	    \includegraphics[scale=0.4]{5_padroes-contexto-funcional/5.1_criacionais/5.1.1_factory-method/diagram.png}
	\end{center}
\end{figure}


\subsection*{Exemplo Orientado a Objetos}

Como exemplo é apresentado um \textit{framework} 
que cria e apresenta para o usuário múltiplos 
documentos. Para isso, a classe abstrata 
Application é definida com a operação abstrata 
CreateDocument e possuindo uma lista de objetos 
que implementam a interface Document. A classe 
concreta MyDocument implementa Document e define 
um tipo de documento que pode ser utilizado pelo 
\textit{framework}, enquanto a classe concreta 
MyApplication herda de Application e implementa 
a operação CreateDocument para que ela crie 
um objeto do tipo MyDocument. O diagrama de classes 
para o exemplo pode ser visto na figura \ref{fmethod_example}, 
enquanto a implementação pode ser vista no 
código \ref{oofactory}.


\begin{figure}[htb]
	\caption{\label{fmethod_example}Exemplo de Factory Method}
	\begin{center}
	    \includegraphics[scale=0.4]{5_padroes-contexto-funcional/5.1_criacionais/5.1.1_factory-method/exemplo_factory.png}
	\end{center}
\end{figure}

\begin{lstlisting}[caption={Factory Method Orientado a Objetos},label=oofactory]
    
    abstract class Document {
        def Open() : Unit
        def Close() : Unit
        def Save() : Unit
        def Revert() : Unit
    }

    abstract class Application {
        
        var docs : List[Document] = List()

        def CreateDocument() : Unit

        def NewDocument() : Unit {
            var doc = CreateDocument()
            docs.Add(doc)
            doc.open()
        }

        def OpenDocument() : Unit {
            // Implementação de abertura de documento
        }

    }

    class MyApplication : Application {
        def CreateDocument() {
            return new MyDocument()
        }
    }

    class MyDocument : Document {
        // Implementação dos métodos abstratos de Document
    }

\end{lstlisting}




\subsection*{Contexto Funcional}

\begin{lstlisting}[caption={Factory Method Funcional},label=fpfactory]
    
    

\end{lstlisting}


%\subsection{Abstract Factory}


\begin{figure}[htb]
	\caption{\label{fig_grafico}Estrutura do Abstract Factory}
	\begin{center}
	    \includegraphics[scale=0.5]{5_padroes-contexto-funcional/5.3_comportamentais/5.3.9_strategy/diagram.png}
	\end{center}
\end{figure}

Exemplo Orientado a Objetos:

\begin{lstlisting}[caption={Abstract Factory Orientação a Objetos},label=oostrategy]
    

\end{lstlisting}

Contexto Funcional:

\begin{lstlisting}[caption={Abstract Factory Orientação a Objetos},label=ooabfactory]
    
    

\end{lstlisting}
%\section{Builder}

Quando é necessário criar objetos  
complexos, o padrão Builder retira a 
responsabilidade da criação do objeto e 
a coloca em classes separadas. Dessa forma, 
um mesmo processo de criação pode criar 
representações diferentes desse mesmo 
objeto. Essas novas classes devem ser 
independentes de todas as partes que 
compõem o objeto que será criado.

A figura \ref{builder_struct} apresenta 
a estrutura do Builder, onde a interface 
Builder representa uma classe responsável 
por criar uma parte de um objeto. A classe 
ConcreteBuilder, do tipo dessa interface, 
implementa os métodos de construção de 
uma parte do tipo Product. A classe Director 
é responsável por chamar o método de criação 
das classes que criam cada parte do objeto 
complexo.

\begin{figure}[htb]
	\caption{\label{builder_struct}Estrutura do Builder}
	\begin{center}
	    \includegraphics[scale=0.4]{5_padroes-contexto-funcional/5.1_criacionais/5.1.3_builder/diagram.png}
	\end{center}
\end{figure}

\subsection*{Exemplo Orientado a Objetos}

Como exemplo, podemos observar um leitor de 
documentos do tipo RTF (\textit{Rich Text Format}), 
que deve permitir a conversão de documentos RTF 
para outros formatos, como texto em ASCII ou em um 
\textit{widget} de texto que pode ser editado de 
forma iterativa. Como a quantidade de formatos 
possíveis é grande, deve ser possível adicionar 
novos formatos sem que seja necessário modificar 
a classe do leitor de documentos RTF. 

O diagrama de classes apresentado na imagem 
\ref{builder_exemplo} demonstra o uso do padrão 
Builder para esse exemplo. Para cada formato possível 
de conversão, uma nova classe Builder é criada. 
As classes ASCIIConverter, TeXConverter e 
TextWidgetConverter representam, respectivamente, 
os \textit{builders} para os conversores para texto 
em ASCII, LaTeX e \textit{widget} de texto. A classe 
RTFReader chama as operações de construção 
apenas dos conversores desejados. O exemplo de 
implementação dessa abordagem é apresentado no 
código \ref{oobuilder}.

\begin{figure}[htb]
	\caption{\label{builder_exemplo}Exemplo de Builder}
	\begin{center}
	    \includegraphics[scale=0.4]{5_padroes-contexto-funcional/5.1_criacionais/5.1.3_builder/exemplo_builder.png}
	\end{center}
\end{figure}

\begin{lstlisting}[caption={Builder Orientado a Objetos},label=oobuilder]

	trait TextConverter {
		def ConvertCharacter(char : Char)
		def ConvertFontChange(font : String)
		def ConvertParagraph()
	} 

	class TextConverter {
		
		private var asciiText : ASCIIText

		def ConvertCharacter(char : Char){
			// Conversão para ASCIIText
		}

		def GetASCIIText() : ASCIIText {
			return this.asciiText
		}
	} 

	class TeXConverter {
		
		private var texText : TeXText

		def ConvertCharacter(char : Char){
			// Conversão para TeXText
		}

		def ConvertFontChange(font : String) {
			// Conversão para TeXText
		}

		def ConvertParagraph() {
			// Conversão para TeXText
		}

		def GetTeXText() : TeXText {
			return this.texText
		}
	} 

	class TextWidgetConverter {
		
		private var textWidget : TextWidget

		def ConvertCharacter(char : Char){
			// Conversão para TextWidget
		}

		def ConvertFontChange(font : String) {
			// Conversão para TextWidget
		}

		def ConvertParagraph() {
			// Conversão para TextWidget
		}

		def GetTeXText() : TextWidget {
			return this.textWidget
		}
	} 

	class RTFReader() {

		private var builder : Builder

		def ParseRTF() {
			// ...

			for(t <- tokens) {
				t.Type match {
					case CHAR => builder.ConvertCharacter(t.Char)
					case FONT => builder.ConvertFontChange(t.Font)
					case PARA => builder.ConvertParagraph()
				}
			}

			// ...
		}

	}


\end{lstlisting}

\subsection*{Contexto Funcional}


\begin{lstlisting}[caption={Builder Funcional},label=fpbuilder]
    

    
\end{lstlisting}
%\section{Prototype}

\begin{figure}[htb]
	\caption{\label{fig_grafico}Estrutura do Prototype}
	\begin{center}
	    \includegraphics[scale=0.5]{5_padroes-contexto-funcional/5.1_criacionais/5.1.4_prototype/diagram.png}
	\end{center}
\end{figure}

Exemplo Orientado a Objetos:

\begin{lstlisting}[caption={Prototype Orientado a Objetos},label=ooprototype]



\end{lstlisting}

Contexto Funcional:


\begin{lstlisting}[caption={Prototype Funcional},label=fpprototype]
    

    
\end{lstlisting}
%\section{Singleton}

O padrão Singleton garante que um objeto possuirá apenas uma 
instância. Além disso, fornece um único ponto, acessível 
globalmente, a essa instância. Essa implementação é útil 
para implementar classes que fornecem serviços sem que seja 
necessário instanciar vários objetos idênticos em 
locais diferentes do código.

A figura \ref{singleton_struct} demonstra a implementação 
do padrão. A classe Singleton possui um método construtor 
privado e armazena no atributo estático uniqueInstance uma 
instância de Singleton. Através do método de classe 
Instance, é verificado se já existe uma instância 
armazenada no atributo uniqueInstance. Caso já exista, 
ela é retornada. Caso não, a instância única é criada 
para ser retornada nas chamadas posteriores de Instance.

\begin{figure}[htb]
	\caption{\label{singleton_struct}Estrutura do Singleton}
	\begin{center}
	    \includegraphics[scale=0.6]{5_padroes-contexto-funcional/5.1_criacionais/5.1.5_singleton/singleton_estrutura.png}
	\end{center}
\end{figure}

\subsection*{Exemplo Orientado a Objetos}

Uma classe define as operações para realizar transações com 
uma base de dados. Como a instância dela é idêntica independente 
do cliente que a utiliza, não existe a necessidade de replicar 
essas instâncias pelo código. Ela pode ser transformada em 
um Singleton, o que faz com que toda classe que deseja fazer 
uma transação na base de dados apenas solicite uma instância 
e realize as operações. A definição da classe do exemplo 
pode ser vista na figura \ref{singleton_exemplo} e no 
código \ref{oosingleton}.

\begin{figure}[htb]
	\caption{\label{singleton_exemplo}Exemplo de Singleton}
	\begin{center}
	    \includegraphics[scale=0.6]{5_padroes-contexto-funcional/5.1_criacionais/5.1.5_singleton/singleton_exemplo.png}
	\end{center}
\end{figure}

\begin{lstlisting}[caption={Singleton Orientação a Objetos},label=oosingleton]

class Database private(){
  def Query(sql : String) : Object = {
    //Execute query
    null
  }
  def Command(sql : String) : Unit = {
    //Execute command
  }
}

object Database {
  private var instance : Database = null

  def Instance() : Database = {
    if(instance == null){
      instance = new Database()
    }
    instance
  }
}

\end{lstlisting}

\subsection*{Contexto Funcional}

\begin{comment}

Não existe uma forma de implementar o Singleton no contexto funcional 
por que ele viola o conceito de função pura, ou seja, a função não 
está mais dependendo apenas de seus parâmetros, mas também de um 
valor global que ainda pode ter seu estado modificado.

Porém, ainda existem formas de alcançar seu objetivo, ou seja, 
oferecer acesso a um serviço em diversos locais do código sem a necessidade 
de repeti-lo. A primeira forma é usando um conceito que não é exlusivamente funcional, 
já que até no contexto orientdo a objetos é considerado um bom substituto 
para o Singleton. Porém, por ser uma abordagem também utilizada por 
programas que seguem o paradigma funcional e consequentemente por não 
violar o paradigma, será mencionado como uma possível solução.

A abordagem consiste no uso da Injeção de Dependência, onde a criação de 
recurso utilizado por uma função ou objeto não é responsabilidade da mesma, 
ao invés disso, esse recurso é injetado, seja pelo construtor (no caso da 
orientação a objetos) ou por parâmetros de uma função (no caso do paradigma 
funcional).

\begin{lstlisting}[caption={Injeção de Dependência funcional},label=fpdi]
    
    

\end{lstlisting}

A segunda abordagem consiste na utilização de um Monad conhecido como 
Reader. As funções que precisam utilizar um determinado serviço são 
encapsuladas em um Monad. O estado desse serviço será acessável dentro 
dessas funções e sempre que suas execuções terminarem, o novo estado 
do serviço será retornado. Dessa forma, a próxima função que deseja 
utilizar o serviço poderá usufruir do estado atualizado.

\begin{lstlisting}[caption={Monad Reader},label=fpreader]
    
    

\end{lstlisting}

Essa abordagem tem algumas vantagens se comparada à injeção de dependência: 
Suponha que três funções são encadeadas em um programa. A primeira e a terceira 
precisarão utilizar o serviço que é injetado através dos parâmetros. A segunda 
função, mesmo sem utilizar o serviço, precisará recebê-lo em seus parâmetros 
para que ele seja passado para a terceira função. Isso diminui a reusabilidade 
dessa função, que poderia ser reaproveitada em um contexto onde o serviço não 
é necessário. Também há a poluição visual ao incluir, em diversas funções, 
parâmetros diferentes para fornecer os serviços. Em casos em que mais de um serviço 
é utilizado, a situação torna-se ainda mais caótica.

\end{comment}

% ----------------------------------------------------------
% ESTRUTURAIS
% ----------------------------------------------------------


%\section{Estruturais}

\subsection{Adapter}

\begin{figure}[htb]
	\caption{\label{fig_grafico}Estrutura do Adapter}
	\begin{center}
	    \includegraphics[scale=0.5]{5_padroes-contexto-funcional/5.2_estruturais/5.2.1_adapter/diagram.png}
	\end{center}
\end{figure}

Exemplo Orientado a Objetos:

\begin{lstlisting}[caption={Adapter Orientado a Objetos},label=ooadapter]



\end{lstlisting}

Contexto Funcional:


\begin{lstlisting}[caption={Adapter Funcional},label=fpadapter]
    

    
\end{lstlisting}
%\section{Bridge}

O padrão \textit{Bridge} permite variar as abstrações e as 
implementações de uma solução de forma independente, 
definindo uma interface que serve como ponte entre ambas. 
As operações da implementação são delegadas para essa 
nova interface, permitindo que as abstrações sejam 
implementadas sem precisar conhecer o tipo de 
implementação que está sendo utilizado. A estrutura do 
padrão pode ser vista na Figura \ref{bridge_struct}.\cite{gamma:1995}

\begin{figure}[htb]
	\caption{\label{bridge_struct}Estrutura do \textit{Bridge}.}
	\begin{center}
	    \includegraphics[scale=0.5]{5_padroes-contexto-funcional/5.2_estruturais/5.2.2_bridge/bridge_estrutura.png}
	\end{center}
  \caption*{Fonte: O Autor (2021)}
\end{figure}

\subsection*{Exemplo Orientado a Objetos}

Como exemplo, pode ser considerada a implementação de 
uma janela em um \textit{toolkit} para construir interfaces 
de usuários que permite o uso de implementações diferentes 
de janela: \textit{PM} e \textit{XWindow}. Além disso, é preciso definir tipos 
diferentes de janela, como janelas para ícones e janelas 
transitórias. Para que não seja necessário implementar 
uma versão diferente de janela de ícone e transitória 
para cada implementação diferente de janela, o padrão 
\textit{Bridge} pode ser usado para separar a implementação 
da abstração em duas hierarquias diferentes. O diagrama 
de classes da Figura \ref{bridge_exemplo} e o Código 
\ref{oobridge} demonstram o uso do padrão para esse 
exemplo.

\begin{figure}[htb]
	\caption{\label{bridge_exemplo}Exemplo de \textit{Bridge}.}
	\begin{center}
	    \includegraphics[scale=0.5]{5_padroes-contexto-funcional/5.2_estruturais/5.2.2_bridge/bridge_exemplo.png}
	\end{center}
  \caption*{Fonte: O Autor (2021)}
\end{figure}

\begin{lstlisting}[caption={\textit{Bridge} Orientado a Objetos.},label=oobridge]

abstract class Window(imp : WindowImp) {
  def DrawText() : Unit = {
    imp.DevDrawText()
  }

  def DrawRect() : Unit = {
    imp.DevDrawLine()
    imp.DevDrawLine()
    imp.DevDrawLine()
    imp.DevDrawLine()
  }
}

class IconWindow(imp : WindowImp) extends Window(imp) {
  def DrawBorder() : Unit = {
    DrawRect()
    DrawText()
  }
}

class TransientWindow(imp : WindowImp) extends Window(imp){
  def DrawCloseBox() : Unit = {
    DrawRect()
  }
}

trait WindowImp {
  def DevDrawText()
  def DevDrawLine()
}

class XWindowImp extends WindowImp {
  def DevDrawText() : Unit = {
    //Desenha texto para janela X
  }
  def DevDrawLine() : Unit = {
    //Desenha linha para janela X
  }
}

class PMWindowImp extends WindowImp {
  def DevDrawLine(): Unit = {
    //Desenha linha para janela PM
  }
  def DevDrawText(): Unit = {
    //Desenha texo para janela PM
  }
}

\end{lstlisting}
\legend{Fonte: O Autor (2021)}

\subsection*{Contexto Funcional}

Funções de alta ordem também podem ser utilizadas 
para separar as abstrações das implementações. 
No Código \ref{fpbridgeabs}, são definidas nas 
linhas 2 e 10, respectivamente, as 
duas funções equivalentes aos métodos das 
classes \texttt{IconWindow} e \texttt{TransientWindow} do exemplo 
orientado a objetos. Ao invés de reutilizar 
funções de uma superclasse, as funções 
\texttt{DrawText} e \texttt{DrawRect} são recebidas por parâmetro.

\begin{lstlisting}[caption={Abstrações no \textit{Bridge} Funcional.},label=fpbridgeabs]
    
def DrawIconBorder(text : String,
                   height : Int, width : Int,
                   DrawText : String => Unit,
                   DrawRect : (Int, Int) => Unit) : Unit = {
  DrawText(text)
  DrawRect(height, width)
}

def DrawTransientCloseBox(height : Int, width : Int,
                          DrawRect : (Int, Int) => Unit) : Unit = {
  DrawRect(height, width)
}
    
\end{lstlisting}
\legend{Fonte: O Autor (2021)}

No Código \ref{fpbridgeimp}, são definidas as 
funções referentes às implementações diferentes 
de \texttt{Window}. Essas são as funções que podem ser 
passadas como parâmetro para as funções vistas 
no Código \ref{fpbridgeabs}. Dessa forma, as 
abstrações de \texttt{Window} tornam-se independentes 
da forma como ela é implementada, resolvendo 
o problema encontrado pelo padrão.

\begin{lstlisting}[caption={Implementações no \textit{Bridge} Funcional.},label=fpbridgeimp]
    
def XDrawLine(size : Int) : Unit = {
  // Desenha linha
}

def XDrawText(text : String) : Unit = {
  // Desenha texto
}

def PMDrawLine(size : Int) : Unit = {
  // Desenha linha
}

def PMDrawText(text : String) : Unit = {
  // Desenha texto
}
    
\end{lstlisting}
\legend{Fonte: O Autor (2021)}
%\section{Composite}

Esse padrão fornece uma estrutura de objetos 
organizados como uma árvore, representados 
por uma hierarquia parte-todo. 
Com isso, é possível tratar tanto o conjunto 
quanto os objetos individuais de forma 
uniforme, sem que seja necessário conhecer 
os elementos pertencentes a um conjunto para 
tratá-lo.

A figura \ref{composite_struct} demonstra a 
estrutura do padrão, onde uma interface Component 
define tanto um objeto nó, representado pela 
classe Composite, quanto um objeto folha, 
representado pela classe Leaf. Os elementos 
filhos da classe Composite são todos instâncias 
de Component, o que faz com que a classe não 
saiba se seus filhos são outros objetos compostos 
ou se são objetos folha.

\begin{figure}[htb]
	\caption{\label{composite_struct}Estrutura do Composite}
	\begin{center}
	    \includegraphics[scale=0.5]{5_padroes-contexto-funcional/5.2_estruturais/5.2.3_composite/diagram.png}
	\end{center}
\end{figure}

\subsection*{Exemplo Orientado a Objetos}

Como exemplo, é apresentada uma ferramenta gráfica 
onde o usuário pode agrupar diversas formas e 
elementos 
para formar diagramas maiores e mais complexos. 
Apesar do usuário tratar esses diagramas como um 
único elemento gráfico, a aplicação precisa levar 
em consideração todos os elementos dos quais ele 
é composto. Dessa forma, o padrão Composite permite 
abstrair os elementos menores, tratando o elemento 
composto como algo único, da mesma forma que 
são tratados os elementos não compostos. A figura 
\ref{composite_exemplo} demonstra o diagrama de 
classes para esse exemplo, enquanto o código 
\ref{oocomposite} traz um exemplo de implementação.

\begin{figure}[htb]
	\caption{\label{composite_exemplo}Exemplo de Composite}
	\begin{center}
	    \includegraphics[scale=0.45]{5_padroes-contexto-funcional/5.2_estruturais/5.2.3_composite/exemplo_composite.png}
	\end{center}
\end{figure}

\begin{lstlisting}[caption={Composite Orientado a Objetos},label=oocomposite]

	trait Graphic {
		def Draw();
		def Add(graphic : Graphic);
		def Remove(graphic : Graphic);
		def GetChild(pos : Int) : Graphic;
	}

	class Picture() extends Graphic {
		
		var graphics : List[Graphic]
	
		def Draw() {
			// desenha o elemento na tela
		}

		def Add(graphic : Graphic) {
			graphics.add(graphic);
		}

		def Remove(graphic : Graphic) {
			graphics.remove(graphic);
		}

		def GetChild(pos : Int) {
			return graphics.get(pos);
		}
	}

	class Text() extends Graphic {
		def Draw() {
			// desenha o elemento na tela
		}
	}

	class Rectangle() extends Graphic {
		def Draw() {
			// desenha o elemento na tela
		}
	}

	class Line() extends Graphic {
		def Draw() {
			// desenha o elemento na tela
		}
	}


\end{lstlisting}

\subsection*{Contexto Funcional}



\begin{lstlisting}[caption={Composite Funcional},label=fpcomposite]
    

    
\end{lstlisting}
%\section{Decorator}

O padrão Decorator permite adicionar responsabilidades a um 
objeto de forma dinâmica. Essa dinamicidade é alcançada 
substituindo a herança por uma agregação, permitindo que a 
classe decorada delegue responsabilidades para as classes que 
a extendem. As classes de extensão implementam uma mesma 
interface que as classes decoradas e possuem um objeto dessa 
mesma classe entre seus atributos. Dessa forma, uma classe 
de extensão pode tanto referenciar outra classe de extensão 
quanto o objeto decorado, formando uma estrutura de pilha 
onde o elemento ao fundo é o objeto decorado que será o 
alvo das operações de todos os extensores presentes na 
estrutura.

O maior problema resolvido pelo Decorator é a grande 
quantidade de classes que deveriam existir caso houvessem 
muitas extensões para uma classe. O problema cresce ainda 
mais quando é necessário que essas funcionalidades mudem 
dinamicamente, gerando diversas combinações de grupos de 
funcionalidades possíveis.

\begin{figure}[htb]
	\caption{\label{fig_grafico}Estrutura do Decorator}
	\begin{center}
	    \includegraphics[scale=0.5]{5_padroes-contexto-funcional/5.2_estruturais/5.2.4_decorator/diagram.png}
	\end{center}
\end{figure}

Exemplo Orientado a Objetos:

\begin{lstlisting}[caption={Decorator Orientado a Objetos},label=oodecorator]



\end{lstlisting}

Contexto Funcional:

O mesmo objetivo é alcançado de forma simples através de 
composição de funções. Caso um valor precise ser decorado 
com diversas funções, uma função recebe esse valor como 
parâmetro e uma lista com todas as funcionalidades que irão 
estendê-lo. Essas funções são então chamadas uma por uma, 
gerando também uma pilha de chamadas que finalmente 
retorna o resultado da combinação de todas as operações.

\begin{lstlisting}[caption={Decorator Funcional},label=fpdecorator]
    

    
\end{lstlisting}
%\subsection{Façade}



\begin{figure}[htb]
	\caption{\label{fig_grafico}Estrutura do Façade}
	\begin{center}
	    \includegraphics[scale=0.42]{5_padroes-contexto-funcional/5.2_estruturais/5.2.5_facade/diagram.png}
	\end{center}
\end{figure}

Exemplo Orientado a Objetos:

\begin{lstlisting}[caption={Façade Orientado a Objetos},label=oofacade]



\end{lstlisting}

Contexto Funcional:


\begin{lstlisting}[caption={Façade Funcional},label=fpfacade]
    

    
\end{lstlisting}
%\subsection{Flyweight}

O padrão Flyweight permite economizar o espaço em memória 
da aplicação ao fornecer uma instância compartilhada de 
uma classe, para que ela não precise ser instanciada 
diversas vezes.

\begin{figure}[htb]
	\caption{\label{fig_grafico}Estrutura do Flyweight}
	\begin{center}
	    \includegraphics[scale=0.5]{5_padroes-contexto-funcional/5.2_estruturais/5.2.6_flyweight/diagram.png}
	\end{center}
\end{figure}

Exemplo Orientado a Objetos:

\begin{lstlisting}[caption={Flyweight Orientado a Objetos},label=ooflyweight]



\end{lstlisting}

Contexto Funcional:

A ideia do Flyweight assemelha-se à de memoização, onde o 
retorno de uma função pura é armazenado para que seu valor 
não precise ser recalculado quando os mesmos parâmetros 
são passados. Essa abordagem só é possível para funções 
puras pois, caso ocorram efeitos colaterais ou a função 
dependa de dados externos, o resultado pode ser diferente 
para os mesmos parâmetros, gerando um resultado não 
confiável.

Apesar da ideia de memoização parecer mais focada no tempo 
de execução no que no espaço em memória, dependendo da 
implementação é possível economizar ambos.

\begin{lstlisting}[caption={Flyweight Funcional},label=fpflyweight]
    

    
\end{lstlisting}
%\section{Proxy}

\begin{figure}[htb]
	\caption{\label{proxy_struct}Estrutura do Proxy}
	\begin{center}
	    \includegraphics[scale=0.5]{5_padroes-contexto-funcional/5.2_estruturais/5.2.7_proxy/diagram.png}
	\end{center}
\end{figure}

Exemplo Orientado a Objetos:

\begin{lstlisting}[caption={Proxy Orientado a Objetos},label=ooproxy]



\end{lstlisting}

Contexto Funcional:


\begin{lstlisting}[caption={Proxy Funcional},label=fpproxy]
    

    
\end{lstlisting}

% ----------------------------------------------------------
% COMPORTAMENTAIS
% ----------------------------------------------------------


%\chapter{Padrões Comportamentais}

\section{Chain of Responsibility}

Chain of Responsability propõe criar uma estrutura para 
tratar solicitações feitas por um objeto cliente. As classes 
que tratam as solicitações são chamadas de \textit{handlers}. 
Uma solicitação é passada adiante por uma cadeia de 
\textit{handlers} até que seja tratada ou 
até que a cadeia chegue ao fim, retornando uma 
indicação de que a solicitação não pôde ser atendida.

Essa abordagem permite desacoplar os clientes das classes 
que tratam as solicitações e permite que os \textit{handlers} 
sejam definidos dinamicamente. Por outro lado, pode não ser 
possível prever se uma solicitação de um cliente será atendida, 
caso um \textit{handler} adequado não esteja na cadeia. A 
estrutura do padrão pode ser vista na figura \ref{chain_struct}.

\begin{figure}[htb]
	\caption{\label{chain_struct}Estrutura do Chain of Responsibility}
	\begin{center}
	    \includegraphics[scale=0.5]{5_padroes-contexto-funcional/5.3_comportamentais/5.3.01_chain-of-responsibility/chainofresponsibility_struct.png}
	\end{center}
\end{figure}

\subsection*{Exemplo Orientado a Objetos}

O exemplo do Chain of Responsibility traz um recurso 
de ajuda utilizado nos componentes de uma 
interface gráfica. O recurso é sensível ao contexto, 
bastando que o usuário solicite a ajuda no local 
desejado. O objeto que fornece a ajuda não é 
conhecido pelos objetos dos \textit{widgets} da 
interface, ele pertence a uma cadeia que é 
percorrida sempre que o usuário solicita 
a ajuda. A figura \ref{chain_exemplo} e o código 
\ref{oochresponsibility} demonstram esse exemplo.

\begin{figure}[htb]
	\caption{\label{chain_exemplo}Exemplo de Chain of Responsibility}
	\begin{center}
	    \includegraphics[scale=0.5]{5_padroes-contexto-funcional/5.3_comportamentais/5.3.01_chain-of-responsibility/chainofresponsibility_exemplo.png}
	\end{center}
\end{figure}

\begin{lstlisting}[caption={Chain of Responsibility Orientação a Objetos},label=oochresponsibility]

class HelpHandler(handler: HelpHandler = null, topic : Topic.Value = Topic.NO_HELP_TOPIC) {

  private var _handler : HelpHandler = handler
  private var _topic : Topic.Value = topic

  def HasHelp(): Boolean = {
    _topic != Topic.NO_HELP_TOPIC
  }

  def SetHandler(handler : HelpHandler, topic : Topic.Value) : Unit = {
    this._handler = handler
    this._topic = topic
  }

  def HandleHelp() : Unit = {
    if(_handler != null){
      _handler.HandleHelp()
    }
  }
}

class Widget(parent : Widget, topic : Topic.Value = Topic.NO_HELP_TOPIC)
  extends HelpHandler(parent, topic)
  
class Button(parent : Widget, topic: Topic.Value = Topic.NO_HELP_TOPIC)
  extends Widget(parent, topic) {

  override def HandleHelp(): Unit = {
    if(HasHelp()){
      //Oferece ajuda sobre o botão
    } else {
      parent.HandleHelp()
    }
  }
}

class Dialog(handler : HelpHandler, topic : Topic.Value = Topic.NO_HELP_TOPIC)
  extends Widget(null) {

  SetHandler(handler, topic)

  override def HandleHelp(): Unit = {
    if(HasHelp()) {
      // Oferece ajuda sobre dialog
    } else {
      handler.HandleHelp()
    }
  }
}

class Application(topic: Topic.Value)
  extends HelpHandler(null, topic) {

  override def HandleHelp(): Unit = {
    //Apresenta uma lista de tópicos de ajuda
  }
}

\end{lstlisting}

\subsection*{Contexto Funcional}

O Chain of Responsability encadeia funções que 
tratam solicitações, de forma que uma próxima 
função seja chamada quando a solicitação não 
pode ser tratada. É possível implementá-lo 
utilizando funções de alta ordem, onde um 
\textit{handler} é uma função, ao invés de 
uma classe, que armazena em uma \textit{closure} 
a função do próximo elemento da cadeia. 

O código \ref{fpchresponsibility} demonstra a 
criação dos \textit{handlers}. Na linha 2, a 
função HandleButton recebe como parâmetro um 
tópico e o próximo \textit{handler} da cadeia, 
assim como a classe Button do exemplo orientado 
a objetos. Ela retorna uma função que verifica 
se a solicitação pode ser tratada e, caso 
não seja, chama o \textit{handler} armazenado. 
A função HandleDialog, na linha 12, é 
implementada de forma análoga. Já a função 
HandleHelp, na linha 22, não recebe novos 
\textit{handlers}, oferecendo ajuda de forma 
genérica, da mesma forma que a classe 
Application do exemplo orientado a objetos.

\begin{lstlisting}[caption={Chain of Responsibility Funcional},label=fpchresponsibility]
    
def HandleButton(topic : Topic.Value, handler : () => Unit) : () => Unit = {
  () => {
    if(HasHelp(topic)){
      //Oferece ajuda sobre o botão
    } else {
      handler()
    }
  }
}

def HandleDialog(topic : Topic.Value, handler : () => Unit) : () => Unit = {
  () => {
    if(HasHelp(topic)){
      //Oferece ajuda sobre o dialog
    } else {
      handler()
    }
  }
}

def HandleHelp() : Unit = {
  //Apresenta uma lista de tópicos de ajuda
}
    
\end{lstlisting}

O código \ref{fpchainfunction} demonstra a criação 
da cadeia, onde as funções apresentadas anteriormente 
são executadas e encadeadas, criando a função 
ChainofResponsibility que será utilizada para 
tratar todas as solicitações.

\begin{lstlisting}[caption={Função Chain of Responsability},label=fpchainfunction]
    
val ChainofResponsibility: () => Unit = HandleButton(Topic.BUTTON,
  HandleDialog(Topic.DIALOG,
    HandleHelp))
      
\end{lstlisting}
%\section{Command}

O padrão Command permite encapsular operações 
em objetos. Isso permite que seja mantido um 
histórico das operações realizadas, que seja 
criada uma sequência de operações a serem 
executadas e até mesmo que operações realizadas 
possam ser desfeitas.

Para alcançar isso, uma classe Command armazena o 
objeto alvo da operação e define uma operação 
de execução e uma de reversão, quando necessário. 
Uma outra classe pode ser responsável por armazenar 
uma coleção de \textit{commands}, mantendo o 
histórico ou sequência de operações. 

Esse padrão funciona como uma solução para definir 
\textit{callbacks}, ou seja, operações que podem ser 
predefinidas e executadas futuramente no código. Sua 
estrutura pode ser vista na imagem \ref{command_struct}.

\begin{figure}[htb]
	\caption{\label{command_struct}Estrutura do Command}
	\begin{center}
	    \includegraphics[scale=0.5]{5_padroes-contexto-funcional/5.3_comportamentais/5.3.02_command/command_struct.png}
	\end{center}
\end{figure}

\subsection*{Exemplo Orientado a Objetos}

O exemplo do Command traz um \textit{toolkit} para 
construção de interfaces de usuário, onde ao clicar 
ou realizar uma ação sobre botões e menus, uma 
operação deve ser executada. Porém, os botões e 
menus não devem conhecer essas operações, elas são 
definidas pelo desenvolvedor que utiliza os toolkits. 
Dessa forma, o padrão Command permite isolar as 
operações dos \textit{widgets}. O exemplo traz as 
operações PasteCommand, que tem como alvo um 
documento da aplicação, e OpenCommand, que tem como 
alvo o objeto da aplicação. Além disso, há um 
MacroCommand que permite executar uma sequência 
de comandos. O exemplo é apresentado no diagrama da 
imagem \ref{command_exemplo} e no código \ref{oocommand}.

\begin{figure}[htb]
	\caption{\label{command_exemplo}Exemplo de Command}
	\begin{center}
	    \includegraphics[scale=0.5]{5_padroes-contexto-funcional/5.3_comportamentais/5.3.02_command/command_exemplo.png}
	\end{center}
\end{figure}

\begin{lstlisting}[caption={Command Orientação a Objetos},label=oocommand]

abstract class Command {
  def Execute()
}

class OpenCommand(var application : Application) extends Command {
  def Execute(): Unit = {
    var name = AskUser()
    if(name != null){
      val document = new Document(name)
      application.Add(document)
      document.Open()
    }
  }

  def AskUser() : String = {
    //Solicita ao usuário o arquivo que será aberto
  }
}

class PasteCommand(var document : Document) extends Command {
  def Execute(): Unit = {
    document.Paste()
  }
}

class MacroCommand extends Command {

  private var commands : List[Command] = List()

  def Execute(): Unit = {
    commands.foreach(command => {
      command.Execute()
    })
  }
}
    
\end{lstlisting}

\subsection*{Contexto Funcional}

A a intenção do Command é encapsular operações 
em objetos. Com as funções de alta ordem da programação 
funcional, essa funcionalidade já é alcançada. 
Entretanto, existem duas particularidades 
do padrão Command que devem ser consideradas. 

O padrão encapsula, junto da operação, uma 
referência para o objeto alvo. Dessa forma, 
quando a operação é realizada, ocorre 
um efeito colateral, desencorajado 
no contexto funcional. Para evitar isso, o 
comando deve receber, ao ser executado, o 
valor alvo como parâmetro e retornar o 
valor atualizado.

A segunda particularidade é que o padrão 
também permite uma operação de desfazer. Como 
tanto a operação de fazer quanto a de desfazer 
são encapsuladas em um mesmo objeto, é necessário 
possuir um valor que armazena ambas as operações 
no contexto funcional. Isso pode ser feito 
através de uma tupla que armazena as duas 
operações.

O código \ref{fpcommand} demonstra uma implementação 
genérica do padrão no contexto funcional. Na linha 2, 
é definido um tipo Command que é uma tupla que 
armazena duas funções: fazer e desfazer. A função 
CreateCommand, na linha 4, é uma função auxiliar para 
a criação de um command. Caso não seja fornecida 
uma função de desfazer, a função é substituída por 
uma função identidade que recebe um valor como 
parâmetro e retorna esse mesmo valor.

As funções auxiliares Execute, na linha 12, e 
Unexecute, na linha 14, recebem como parâmetro o valor 
alvo e um comando. Elas executam a primeira e a 
segunda função armazenadas na tupla Command, 
respectivamente.

A função CommandMany, da linha 16, é uma função auxiliar 
para executar uma lista de \textit{commands}. Ela recebe 
como parâmetro um alvo, uma lista de \textit{commands} e 
uma operação que recebe um alvo e um comando. Essa 
operação genérica é utilizada para que essa função 
possa ser reaproveitada tanto para executar uma 
sequência de \textit{commands} quando desfazê-los. 
As funções ExecuteMany, na linha 20, e UnexecuteMany, 
na linha 23, chamam a função CommandMany passando 
como operação as função Execute e Unexecute, 
respectivamente.


\begin{lstlisting}[caption={Command Funcional},label=fpcommand]
    
type Command[A] = (A => A, A => A)

def CreateCommand[A](Do : (A) => A,
                     Undo : Option[(A) => A] = None) : Command[A] =
  (Do,
    Undo match {
      case Some(function) => function
      case None => (x) => x
  })

def Execute[A](target : A, command : Command[A]) : A = command._1(target)

def Unexecute[A](target : A, command: Command[A]) : A = command._2(target)

def CommandMany[A](target : A, commands : List[Command[A]], operation : (A, Command[A]) => A): A =
    if(commands.isEmpty) target
    else CommandMany(operation(target, commands.head), commands.tail, operation)

  def ExecuteMany[A](target : A, commands : List[Command[A]]) : A =
    CommandMany(target, commands, Execute)

  def UnexecuteMany[A](target : A, commands : List[Command[A]]) : A = {
    CommandMany(target, commands, Unexecute)
    
\end{lstlisting}

O código \ref{fpcommandexample} demonstra como o exemplo 
orientado a objetos poderia ser implementado a partir das 
funções auxiliares vistas. O valor ExecuteOpen, visto 
na linha 2, implementa a abertura de um documento em uma 
aplicação e retorna o estado atualizado dessa aplicação 
com o documento aberto. Já a função ExecutePaste, na 
linha 8, realiza a operação de colar um documento. 
O valor resultingApplication na linha 14 é o estado 
resultante da aplicação após a execução de ambos os 
comandos através da função auxiliar ExecuteMany.

\begin{lstlisting}[caption={Exemplo funcional de Command},label=fpcommandexample]
    
val ExecuteOpen = CreateCommand[Application](
  (target : Application) => {
    //Executa comando de abrir
  } : Application
)

val ExecutePaste = CreateCommand[Application](
  (target : Application) => {
    //Executa comando de colar
  }
)
  
val resultingApplication = ExecuteMany(application, List(ExecuteOpen, ExecutePaste))
    
\end{lstlisting}

\begin{comment}
\subsection*{Vantagens e Desvantagens}

O Command funcional consegue economizar classes, resumindo a 
complexidade do padrão à criação de uma tupla para os casos 
onde é desejado implementar a operação de desfazer. Para 
os casos onde isso não é necessário, a implementação se torna 
ainda mais simples, já que as funções de alta ordem em si 
já suprem a necessidade de encapsular operações. A grande 
desvantagem da implementação funcional é que a execução das 
operações se torna mais dependente do valor alvo. O valor 
atualizado sempre precisa ser repassado para o comando para 
que qualquer modificação feita anteriormente - seja por outro 
trecho da aplicação ou por um comando executado anteriormente - 
seja considerada durante a execução.
\end{comment}
%\subsection{Interpreter}

\begin{figure}[htb]
	\caption{\label{fig_grafico}Estrutura do Interpreter}
	\begin{center}
	    \includegraphics[scale=0.5]{5_padroes-contexto-funcional/5.3_comportamentais/5.3.03_interpreter/diagram.png}
	\end{center}
\end{figure}

Exemplo Orientado a Objetos:

\begin{lstlisting}[caption={Interpreter Orientação a Objetos},label=oointerpreter]


    
\end{lstlisting}

Contexto Funcional:


\begin{lstlisting}[caption={Interpreter Funcional},label=fpinterpreter]
    

    
\end{lstlisting}
%\subsection{Iterator}

\begin{figure}[htb]
	\caption{\label{fig_grafico}Estrutura do Iterator}
	\begin{center}
	    \includegraphics[scale=0.5]{5_padroes-contexto-funcional/5.3_comportamentais/5.3.04_iterator/diagram.png}
	\end{center}
\end{figure}

Exemplo Orientado a Objetos:

\begin{lstlisting}[caption={Iterator Orientação a Objetos},label=ooiterator]


    
\end{lstlisting}

Contexto Funcional:


\begin{lstlisting}[caption={Iterator Funcional},label=fpiterator]
    

    
\end{lstlisting}
%\section{Mediator}


Nesse padrão, um objeto chamado de \textit{Mediator} age como intermediário 
entre um grupo de objetos, ficando responsável por qualquer 
interação entre eles. O \textit{Mediator} conhece todos 
esses objetos enquanto cada objeto conhece apenas o 
\textit{Mediator}, o que os torna 
mais independentes, simplificando sua reutilização
 e concentrando as dependências entre eles 
em um só lugar. \cite{gamma:1995}

A estrutura do padrão é apresentada na Figura \ref{mediator_struct}. 
Uma interface \texttt{Mediator} define as operações que um tipo de 
objeto \textit{Mediator} deve possuir. \texttt{ConcreteMediator} representa 
uma classe que implementa essas operações. Um \texttt{Colleague} 
é um objeto conhecido pelo \texttt{Mediator} e cada \texttt{ConcreteColleague} 
pode ser tanto um objeto que possui operações refletidas 
em outros objetos quanto ser um dos objetos afetados 
indiretamente por outro \texttt{Colleague}.

\begin{figure}[htb]
	\caption{\label{mediator_struct}Estrutura do \textit{Mediator}.}
	\begin{center}
	    \includegraphics[scale=0.5]{5_padroes-contexto-funcional/5.3_comportamentais/5.3.05_mediator/mediator_estrutura.png}
	\end{center}
  \caption*{Fonte: O Autor (2021)}
\end{figure}

\subsection*{Exemplo Orientado a Objetos}

Como exemplo, é considerada uma janela de uma aplicação 
que apresenta diversos \textit{widgets}, entre eles uma caixa 
de entrada de texto e uma lista de seleção. Quando um item é 
selecionado na lista, o texto contido nele deve aparecer 
na caixa de entrada de texto. O \textit{Mediator} é responsável 
por alterar a caixa de entrada de texto quando um item 
é selecionado na lista, enquanto a lista é responsável 
por informar ao \textit{Mediator} quando um item for selecionado. 
A Figura \ref{mediator_exemplo} apresenta o diagrama 
de classes para esse exemplo. O Código \ref{oomediator} 
apresenta a implementação do padrão para esse exemplo.

\begin{lstlisting}[caption={\textit{Mediator} Orientado a Objetos.},label=oomediator]

abstract class DialogDirector {

  def ShowDialog() : Unit = {
    //Exibe o dialog
  }

  def CreateWidget()
  def WidgetChanged(widget: Widget)
}

class FontDialogDirector() extends DialogDirector {

  private var list : ListBox = null
  private var field : EntryField = null

  override def CreateWidget(): Unit = {
    this.list = new ListBox(this)
    this.field = new EntryField(this)
  }

  override def WidgetChanged(widget: Widget): Unit = {
    this.field.text = this.list.selection
  }
}

abstract class Widget(val director : DialogDirector) {
  def Changed() : Unit = director.WidgetChanged(this)
}

class EntryField(director : DialogDirector) extends Widget(director) {
  var text : String = ""
}

class ListBox(director : DialogDirector) extends Widget(director){
  private var _selection : String = ""
  def selection : String = _selection

  def SetSelection(selection : String) : Unit = {
    this._selection = selection
    Changed()
  }
}
    
\end{lstlisting}
\legend{Fonte: O Autor (2021)}

\begin{figure}[htb]
	\caption{\label{mediator_exemplo}Exemplo de \textit{Mediator}.}
	\begin{center}
	    \includegraphics[scale=0.5]{5_padroes-contexto-funcional/5.3_comportamentais/5.3.05_mediator/mediator_exemplo.png}
	\end{center}
  \caption*{Fonte: O Autor (2021)}
\end{figure}

\subsection*{Contexto Funcional}

O Código \ref{fpmediator} demonstra a implementação 
funcional do \textit{Mediator}. Uma função é responsável por 
gerenciar as interdependências entre os valores 
dos tipos \texttt{EntryField}, definido na linha 2, e \texttt{ListBox}, 
definido na linha 12. Quando a função mediadora \texttt{ChangeSelection}, 
definida na linha 22, é chamada, ela precisa receber 
como parâmetro o elemento alvo e o elemento dependente, 
além das informações necessárias para executar a 
operação \texttt{ChangeListBoxSelection}. A função retorna 
tanto o \textit{colleague} alvo da operação 
quanto os \textit{colleagues} afetados, mantendo 
a função cliente que chama essa operação atualizada 
quanto ao estado de ambos os valores. 

A vantagem dessa abordagem é que ela torna possível 
que as funções dos \textit{colleagues} (no Código 
\ref{fpmediator}, \texttt{ChangeEntryFieldText} e 
\texttt{ChangeListBoxSelection}) sejam idependentes dos 
mediadodores, favorecendo seu reuso. A desvantagem 
é que é necessário realizar, na função cliente, 
um gerenciamento quanto ao estado de todos os 
\textit{colleagues}, já que a função mediadora 
deve ser pura e não realiza efeitos colaterais.

\begin{lstlisting}[caption={\textit{Mediator} Funcional.},label=fpmediator]
    
type EntryField = String

def ChangeEntryFieldText(text : String,
						 entryField: EntryField)
: EntryField =
  text
  
def GetText(entryField : EntryField) : String =
  entryField
  
type ListBox = String
  
def ChangeListBoxSelection(selection : String,
						   listBox: ListBox)
: ListBox =
  selection
  
def GetSelection(listBox : ListBox)  : String=
  listBox
  
def ChangeSelection(selection : String,
					entryField: EntryField,
					listBox : ListBox) : (EntryField, ListBox) = {
  (ChangeListBoxSelection(selection, listBox), ChangeEntryFieldText(selection, entryField))
}
	    
\end{lstlisting}
\legend{Fonte: O Autor (2021)}
%\section{Memento}

O Memento permite armazenar e restaurar o estado interno de um 
objeto sem expor esse estado. Dessa forma, o encapsulamento do 
objeto em questão não é violado, mesmo seu estado sendo armazenado 
externamente.

Isso é alcançado através de uma classe Memento que armazena os 
atributos da classe que precisa ser salva (Originator). A geração 
de um objeto Memento só é possível através do próprio Originator, 
assim como a recuperação de seus atributos. Uma classe Caretaker 
é usada para armazenar objetos do tipo Memento e repassá-los para 
um Originator que precisa acessar o estado do Memento.

\begin{figure}[htb]
	\caption{\label{fig_grafico}Estrutura do Memento}
	\begin{center}
	    \includegraphics[scale=0.5]{5_padroes-contexto-funcional/5.3_comportamentais/5.3.06_memento/diagram.png}
	\end{center}
\end{figure}

Exemplo Orientado a Objetos:

\begin{lstlisting}[caption={Memento Orientação a Objetos},label=oomemento]



\end{lstlisting}

Contexto Funcional:

A ideia de armazenar estados anteriores pode ser alcançada com uma 
estrutura que armazena o valor atual do Originator e uma cópia do 
elemento desejado (ou uma lista armazenando um histórico de cópias). 
A partir dessa estrutura, é simples implementar funções que criam 
a cópia, atualizam o valor do Originator externamente e atualizam 
o valor do Originator a partir dos Mementos.

\begin{lstlisting}[caption={Memento Funcional},label=fpmemento]



\end{lstlisting}
%\section{Observer}

\begin{figure}[htb]
	\caption{\label{observer_struct}Estrutura do Observer}
	\begin{center}
	    \includegraphics[scale=0.5]{5_padroes-contexto-funcional/5.3_comportamentais/5.3.07_observer/diagram.png}
	\end{center}
\end{figure}

\subsection*{Exemplo Orientado a Objetos}

\begin{lstlisting}[caption={Observer Orientação a Objetos},label=ooobserver]


    
\end{lstlisting}

\subsection*{Contexto Funcional}


\begin{lstlisting}[caption={Observer Funcional},label=fpobserver]
    

    
\end{lstlisting}
%\section{State}

O State permite alterar o comportamento de um objeto baseado 
em seu estado interno. Uma interface define os comportamentos 
que dependem do estado do objeto e classes que a implementam 
definem a implementação dos mesmos. Dessa forma, o objeto 
principal delega as operações às classes que representam 
seu estado.

Esse padrão contribui para o reuso de operações comuns quando 
diversas classes relacionadas teriam que ser reinstanciadas 
durante uma mudança de estado. Também é permitido que o 
estado mude dinamicamente durante a execução.

\begin{figure}[htb]
	\caption{\label{state_struct}Estrutura do State}
	\begin{center}
	    \includegraphics[scale=0.5]{5_padroes-contexto-funcional/5.3_comportamentais/5.3.08_state/diagram.png}
	\end{center}
\end{figure}

\subsection*{Exemplo Orientado a Objetos}

\begin{figure}[htb]
	\caption{\label{state_exemplo}Exemplo de State}
	\begin{center}
	    \includegraphics[scale=0.5]{5_padroes-contexto-funcional/5.3_comportamentais/5.3.08_state/state_exemplo.png}
	\end{center}
\end{figure}

\begin{lstlisting}[caption={State Orientação a Objetos},label=oostate]

trait State{
    def pressButton() : State
}

class OnState() extends State {
    def pressButton() : State = new OffState()
}

class OffState() extends State {
    def pressButton() : State = new OnState()
}

class Lamp(state : State) {
    def pressButton() : Unit {
        this.state = state.pressButton()
    }
}
    
\end{lstlisting}

\subsection*{Contexto Funcional}

Normalmente, a primeira alternativa que se tem em mente é 
o monad State. Porém, esse monad é focado em comportamentos 
que alteram o estado atual do nosso valor. Por mais que isso 
seja possível através do padrão State, por definição, sua 
intenção é fornecer comportamentos que não necessariamente 
altera o estado interno do valor.

Dessa forma, uma maneira interessante de definir o State 
no contexto funcional é utilizando uma case class que armazena, 
além dos valores comuns, um valor referente a um State. 
Esse State nada mais é do que outra clase class que 
irá armazenar, através de funções, os comportamentos que 
dependem de um estado. Da mesma forma que uma interface 
define as assinaturas das operações no exemplo orientado a 
objetos, aqui a definição da case class definirá que tipos 
de comportamentos a case class principal deverá possuir.

\begin{lstlisting}[caption={State Funcional},label=fpstate]
    
case class LampState(pressButton : () => Lamp)

case class Lamp(state : LampState)

def pressButton(lamp : Lamp) : Lamp =
    lamp.state.pressButton()

val onState : LampState = LampState(() => offState)

val offState : LampState = LampState(() => onState)
    
\end{lstlisting}

É importante notar que, aqui, quando o estado do valor 
principal precisa ser supostamente modificado, o que na 
verdade acontecerá é que a função da case class State 
irá retornar o nosso valor atualizado.
%\section{Strategy}


O padrão Strategy define grupos de algoritmos encapsulados e
intercambiáveis para um determinado contexto. Esses 
algoritmos podem ser definidos ou trocados em tempo de 
execução, permitindo que os clientes que os utilizem possam
alternar entre as implementações definidas livremente.

O Strategy soluciona o problema de classes relacionadas 
diferirem apenas em algum comportamento, permitindo que 
esse comportamento possa ser isolado e o resto da implementação 
das classes reaproveitado. Ele também evita a utilização de 
muitas operações condicionais. Ao invés de verificar qual 
deve ser o comportamento toda vez que ele precisar ser 
executado, o comportamento é pré-definido pelo contexto. 

A estrutura do padrão pode ser vista na figura \ref{strategy_struct}, 
onde uma interface é responsável por definir que operações 
uma estratégia deve possuir, enquanto diversas classes 
concretas implementam essas operações definindo suas 
estratégias. A classe cliente é responsável por manter 
uma referência para a estratégia e chamar as 
operações desejadas.

\begin{figure}[htb]
	\caption{\label{strategy_struct}Estrutura do Strategy}
	\begin{center}
	    \includegraphics[scale=0.5]{5_padroes-contexto-funcional/5.3_comportamentais/5.3.09_strategy/strategy_estrutura.png}
	\end{center}
\end{figure}

\subsection*{Exemplo Orientado a Objetos}

O exemplo do Strategy apresenta uma \textit{stream} 
de texto que pode ser quebrada em linhas utilizando 
estratégias diferentes. Nele, a classe Composition 
gerencia as quebras de linha do texto exibidas em 
um editor, enquanto a interface Compositor define 
uma estratégia de quebra de linha. As estratégias 
apresentadas são representadas pelas classes 
SimpleCompositor, TeXCompositor e ArrayCompositor. 
A implementação desse exemplo pode ser vista no código 
\ref{oostrategy}, enquanto o diagrama de classes pode 
ser visto na figura \ref{strategy_exemplo}.

\begin{figure}[htb]
	\caption{\label{strategy_exemplo}Exemplo de Strategy}
	\begin{center}
	    \includegraphics[scale=0.5]{5_padroes-contexto-funcional/5.3_comportamentais/5.3.09_strategy/strategy_exemplo.png}
	\end{center}
\end{figure}

\begin{lstlisting}[caption={Strategy Orientação a Objetos},label=oostrategy]
    
trait Compositor {
  def Compose(
             natural : Vector[Coord],
             stretch : Vector[Coord],
             shrink : Vector[Coord],
             componentCount : Int,
             lineWidth : Int,
             breaks : Vector[Coord]
             ) : Int
}

class SimpleCompositor extends Compositor {
  override def Compose(natural: Vector[Coord],
                       stretch: Vector[Coord],
                       shrink: Vector[Coord],
                       componentCount: Int,
                       lineWidth: Int,
                       breaks: Vector[Coord])
  : Int = {
    //Implementação do Simple Compositor
    0
  }
}

class TeXCompositor extends Compositor {
  override def Compose(natural: Vector[Coord],
                       stretch: Vector[Coord],
                       shrink: Vector[Coord],
                       componentCount: Int,
                       lineWidth: Int,
                       breaks: Vector[Coord])
  : Int = {
    //Implementação do TeX Compositor
    0
  }
}

class ArrayCompositor extends Compositor {
  override def Compose(natural: Vector[Coord],
                       stretch: Vector[Coord],
                       shrink: Vector[Coord],
                       componentCount: Int,
                       lineWidth: Int,
                       breaks: Vector[Coord])
  : Int = {
    //Implementação do Array Compositor
    0
  }
}

class Composition(var compositor: Compositor) {
  private var lineWidth : Int = 0
  private var lineBreaks : Vector[Coord] = Vector.empty
  private var lineCount : Int = 0

  def Repair() : Unit = {
    var natural = new Vector[Coord]
    var stretchability = new Vector[Coord]
    var shrinkability = new Vector[Coord]
    // Implementação da função repair
    val breakCount = compositor.Compose(
      natural, stretchability, shrinkability,
      lineCount, lineWidth, lineBreaks)
  }
}

\end{lstlisting}

\subsection*{Contexto Funcional}

Como a intenção do padrão é permitir a criação de 
grupos de algoritmos, ele é um candidato para o uso 
de funções de alta ordem. Ao invés de definir 
uma classe nova para cada implementação, a 
função Repair pode receber como parâmetro uma 
função compose, cuja assinatura predefinida 
recebe um parâmetro do tipo Compositor e retorna 
um inteiro. Dessa forma, basta definir as 
funções ComposeSimple, ComposeTeX e ComposeArray 
que podem ser passadas por parâmetro quando a 
função Repair for chamada. A implementação pode 
ser vista no código \ref{fpstrategy}.

\begin{lstlisting}[caption={Strategy Funcional},label=fpstrategy]
    
type Composition = (Int, List[Coord], Int)

type Compositor = (
    List[Coord],
      List[Coord],
      List[Coord],
      Int, Int,
      List[Coord])

def Repair(composition: Composition,
        compose : (Compositor) => Int): Unit = {
  var natural = List.empty
  var stretchability = List.empty
  var shrinkability = List.empty
  //Implementação da função Repair
  val breakCount = compose(
      (natural, stretchability, shrinkability,
     composition._3, composition._1, composition._2)
  )
}

def ComposeSimple(compositor : Compositor) : Int = {
  // implementação para SimpleCompositor
}

def ComposeTeX(compositor : Compositor) : Int = {
  // implementação para TeXCompositor
}

def ComposeArray(compositor : Compositor) : Int = {
  // implementação para ArrayCompositor
}
    
\end{lstlisting}

\begin{comment}
\subsection*{Vantagens e Desvantagens}

Utilizar funções de alta ordem no lugar do 
Strategy permite uma versatilidade maior quanto 
a que tipo de função pode ser aceita. Isso traz 
a vantagem de favorecer o reúso de funções que não 
foram planejadas para ser usadas como estratégias 
diferentes de implementação. Porém, traz a desvantagem 
de permitir que funções que possuam a mesma assinatura 
porém não realizem a operação desejada sejam passadas 
como estratégias válidas.
\end{comment}
%\section{Template Method}

A ideia do \textit{Template Method} é fornecer um esqueleto para um algoritmo 
e deixar para outras classes a tarefa de implementar as funções que 
compõem esse algoritmo. Uma classe abstrata define a operação 
\textit{Template Method}, onde são executas as etapas do 
algoritmo, definidas em função 
das operações abstratas ainda não implementadas.\cite{gamma:1995}

Esse padrão ajuda a evitar repetição de código, concentrando 
em apenas uma classe a estrutura de uma operação. Além disso, 
não só apenas a estrutura do algoritmo como qualquer etapa em 
comum para todas as subclasses pode ser concentrada na superclasse, 
evitando mais ainda a repetição do código. A estrutura do padrão 
pode ser vista na Figura \ref{tpmethod_struct}.

\begin{figure}[htb]
	\caption{\label{tpmethod_struct}Estrutura do \textit{Template Method}.}
	\begin{center}
	    \includegraphics[scale=0.5]{5_padroes-contexto-funcional/5.3_comportamentais/5.3.10_template-method/templatemethod_estrutura.png}
	\end{center}
  \caption*{Fonte: O Autor (2021)}
\end{figure}

\subsection*{Exemplo Orientado a Objetos}

Como exemplo pode ser consideraro um \textit{framework} que 
fornece uma classe para abrir documentos. Essa classe 
possui operações abstratas para cada etapada da abertura 
de um documento, de forma que as subclasses que a estendem 
podem definir formas diferentes de abrir documentos de tipos 
diferentes. A operação \texttt{AddDocument} é o \textit{template method}, 
responsável por chamar as demais operações implementadas 
pelas subclasses. 
O diagrama de classes do exemplo pode ser visto na 
Figura \ref{tpmethod_exemplo}, 
enquanto a implementação pode ser vista no Código \ref{ootpmethod}.

\begin{figure}[htb]
	\caption{\label{tpmethod_exemplo}Exemplo de \textit{Template Method}.}
	\begin{center}
	    \includegraphics[scale=0.5]{5_padroes-contexto-funcional/5.3_comportamentais/5.3.10_template-method/templatemethod_exemplo.png}
	\end{center}
  \caption*{Fonte: O Autor (2021)}
\end{figure}

\begin{lstlisting}[caption={\textit{Template Method} Orientado a Objetos.},label=ootpmethod]

abstract class Document {
  def Save() : Unit = {
    //Salva um document
  }
  def Open() : Unit = {
    //Abre um documento
  }
  def Close() : Unit = {
    //Fecha um documento
  }
  def DoRead()
}

class MyDocument extends Document {
  def DoRead(): Unit = {
    //Faz leitura do documento
  }
}

abstract class Application {

  var documents : List[Document] = List.empty

  def AddDocument(document : Document) : Unit = {
    documents = document :: documents
  }

  def OpenDocument(fileName : String) : Unit = {
    if(!CanOpenDocument(fileName)){

    } else {
      val doc = DoCreateDocument()
      AddDocument(doc)
      AboutToOpenDocument(doc)
      doc.Open()
      doc.DoRead()
    }
  }

  def DoCreateDocument() : Document
  def CanOpenDocument(fileName : String) : Boolean
  def AboutToOpenDocument(document : Document)
}

class MyApplication extends Application {
  def DoCreateDocument() : Document = new MyDocument
  def CanOpenDocument(fileName : String) : Boolean = {
    //Verifica se documento pode ser aberto
    true
  }
  def AboutToOpenDocument(document : Document): Unit = {
    //Operação ao abrir documento
  }
}

\end{lstlisting}
\legend{Fonte: O Autor (2021)}

\subsection*{Contexto Funcional}

No contexto funcional, a mesma ideia pode ser alcançada 
através de funções de alta ordem e composição de funções. 
O método \texttt{OpenDocument} (o \textit{template method}) é 
uma função de alta ordem que recebe como parâmetro todas 
as funções necessárias para executar o algoritmo. 
O exemplo pode ser visto no Código \ref{fptpmethod}.

\begin{lstlisting}[caption={\textit{Template Method} Funcional.},label=fptpmethod]
    
object Application {

  def AddDocument(document : Document,
                  documents : List[Document]) : List[Document] =
    document :: documents


  def OpenDocument(filename : String,
                   documents : List[Document],
                   DoCreateDocument : () => Document,
                   CanOpenDocument : (String) => Boolean,
                   AboutToOpenDocument : (Document) => Unit) : Unit = {
    if(!CanOpenDocument(filename)){
      //...
    } else {
      var _documents : List[Document] = Nil
      val doc = DoCreateDocument()
      _documents = AddDocument(doc, documents)
      AboutToOpenDocument(doc)
      OpenDocument(doc)
      ReadDocument(doc)
    }
  }

}

\end{lstlisting}
\legend{Fonte: O Autor (2021)}

Para definir uma implementação do algoritmo, basta 
definir uma nova função que é a combinação do método 
template com as funções que representam as etapas do 
algoritmo. O Código \ref{algotpmethod} define uma 
nova função, \texttt{MyApplicationOpenDocument}, que recebe 
como parâmetro o nome do arquivo e uma lista de 
documentos. Ela retorna a função \texttt{OpenDocument} com as 
funções específicas para o tipo de documento 
desejado sendo passadas como parâmetro, nas linhas 
6, 7 e 8. Dessa forma, a função \texttt{MyApplicationOpenDocument} 
pode ser reutilizada da mesma forma que a classe 
\texttt{MyApplication} seria reutilizada no exemplo orientado 
a objetos.

\begin{lstlisting}[caption={Definição do algoritmo.},label=algotpmethod]
    
val MyApplicationOpenDocument = 
    (filename : String, documents : List[Document]) => 
      OpenDocument(filename, documents,
        CreateMyDocument,
        MyCanOpenDocument,
        MyAboutToOpenDocument)

\end{lstlisting}
\legend{Fonte: O Autor (2021)}

\begin{comment}
\subsection*{Vantagens e Desvantagens}

É possível definir novos \textit{templates} com operações 
pré-definidas facilmente criando uma combinação parcial 
das funções necessárias para a execução do algoritmo. 
Assim, poderia ser implementada uma função cujas 
operações DoCreateDocument e CanOpenDocument sejam idênticas, 
mas que permitisse implementações diferentes de 
AboutToOpenDocument. Para que isso pudesse ser feito 
no exemplo orientado a objetos, seria necessário 
definir uma nova subclasse abstrata de Application.
\end{comment}

%\section{Visitor}

O padrão Visitor define uma estrutura que permite realizar 
operações em uma estrutura de objetos sem precisar alterá-los 
e sem precisar que os objetos dessa estrutura conheçam as 
operações que estão sendo realizadas. 

Uma estrutura alvo de um Visitor pode possuir objetos de 
classes diferentes. Por isso, o padrão deve implementar 
uma operação \textit{visit} para cada uma dessas classes, 
através de uma sobrecarga de métodos. Para que essa 
abordagem funcione, as classes da estrutura devem implementar 
uma operação \textit{accept} que recebe como parâmetro um 
Visitor genérico e chama sua operação \textit{visit}, 
passando uma referência para a instância atual (this) 
como parâmetro. Dessa forma, a função chamada é a 
que recebe a classe em questão como parâmetro.

Esse padrão permite estender objetos para novas operações 
sem comprometer sua implementação ou poluir as classes com 
diversas operações que não são de sua responsabilidade. 
Sua estrutura pode ser vista na imagem \ref{visitor_struct}.

\begin{figure}[htb]
	\caption{\label{visitor_struct}Estrutura do Visitor}
	\begin{center}
	    \includegraphics[scale=0.5]{5_padroes-contexto-funcional/5.3_comportamentais/5.3.11_visitor/visitor_estrutura.png}
	\end{center}
\end{figure}

\subsection*{Exemplo Orientado a Objetos}

Um compilador precisa fazer a análise de árvore abstrata sintática de 
um programa. Essa análise inclui diversas operações diferentes, como 
checagem de tipos e geração de código. Para que os nós da árvore 
não precisem implementar essas operações, elas implementam uma operação 
genérica que recebe como parâmetro qualquer Visitor. Dessa forma, para 
cada operação desejada, basta implementar uma nova classe Visitor 
que percorrerá os elementos da árvore abstrata sintática. O diagrama 
de classes para o exemplo pode ser visto na imagem \ref{visitor_exemplo1}, 
enquanto a implementação pode ser vista no código \ref{oovisitor}.

\begin{figure}[htb]
	\caption{\label{visitor_exemplo1}Exemplo de Visitor}
	\begin{center}
	    \includegraphics[scale=0.5]{5_padroes-contexto-funcional/5.3_comportamentais/5.3.11_visitor/visitor_exemplo.png}
	\end{center}
\end{figure}

\begin{lstlisting}[caption={Visitor Orientação a Objetos},label=oovisitor]

trait NodeVisitor {
  def VisitAssignment(node : AssignmentNode)
  def VisitVariableRef(node : VariableRefNode)
}

trait Node {
  def Accept(visitor : NodeVisitor)
}

class AssignmentNode extends Node {
  def Accept(visitor: NodeVisitor): Unit = visitor.VisitAssignment(this)
}

class VariableRefNode extends Node {
  def Accept(visitor : NodeVisitor) : Unit = visitor.VisitVariableRef(this)
}

class TypeCheckingVisitor extends NodeVisitor {

  def VisitAssignment(node: AssignmentNode): Unit = {
    //Operações de checagem de tipo para atribuição
  }

  def VisitVariableRef(node: VariableRefNode): Unit = {
    //Operações de checagem de tipo para variáveis
  }
}

class CodeGeneratingVisitor extends NodeVisitor {

  def VisitAssignment(node: AssignmentNode): Unit = {
    //Operações de geração de código para atribuição
  }

  def VisitVariableRef(node: VariableRefNode): Unit = {
    //Operações de geração de código para variáveis
  }
}

\end{lstlisting}

\subsection*{Contexto Funcional}

\begin{comment}
att
O Visitor é mais um caso em que funções de alta ordem ajudam a 
economizar novas classes e interfaces. Basta definir uma função 
que receba como parâmetro um valor do tipo encapsulado pela 
coleção e retornar um valor do mesmo tipo com a operação 
realizada. Funções do tipo map, que podem ser usadas para 
realizar uma operação em uma coleção, contribuem para essa 
implementação.

Porém, a funcionalidade do Visitor que permite realizar 
operações diferentes dependendo da implementação do objeto 
também é interessante e pode ser alcançada utilizando 
pattern matching. [terminar esse texto]
\end{comment}

\begin{lstlisting}[caption={Visitor Funcional},label=fpvisitor]
    

    
\end{lstlisting}

% ----------------------------------------------------------
% PARTE
% ----------------------------------------------------------
%\part{Resultados}
% ----------------------------------------------------------

%% ---
% Apresentação dos Resultados
% ---
\chapter{Resultados}
% ---

Nos capítulos anteriores, foram analisados todos 
vinte e três padrões de projeto \textit{Gang of 
Four}, com o objetivo de extrair o problema 
resolvido pelo padrão e analisar, a partir dos 
conceitos de programação funcional apresentados, 
como uma solução para o mesmo problema poderia ser 
alcançada. Foi possível separar as análises realizadas em 
quatro grandes grupos, com o primeiro sendo 
dividido em três subgrupos: 

\begin{alineas}
    \item Padrões resolvidos por funções de alta ordem
    \begin{alineas}
        \item Funções de alta ordem como alternativa a classes abstratas ou interfaces
        \item Valores que armazenam funções
        \item Funções que armazenam valores em closures
    \end{alineas}
    \item Padrões com soluções alternativas
    \item Padrões sem diferenças relevantes
    \item Padrões que não fazem sentido no contexto funcional
\end{alineas}

Esses grupos estão organizados na tabela 
\ref{resultados}, onde são apresentados os 
padrões pertencentes a cada grupo com a 
quantidade de padrões pertencentes ao grupo. 
Cada grupo será explicado com mais detalhes 
no decorrer do capítulo.

\begin{quadro}[htb]
    \caption{\label{resultados}Agrupamento de análises dos padrões}
    \begin{tabular}{@{}llll@{}}
        \toprule
                                  &                            & Padrões                 & Quantidade         \\ \midrule
        \multirow{14}{*}{Grupo A} & \multirow{7}{*}{A.1} & Factory Method          & \multirow{7}{*}{7} \\
                                  &                            & Builder                 &                    \\
                                  &                            & Adapter                 &                    \\
                                  &                            & Bridge                  &                    \\
                                  &                            & Proxy                   &                    \\
                                  &                            & Strategy                &                    \\
                                  &                            & Template Method         &                    \\ \cmidrule(r){2-4}
                                  & \multirow{3}{*}{A.2} & Abstract Factory        & \multirow{3}{*}{3} \\
                                  &                            & Command                 &                    \\
                                  &                            & State                   &                    \\ \cmidrule(r){2-4}
                                  & \multirow{4}{*}{A.3} & Composite               & \multirow{4}{*}{4} \\
                                  &                            & Decorator               &                    \\
                                  &                            & Chain of Responsibility &                    \\
                                  &                            & Interpreter             &                    \\ \cmidrule(r){1-4}
        \multicolumn{2}{l}{\multirow{3}{*}{Grupo B}}           & Iterator                & \multirow{3}{*}{3} \\
        \multicolumn{2}{l}{}                                   & Observer                &                    \\
        \multicolumn{2}{l}{}                                   & Visitor                 &                    \\ \cmidrule(r){1-4}
        \multicolumn{2}{l}{\multirow{3}{*}{Grupo C}}           & Façade                  & \multirow{3}{*}{3} \\
        \multicolumn{2}{l}{}                                   & Flyweight               &                    \\
        \multicolumn{2}{l}{}                                   & Mediator                &                    \\ \cmidrule(r){1-4}
        \multicolumn{2}{l}{\multirow{3}{*}{Grupo D}}           & Prototype               & \multirow{3}{*}{3} \\ 
        \multicolumn{2}{l}{}                                   & Singleton               &                    \\
        \multicolumn{2}{l}{}                                   & Memento                 &                    \\ \cmidrule(r){1-4} %\cmidrule(l){4-4} 
    \end{tabular}
\end{quadro}

\section{Padrões resolvidos por funções de alta ordem}

Entre as soluções vistas, a que é aplicada na 
maior parte dos padrões é o uso de funções de alta 
ordem como alternativa a interfaces ou classes. 
Um total de catorze dos vinte e três padrões 
se encaixa nessa categoria. Por isso, ainda foi 
possível dividir esses padrões em três  
subcategorias quanto a como as funções de alta 
ordem foram aplicadas. Alguns padrões acabam se 
encaixando em mais de uma das categorias, mas para 
evitar repetições e para simplificar o agrupamento, 
cada padrão será explicado no grupo mais próximo 
da solução proposta. 

% Factory Method, Builder, Adapter, Bridge, 
% Proxy, Strategy, Template Method
\subsection{Funções de alta ordem como alternativa a 
            classes abstratas ou interfaces}

Como foi visto no mapeamento de conceitos orientados 
a objeto para conceitos de programação funcional, 
funções de alta ordem podem servir como alternativas 
para o uso de interfaces. Alguns dos padrões analisados 
baseiam-se no uso de interfaces para definir a assinatura 
de funções que a classe cliente deve receber. De forma 
equivalente, é possível que uma função cliente receba, 
por parâmetro, uma função com a assinatura equivalente 
à da interface. Essa é a abordagem utilizada com os 
padrões Builder, Adapter, Bridge, Proxy e Strategy. 

De forma semelhante, as funções de alta ordem são 
alternativas ao uso de 
classes abstratas, como é o caso dos padrõess 
Factory Method e Template Method. Ambos baseiam-se 
em definir uma operação abstrata que deve ser 
implementada por uma subclasse. Uma alternativa 
a essa implementação é fazer com que as funções 
que dependam dessa operação abstrata a recebam 
por parâmetro.

% Abstract Factory, Command, State
\subsection{Valores que armazenam funções}

Os padrões Abstract Factory, Command e 
State também baseiam-se em passar funções de 
alta ordem como parâmetro para funções clientes. 
Porém, para esses três padrões, as funções 
passadas são relacionadas, ou seja, 
todas as funções de um Strategy pertencem a uma mesma 
estratégia, todas as funções do Abstract Factory criam 
o mesmo tipo de valor e a função de desfazer do 
Command está relacionada à função principal executada. 
Agrupar essas funções em um mesmo valor pode 
contribuir para retirar das funções clientes a 
responsabilidade de garantir que elas possuam 
essa equivalência.

% Composite, Decorator, Chain of Responsibility, Interpreter
\subsection{Funções que armazenam valores em closures}

Algumas implementações basearam-se em definir funções 
que retornam novas funções. Isso é necessário para 
que seja possível configurar as funções retornadas 
com valores recebidos por parâmetro pelas funções 
que as criam. Esse é o comportamento definido pelas 
closures, onde uma determinada função armazena 
um valor do escopo da função que a retornou. 

O padrão Composite define funções para gerar os 
elementos nós e os elementos folha. Dessa forma, 
a partir de uma única função, vários elementos 
folha e nó configurados com valores diferentes 
podem ser gerados.

De forma semelhante, o padrão Decorator define 
funções que podem receber como parâmetro valores 
variados enquanto retorna uma nova função com 
a mesma assinatura da função decorada. 

O padrão Chain of Responsibility aproveita a mesma 
ideia para gerar as funções da cadeia. O tópico e 
a próxima função da cadeia são passadas por parâmetro 
previamente, armazenadas na closure e reaproveitadas 
na função retornada. 

Por fim, o padrão Interpreter apresenta um comportamento 
semelhante ao Composite, definindo funções que retornam 
os elementos terminais e não terminais, permitindo 
inclusive definir funções mais genéricas que recebem 
a função que aplica a regra da gramática por parâmetro.

% Iterator, Observer, Visitor
\section{Padrões com soluções alternativas}

Os padrões Iterator, Observer e Visitor possuem 
implementações alternativas existentes que não 
necessariamente 
seguem à risca a ideia do padrão. No caso do Iterator, 
existe a diferença entre o gerencimaneto por parte 
do cliente e por parte da coleção entre as 
alternativas orientada a objetos e funcional. No 
caso do Observer, o conceito no qual ele se 
baseia - programação reativa - é levado em 
consideração ao substituí-lo pela programação 
reativa funcional. Já o Visitor, apesar de também 
aproveitar-se de funções de alta ordem, na verdade 
é resolvido pelo recurso \textit{pattern matching}, 
que não é um conceito exclusivamente funcional, 
porém costuma ser implementado em linguagens 
funcionais. Nesse caso, o padrão não foi 
resolvido por programação funcional em si, 
mas sim por uma alternativa próxima. 

% Façade, Flyweight, Mediator
\section{Padrões sem diferenças relevantes}

Alguns dos padrões analisados possuem 
uma implementação muito semelhante ou 
equivalente entre os contextos orientados 
a objeto e funcionais, como se a solução 
proposta pelo padrão estivesse apenas 
sendo reutilizada por si só, sem 
recursos adicionais oriundos da programação 
funcional que contribuem para a 
resolução do problema. 

Da mesma forma que o padrão Façade orientado 
a objetos baseia-se no acesso entre as classes, 
a implementação funcional baseia-se no acesso 
entre módulos. Como ambas as ideias são 
análogas, a implementação do padrão na verdade 
não possui mudanças significativas. 
O padrão Flyweight, tanto no 
contexto orientado a objetos quanto no funcional, 
é implementado através de memoização. Já o Mediator 
baseia-se em possuir uma função (ou classe) que 
gerencia as dependências entre valores (ou objetos). 
No caso desses dois padrões, ambos possuem 
pequenas diferenças como consequência de suas 
implementações - por exemplo, não existem os dois 
tipos de Flyweight intrínseco ou extrínseco 
graças à imutabilidade e há a necessidade do 
cliente gerenciar a mudança de estado dos 
\textit{colleagues} no Mediator -, mas a ideia 
por trás da implementação de ambos é análoga 
à versão orientada a objetos.

% Prototype (imutabilidade), Singleton (funções puras), 
% Memento (imutabilidade)
\section{Padrões que não fazem sentido no contexto funcional}

Para alguns padrões, o problema proposto deixa 
de existir graças aos conceitos já implementados 
em linguagens funcionais. No caso dos padrões 
analisados neste trabalho, encaixam-se nessa 
categoria o Singleton, o Prototype e o Memento. 

Como a intenção do padrão Singleton pode 
ser entendida como a definição de uma variável 
global, sua implementação no contexto funcional 
viola o conceito de funções puras e imutabilidade - 
para o caso em que o Singleton armazena algum 
estado compartilhado. Portanto, tentar 
implementá-lo não faz sentido. 

Para o caso do Prototype, como visto anteriormente, 
o uso de estruturas de dados imutáveis trazem um 
gerenciamento mais simples de memória. Não existe 
preocupação quanto a uma referência compartilhada 
para um mesmo valor, já que seu estado não pode ser 
modificado. Dessa forma, não há a necessidade de 
definir uma implementação que compartilhe dados.

Por fim, o padrão Memento, também graças à 
imutabilidade, não precisaria se preocupar quanto 
a expor o estado interno de um valor. Assim, é 
possível gerar \textit{snapshots} apenas copiando 
valores anteriores, sem preocupações adicionais. 

\section{Avaliação dos Resultados}
% É importante descrever quais resultados são mais
% relevantes e como eles contribuem com a área 
% de estudo
%\section{Avaliação}

Como pôde ser visto, a aplicação dos 
conceitos de programação funcional pode 
contribuir para a solução de alguns dos padrões. 
Com o uso de funções de alta ordem, padrões 
que apresentam interfaces com poucas funções, como o 
Template Method, ou muitas funções 
não necessariamente relacionadas, como 
o Builder, poderiam aproveitar-se do 
uso de funções de alta ordem para diminuir a 
quantidade de classes definidas pela 
aplicação. Para os casos em que as funções 
são relacionadas e agrupadas em 
novos valores como no padrão State, 
o comportamento final torna-se semelhante 
ao de uma interface, onde são declarados 
tipos que armazenam funções e criados 
valores que definem quais funções são 
equivalentes àquele tipo. 
O diferencial é que 
é mais simples reaproveitar uma mesma 
função para valores diferentes, já que 
em um contexto orientado a objetos seria 
necessário definir classes abstratas 
para definir um comportamento compartilhado 
por várias classes que implementem uma 
mesma interface.
A última abordagem, usando closures, também 
demonstrou flexibilidade e possibilidade 
de reuso, como visto no padrão Interpreter. 
Além de permitir a criação de novas funções, 
é possível parametrizá-las através de 
outras funções.

O uso de conceitos de programação funcional 
para resolver alguns dos problemas propostos 
pelos padrões não é algo novo, como foi visto 
pelos padrões que já possuíam soluções alternativas 
conhecidas. As funções \textit{map} e \textit{reduce} 
usadas como alternativa para o Iterator já 
são implementadas em diversas linguagens funcionais
\cite{realworldhaskell,wampler2021,braveclojure}, 
enquanto o conceito de programação reativa, 
um tipo de Observer, já é 
utilizado em \textit{frameworks} de  
\textit{single page application}\cite{rxjs}. 
O recurso de \textit{pattern matching} 
utilizado na análise do Visitor também já 
é um conceito conhecido como alternativa 
aos multimétodos\cite{patternmatchingvisitor}.

Em seguida, com os padrões sem diferenças 
relevantes de implementação, pôde ser visto 
que nem todos os padrões utilizam abordagens 
exclusivamente orientadas a objetos. Dessa forma, 
transpô-los para o contexto funcional 
permite que a mesma solução seja reaproveitada. 
Apesar de terem sido apresentadas algumas 
ressalvas, elas não estão diretamente ligadas 
à solução proposta, mas às consequências de 
se utilizá-la, como é o caso do Mediator 
precisar que a função cliente gerencie o 
estado dos \textit{colleagues}. 

Por fim, foi visto o grupo dos padrões 
que não fazem sentido no contexto funcional, 
já que os problemas propostos não existem. 
Esse foi o caso dos padrões Prototype e Memento.
Outro caso desse grupo foi a existência de 
um padrão cuja intenção viola algum conceito da 
programação funcional, o Singleton. 
É relevante informar que o padrão Singleton 
pode ser considerado um antipadrão que poderia 
ser substituído pela injeção de dependência.
\cite{singletonantipattern}. Isso poderia 
ser um indício de que a aplicação de 
conceitos funcionais contribui para 
identificar problemas de \textit{design} de 
\textit{software}, mas como o Singleton 
foi o único dos vinte e três padrões que 
apresentou esse comportamento, faltam 
evidências para considerar essa 
possibilidade.

% Um bom embasamento teórico é fundamental para 
% dar credibilidade aos resultados da pesquisa 
% Portanto, o investigador deve comparar as 
% ideais de outros autores com a descoberta do 
% seu trabalho
%\section{Comparação com a literatura}

% Quando a coleta de dados é representativa, 
% o pesquisador pode fazer generalizações
%\section{Generalização}

%% ---
% Conclusão
% ---
\chapter{Conclusão}
% ---

% *Tema e relevância do trabalho.*

Este trabalho trouxe uma análise de como os 
problemas resolvidos pelos padrões de projeto 
\textit{Gang of Four} podem ser solucionados no 
contexto de programação funcional. Essa abordagem 
pode contribuir tanto para desenvolvedores que 
programam em linguagens funcionais quanto para a 
melhoria de soluções de problemas já conhecidos 
no desenvolvimento de \textit{software}.

% *Cumprimento dos objetivos.*
% - objetivo: verificar como os padrões 
% se comportam no contexto funcional

Foi possível analisar cada um dos vinte e 
três padrões de projeto \textit{Gang of Four} 
no contexto funcional, extraindo o problema 
solucionado pelo padrão e aplicando conceitos 
de programação funcional para solucioná-lo de 
forma alternativa.

% *Resultados obtidos.*
% - alguns padrões se comportam melhor
% - alguns padrões não possuem diferenças
% - alguns padrões não existem

Através dessa análise foi verificado que 
alguns conceitos de programação funcional, 
principalmente as funções de alta 
ordem, podem servir como alternativa às soluções 
propostas por alguns dos padrões. Também foi 
visto que alguns dos problemas catalogados 
sequer existem no contexto funcional. 
Entretanto, alguns padrões apresentaram 
dificuldades de implementação durante a 
análise e outros não apresentaram diferenças 
significativas.

% *Verificação da hipótese.*
% - qual seria a hipótese? Contexto funcional 
% contribui pros padrões? Caso sim, a resposta 
% é não necessariamente

Com isso, fica evidente que nem sempre 
a solução funcional do problema de um 
padrão trará um bom resultado. Alguns 
conceitos restritivos, como a imutabilidade 
e as funções puras, impossibilitam 
o uso de algumas abordagens utilizadas 
pelos padrões para solucionar os problemas. 
Ainda assim, outras análises demonstraram 
que o uso de conceitos funcionais para 
padrões que não dependem dessas restrições 
podem servir como alternativa. 


% *Resposta ao problema de pesquisa.*
% - Sim, padrões de projeto podem ser 
% aplicáveis no contexto funcional

Dessa forma, é possível concluir que a maior 
parte dos problemas solucionados pelos 
padrões de projeto \textit{Gang of Four} 
existem e são solucionáveis no contexto 
funcional. Os problemas que não existem 
dependem de ou são resolvidos por conceitos 
restritivos do paradigma funcional. Já 
os problemas sem diferenças de solução não 
dependem de conceitos relacionados ao 
paradigma orientado a objetos nem ao 
paradigma funcional.

% *Melhorias e direcionamentos futuros. *
% - misturar os dois paradigmas para tirar 
% o melhor de cada
% - linguagens orientadas a objeto que 
% já suportam funções de alta ordem 
% e estruturas de dados imutáveis

Por fim, trabalhos futuros podem se aproveitar 
da análise realizada para verificar como 
é possível utilizar os conceitos de 
programação funcional em conjunto com os 
conceitos de orientação a objetos na 
solução dos padrões. Como a linguagem Scala 
permite o uso de ambos os paradigmas, 
é possível que uma nova 
análise dos vinte e três padrões seja feita 
sem ater-se às restrições impostas pela 
programação funcional. 

% ----------------------------------------------------------
% Finaliza a parte no bookmark do PDF
% para que se inicie o bookmark na raiz
% e adiciona espaço de parte no Sumário
% ----------------------------------------------------------
\phantompart


% ----------------------------------------------------------
% ELEMENTOS PÓS-TEXTUAIS
% ----------------------------------------------------------
\postextual
% ----------------------------------------------------------

% ----------------------------------------------------------
% Referências bibliográficas
% ----------------------------------------------------------
\bibliography{functionaldesignpatterns-references}





%---------------------------------------------------------------------
% INDICE REMISSIVO
%---------------------------------------------------------------------
\phantompart
\printindex
%---------------------------------------------------------------------

\end{document}
