% ---
% Conclusão
% ---
\chapter{Conclusão}
% ---

% *Tema e relevância do trabalho.*

Este trabalho trouxe uma análise de como os 
problemas resolvidos pelos padrões de projeto 
\textit{Gang of Four} podem ser solucionados no 
contexto de programação funcional. Essa abordagem 
pode contribuir tanto para desenvolvedores que 
programam em linguagens funcionais quanto para a 
melhoria de soluções de problemas já conhecidos 
no desenvolvimento de software.

% *Cumprimento dos objetivos.*
% - objetivo: verificar como os padrões 
% se comportam no contexto funcional

Foi possível analisar cada um dos vinte e 
três padrões de projeto \textit{Gang of Four} 
no contexto funcional, extraindo o problema 
solucionado pelo padrão e aplicando conceitos 
de programação funcional para solucioná-lo de 
forma alternativa.

% *Resultados obtidos.*
% - alguns padrões se comportam melhor
% - alguns padrões não possuem diferenças
% - alguns padrões não existem

Através dessa análise foi verificado que 
alguns conceitos de programação funcional, 
principalmente as funções de alta 
ordem, podem servir como alternativa às soluções 
propostas por alguns dos padrões. Também foi 
visto que alguns dos problemas catalogados 
sequer existem no contexto funcional. 
Entretanto, alguns padrões apresentaram 
dificuldades de implementação durante a 
análise e outros não apresentaram diferenças 
significativas.

% *Verificação da hipótese.*
% - qual seria a hipótese? Contexto funcional 
% contribui pros padrões? Caso sim, a resposta 
% é não necessariamente

Com isso, fica evidente que nem sempre 
a solução funcional do problema de um 
padrão trará um bom resultado. Alguns 
conceitos restritivos, como a imutabilidade 
e as funções puras, impossibilitam 
o uso de algumas abordagens utilizadas 
pelos padrões para solucionar os problemas. 
Ainda assim, outras análises demonstraram 
que o uso de conceitos funcionais para 
padrões que não dependem dessas restrições 
podem servir como alternativa. 


% *Resposta ao problema de pesquisa.*
% - Sim, padrões de projeto podem ser 
% aplicáveis no contexto funcional

Dessa forma, é possível concluir que a maior 
parte dos problemas solucionados pelos 
padrões de projeto \textit{Gang of Four} 
existem e são solucionáveis no contexto 
funcional. Os problemas que não existem 
dependem de ou são resolvidos por conceitos 
restritivos do paradigma funcional. Já 
os problemas sem diferenças de solução não 
dependem de conceitos relacionados ao 
paradigma orientado a objetos nem ao 
paradigma funcional.

% *Melhorias e direcionamentos futuros. *
% - misturar os dois paradigmas para tirar 
% o melhor de cada
% - linguagens orientadas a objeto que 
% já suportam funções de alta ordem 
% e estruturas de dados imutáveis

Por fim, trabalhos futuros podem aproveitar-se 
da análise realizada para verificar como 
é possível utilizar os conceitos de 
programação funcional em conjunto com os 
conceitos de orientação a objetos na 
solução dos padrões. Como a linguagem Scala 
permite o uso de ambos os paradigmas, 
é possível que uma nova 
análise dos vinte e três padrões seja feita 
sem ater-se às restrições impostas pela 
programação funcional. 