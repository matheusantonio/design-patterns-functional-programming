% ---
% Conclusão
% ---
\chapter{Conclusão}
% ---

% *Tema e relevância do trabalho.*

Este trabalho trouxe uma análise de como os 
padrões de projeto \textit{Gang of Four} 
podem comportar-se no contexto de programação 
funcional. Essa abordagem é útil tanto 
para desenvolvedores que utilizam programação 
funcional quanto para contribuir para a 
melhoria da solução de problemas já 
conhecidos no desenvolvimento de software.

% *Cumprimento dos objetivos.*
% - objetivo: verificar como os padrões 
% se comportam no contexto funcional

% Talvez essa seção esteja incluída na anterior

% *Resultados obtidos.*
% - alguns padrões se comportam melhor
% - alguns padrões não possuem diferenças
% - alguns padrões não existem

Foi verificado que o contexto funcional, 
principalmente através das funções de alta 
ordem, pode contribuir para as soluções dos 
padrões. Além disso, alguns dos problemas 
propostos sequer existem no contexto funcional. 
Entretanto, alguns padrões apresentaram 
dificuldades de implementação graças a 
conceitos mais restritivos, como a 
imutabilidade.

% *Verificação da hipótese.*
% - qual seria a hipótese? Contexto funcional 
% contribui pros padrões? Caso sim, a resposta 
% é não necessariamente



% *Resposta ao problema de pesquisa.*
% - Sim, padrões de projeto podem ser 
% aplicáveis no contexto funcional


% *Melhorias e direcionamentos futuros. *
% - misturar os dois paradigmas para tirar 
% o melhor de cada
% - linguagens orientadas a objeto que 
% já suportam funções de alta ordem 
% e estruturas de dados imutáveis