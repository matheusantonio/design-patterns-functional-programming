% [Esboço de introdução]
% [resumir o que são padrões de projeto e como eles são encontrados no dia a dia
% mencionar GOF
% introduzir o conceito de programação funcional, como é uma abordagem útil para o desenvolvimento e como não há nenhum tipo de catálogo de padrões ou de problemas] {apresentação do tema}
% [Explicar as abordagens (padrões OO que podem ser aproveitados, padrões OO que não são necessários e padrões relacionados a PF)] {delimitação do tema}
% [ ? talvez parte disso esteja sendo abordado no item anterior] {apresentar hipóteses}
% [verificar que tipos de problemas são reaproveitáveis entre os dois paradigmas através dos padrões e que novos problemas surgem através de uma análise dos recursos da linguagem, verificar se existe relação entre isso e a classificação dos padrões (criacional, comportamental, estrutural)] {objetivos e metodologia}
% [esclarecer como isso é útil para contribuir para a evolução de linguagens que não possuem certos recursos que tornam necessários padrões complexos] {relevância da pesquisa}
% [explicar o que é PF → explicar o que são padrões de projeto → abordar cada padrão GOF no contexto funcional → Abordar possíveis padrões vindos do paradigma funcional → mostrar os resultados obtidos nas análises → concluir] {estrutura do trabalho}



% ----------------------------------------------------------
% Introdução (exemplo de capítulo sem numeração, mas presente no Sumário)
% ----------------------------------------------------------
\chapter{Introdução}
% ----------------------------------------------------------


Durante o processo de construção de um software diversos problemas de design são 
enfrentados, alguns mais simples, outros mais trabalhosos. Alguns desses problemas 
são tão comuns que achou-se necessário definir um padrão de solução para eles, 
reduzindo o tempo que desenvolvedores que passariam pelo mesmo problema futuramente 
gastariam tentando chegar até a mesma solução que outros desenvolvedores chegaram 
no passado. Essa ideia deu origem ao que chamamos de Padrões de Projeto, soluções 
ideais para problemas comuns ou difíceis de se resolver no desenvolvimento de software.
Alguns desses problemas deram origem a padrões tão comuns que quatro desenvolvedores, 
conhecidos como Gang of Four, reuniram-se para catalogar esses padrões, dando 
origem aos vinte e três Padrões de Projeto GOF.
Entretanto, esses padrões comuns são voltados para um paradigma de programação 
tão comum quanto: o Orientado a Objetos. Sendo um paradigma mais conhecido através 
de linguagens de programação famosas como Java, normalmente são esses os padrões 
aprendidos pelos estudantes ou desenvolvedores comuns. O problema é que a 
Orientação a Objetos não é o único paradigma 