% [Esboço de introdução]
% [resumir o que são padrões de projeto e como eles são encontrados no dia a dia
% mencionar GOF
% introduzir o conceito de programação funcional, como é uma abordagem útil para o desenvolvimento e como não há nenhum tipo de catálogo de padrões ou de problemas] {apresentação do tema}
% [Explicar as abordagens (padrões OO que podem ser aproveitados, padrões OO que não são necessários e padrões relacionados a PF)] {delimitação do tema}
% [ ? talvez parte disso esteja sendo abordado no item anterior] {apresentar hipóteses}
% [verificar que tipos de problemas são reaproveitáveis entre os dois paradigmas através dos padrões e que novos problemas surgem através de uma análise dos recursos da linguagem, verificar se existe relação entre isso e a classificação dos padrões (criacional, comportamental, estrutural)] {objetivos e metodologia}
% [esclarecer como isso é útil para contribuir para a evolução de linguagens que não possuem certos recursos que tornam necessários padrões complexos] {relevância da pesquisa}
% [explicar o que é PF → explicar o que são padrões de projeto → abordar cada padrão GOF no contexto funcional → Abordar possíveis padrões vindos do paradigma funcional → mostrar os resultados obtidos nas análises → concluir] {estrutura do trabalho}

% ----------------------------------------------------------
% Introdução (exemplo de capítulo sem numeração, mas presente no Sumário)
% ----------------------------------------------------------
\chapter{Introdução}
% ----------------------------------------------------------
%
%contextualização

Enquanto Alan Turing definia o que viria a 
se tornar a Máquina de Turing, Alonzo Church 
trabalhava em uma abordagem diferente: o Cálculo Lambda
\cite{church1932set,church1936unsolvable,sep-turing-machine}. 
O primeiro baseia-se em modificação do estado em 
uma fita, enquanto o segundo abordava a aplicação 
de funções. Apesar do princípio por trás de ambos 
ser bem diferente, sua aplicação é computacionalmente 
equivalente, permitindo que as duas ideias possam 
ser usadas para resolver os mesmos problemas
\cite{sep-church-turing}.

%delimitação do tema

Durante o processo de construção de um 
software, podem surgir problemas de \textit{design} 
recorrentes, sendo alguns desses problemas 
tão comuns que tornaram necessário definir soluções 
reutilizáveis para os mesmos, os padrões 
de projeto. Por o paradigma 
de programação orientado a objetos ser o mais 
popular e mais utilizado no mercado de trabalho, 
existem catálogos de padrões de projetos voltados 
para o desenvolvimento orientado a objetos. 
Entretanto, existem outros paradigmas de 
programação, como a programação funcional, 
que apresenta uma abordagem diferente 
do desenvolvimento de software.

%problema de pesquisa

O paradigma orientado a objetos possui uma 
abordagem mais equivalente à máquina de Turing, 
enquanto o paradigma funcional 
tem como origem o próprio cálculo lambda. 
Dessa forma, é possível que os padrões de 
projeto desenvolvidos com base no paradigma 
orientado a objetos também sejam aplicáveis 
durante o desenvolvimento no contexto funcional. 

%objetivos gerais e específicos do trabalho

Com isso, este trabalho tem como objetivo 
analisar um conjunto de padrões de projeto 
conhecidos utilizando conceitos de programação 
funcinal. A intenção é verificar se os 
problemas mencionados pelo padrão existem 
em um contexto funcional e quais 
seriam as consequências de implementá-lo 
levando em consideração tanto as vantagens 
quanto as limitações desse paradigma.

%justificativa das escolhas

O conjunto de padrões de projeto escolhido 
para a análise é composto dos vinte e três 
padrões de projeto \textit{Gang of Four}. 
Esses padrões foram catalogados em um livro 
por quatro desenvolvedores e representam 
um conjunto conhecido de padrões de projeto 
de desenvolvimento orientado a objetos. 
Já a linguagem funcional escolhida é a 
linguagem Scala. Essa linguagem 
apresenta tanto conceitos de orientação 
a objetos quanto de programação funcional, 
o que permite que os exemplos em código 
apresentados no trabalho sejam feitos 
utilizando a mesma linguagem, facilitando 
a assimilação dos mesmos por parte do 
leitor. Entretanto, ela permite tanto 
escrever código apenas orientado a 
objetos quanto código apenas funcional, 
o que permite que os exemplos em 
código não misturem ambos os conceitos 
quando não desejado.

%metodologia



%estrutura de capitulos

\begin{comment}
    
Na parte de desenvolvimento, haverá uma introdução 
sobre como alguns conceitos de orientação a objetos, 
como classes e encapsulamento, podem ser representados 
em uma 
linguagem funcional. Em seguida, os capítulos serão 
divididos entre padrões criacionais, estruturais e 
comportamentais. Ao todo, os vinte e três padrões 
GoF serão abordados nesses três capítulos, onde serão 
apresentadas as ideias básicas do problema que o padrão 
busca resolver e como o resolve, seguido da abordagem 
funcional de resolver o mesmo problema.
    
Após analisar todos os padrões, o capítulo de resultados 
analisará as vantagens e desvantagens da abordagem 
funcional para cada solução, destacando onde ela 
contribuiu, onde atrapalhou, ou onde não fazia sentido ser 
implementada. Essas classificações dependerão das análises 
que serão realizadas na etapa de desenvolvimento.
    
Por fim, no capítulo de conclusão serão analisadas 
as consequências dessas análises e como elas podem 
impactar o conceito de padrões de projeto e as vantagens 
e desvantagens de revisá-los no ponto de vista funcional.
    
\end{comment}


