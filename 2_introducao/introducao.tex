% [Esboço de introdução]
% [resumir o que são padrões de projeto e como eles são encontrados no dia a dia
% mencionar GOF
% introduzir o conceito de programação funcional, como é uma abordagem útil para o desenvolvimento e como não há nenhum tipo de catálogo de padrões ou de problemas] {apresentação do tema}
% [Explicar as abordagens (padrões OO que podem ser aproveitados, padrões OO que não são necessários e padrões relacionados a PF)] {delimitação do tema}
% [ ? talvez parte disso esteja sendo abordado no item anterior] {apresentar hipóteses}
% [verificar que tipos de problemas são reaproveitáveis entre os dois paradigmas através dos padrões e que novos problemas surgem através de uma análise dos recursos da linguagem, verificar se existe relação entre isso e a classificação dos padrões (criacional, comportamental, estrutural)] {objetivos e metodologia}
% [esclarecer como isso é útil para contribuir para a evolução de linguagens que não possuem certos recursos que tornam necessários padrões complexos] {relevância da pesquisa}
% [explicar o que é PF → explicar o que são padrões de projeto → abordar cada padrão GOF no contexto funcional → Abordar possíveis padrões vindos do paradigma funcional → mostrar os resultados obtidos nas análises → concluir] {estrutura do trabalho}

% ----------------------------------------------------------
% Introdução (exemplo de capítulo sem numeração, mas presente no Sumário)
% ----------------------------------------------------------
\chapter{Introdução}
% ----------------------------------------------------------
%

%contextualização

Enquanto Alan Turing definia o que viria a 
se tornar a Máquina de Turing, Alonzo Church 
trabalhava em uma abordagem diferente: o Cálculo Lambda
\cite{church1932set,church1936unsolvable,sep-turing-machine}. 
O primeiro baseia-se em modificação do estado de 
uma fita enquanto o segundo aborda a aplicação 
de funções. Apesar do princípio por trás de ambos 
ser diferente, sua aplicação é computacionalmente 
equivalente, permitindo que as duas ideias possam 
ser usadas para resolver os mesmos problemas
\cite{sep-church-turing}.

%delimitação do tema

Entre os problemas que podem surgir no 
processo de construção de um \textit{software} 
estão os problemas de \textit{design}. Alguns 
deles são tão recorrentes que tornou-se 
necessário definir soluções reutilizáveis 
para os mesmos. Essas soluções são conhecidas 
como padrões de projeto de \textit{software}. 
\cite{gamma:1995} O conceito de padrões de 
projeto tem sua origem na arquitetura, a 
partir do livro \textit{A Pattern Language} 
de Christopher Alexander\cite{alexanderpatternlanguage}. 
A ideia se difundiu no desenvolvimento 
de \textit{software} a partir do catálogo de 
padrões \textit{Gang of Four} e dos padrões 
catalogados por Martin Fowler no livro 
\textit{Patterns of Enterprise Application Architecture}.
\cite{gamma:1995,fowler2002eea}

%problema de pesquisa

Os catálogos de padrões mencionados são 
voltados para o desenvolvimento de \textit{software} 
orientado a objetos.\cite{gamma:1995} Entretanto, existem 
outros paradigmas de programação, como 
a programação funcional, que apresentam 
uma abordagem diferente de desenvolvimento 
de software.
Como o paradigma orientado a objetos possui uma 
abordagem associável à da máquina de Turing, onde o 
primeiro  
busca modificar o estado de uma fita enquanto o segundo 
busca modificar o estado de um objeto, e o paradigma 
funcional tem como origem o cálculo lambda
\cite{michaelson:2011}, pode ser possível que os problemas 
resolvidos pelos padrões de projeto desenvolvidos 
com base no paradigma orientado a objetos também 
sejam solucionáveis durante o desenvolvimento 
no contexto funcional. 

%objetivos gerais e específicos do trabalho

Com isso, este trabalho tem como objetivo 
analisar um conjunto de padrões de projeto 
conhecidos utilizando conceitos de programação 
funcional. A intenção é verificar se os 
problemas mencionados pelo padrão existem 
em um contexto funcional e quais 
seriam as consequências de implementá-lo 
levando em consideração tanto as vantagens 
quanto as limitações desse paradigma.

%justificativa das escolhas

O conjunto de padrões de projeto escolhido 
para a análise é composto dos vinte e três 
padrões de projeto \textit{Gang of Four}. 
Esses padrões foram catalogados em um livro 
por quatro desenvolvedores e representam 
um conjunto conhecido de padrões de projeto 
de desenvolvimento orientado a objetos. \cite{gamma:1995}
Já a linguagem funcional escolhida é a 
linguagem Scala. Essa linguagem 
apresenta tanto conceitos de orientação 
a objetos quanto de programação funcional, 
o que permite que os exemplos em código 
apresentados no trabalho sejam feitos 
utilizando sempre a mesma linguagem, 
facilitando o entendimento.\cite{wampler2021}
Apesar de permitir a mistura dos 
paradigmas, a linguagem também permite  
escrever código apenas orientado a 
objetos ou código apenas funcional, 
o que permite que os exemplos em 
código não misturem ambos os conceitos 
quando não desejado.

%metodologia

Durante a análise dos padrões será 
fornecida uma breve descrição do problema 
resolvido, assim como um exemplo de aplicação 
do padrão. A descrição será acompanhada de 
um diagrama de classes que demonstra a 
estrutura do padrão, enquanto o exemplo 
trará um diagrama de classes com a estrutura 
adaptada ao contexto do exemplo. No exemplo 
também será apresentada a implementação 
do padrão na linguagem Scala. Por fim, será 
analisado o problema resolvido pelo padrão 
no contexto do paradigma de programação 
funcional, com foco principal no problema 
e não na implementação do padrão, já que 
a intenção da análise não é reimplementar os 
conceitos de orientação a objetos em uma 
linguagem funcional. 

%estrutura de capitulos

Este trabalho será dividido em quatro partes. 
A primeira, introdutória, é composta deste capítulo 
e do capítulo dois, que traz trabalhos relacionados 
ao tema proposto. A parte dois, de conceitos 
básicos, tem início com o capítulo três, que é dedicado 
à programação funcional e apresentará os principais 
conceitos do paradigma. O capítulo quatro demonstrará a 
estrutura de um padrão de projeto como é apresentada 
no livro \textit{Gang of Four}, enquanto o capítulo 
cinco discorrerá sobre alguns conceitos importantes 
da linguagem Scala para leitores não familiarizados 
com a sintaxe. A parte três é composta do desenvolvimento 
do trabalho, onde o capítulo seis apresenta um 
mapeamento de conceitos do paradigma orientado a 
objetos para o funcional. Já os capítulos sete, 
oito e nove apresentam as análises dos vinte e 
três padrões de projeto, estando organizadas 
de forma a separar os padrões criacionais, 
estruturais e comportamentais, uma divisão 
abordada no livro \textit{Gang of Four} que será 
explicada nos conceitos básicos. Por fim, 
a parte quatro apresenta os resultados do 
trabalho, iniciando com o capítulo dez, 
que agrupará, analisará e avaliará os 
resultados encontrados na seção de 
desenvolvimento. O último capítulo, 
o capítulo onze, apresentará a conclusão 
do trabalho.