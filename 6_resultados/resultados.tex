% ---
% Resultados
% ---
\chapter{Resultados}
% ---

\section{Padrões resolvidos por funções de alta ordem}

Entre as soluções vistas, a que é aplicada na 
maior parte dos padrões é o uso de funções de alta 
ordem como alternativa a interfaces ou classes. 


\section{Padrões com soluções alternativas}

Os padrões Iterator, Observer e Visitor possuem 
implementações alternativas que não necessariamente 
seguem à risca a ideia do padrão. No caso do Iterator, 
existe a diferença entre o gerencimaneto por parte 
do cliente e por parte da coleção entre as 
alternativas orientada a objetos e funcional. No 
caso do Observer, o conceito no qual ele se 
baseia - programação reativa - é levado em 
consideração ao substituí-lo pela programação 
reativa funcional. Já o Visitor, apesar de também 
aproveitar-se de funções de alta ordem, na verdade 
é resolvido pelo recurso \textit{pattern matching}, 
que não é um conceito exclusivamente funcional, 
porém costuma ser implementado em linguagens 
funcionais. Nesse caso, o padrão não foi 
resolvido por programação funcional em si, 
mas sim por uma alternativa próxima. 

\section{Padrões sem diferenças relevantes}

Da mesma forma que o padrão Façade orientado 
a objetos baseia-se no acesso entre as classes, 
a implementação funcional baseia-se no acesso 
entre módulos. Como ambas as ideias são 
análogas, a implementação do padrão na verdade 
não possui mudanças significativas. 

Da mesma forma, o padrão Flyweight, tanto no 
contexto orientado a objetos quanto no funcional, 
é implementado através de memoização. Já o Mediator 
baseia-se em possuir uma função (ou classe) que 
gerencia as dependências entre valores (ou objetos). 
No caso desses dois padrões, ambos possuem 
pequenas diferenças como consequência de suas 
implementações - por exemplo, não existem os dois 
tipos de Flyweight intrínseco ou extrínseco 
graças à imutabilidade e há a necessidade do 
cliente gerenciar a mudança de estado dos 
\textit{colleagues} no Mediator -, mas a ideia 
por trás da implementação de ambos é análoga 
à versão orientada a objetos.

\section{Padrões que não fazem sentido no contexto funcional}


